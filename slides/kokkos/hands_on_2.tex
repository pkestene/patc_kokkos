%%%%%%%%%%%%%%%%%%%%%%%%%%%%%%%%%%%%%%%%%%%%%%%%%%%%%%%%%%%%%%%%%%%%%%%% 
%%%%%%%%%%%%%%%%%%%%%%%%%%%%%%%%%%%%%%%%%%%%%%%%%%%%%%%%%%%%%%%%%%%%%%%% 
\begin{frame}[fragile=singleslide]
  \frametitle{Hands-On 2 : SAXPY}

  {\large \textcolor{red}{\textbf{Purpose:}} The simplest computing kernel in Kokkos, importance of hwloc}

  \begin{itemize}
  \item There 5 differents versions
  \item \textbf{1. Serial : no Kokkos)}
  \item \textbf{2. OpenMP : no Kokkos)}
  \item 3. Kokkos-Lambda-CPU : Kokkos with lambda for threads dispatch
  \item \textbf{4. Kokkos-Lambda : Kokkos with lambda for threads dispatch and data buffer (Kokkos::View)}
  \item 5. Kokkos-Functor-CPU : Kokkos with functor for threads dispatch only
  \end{itemize}
  
  \begin{itemize}
  %\item \textbf{Proposed activity}
  \item \textcolor{red}{\textbf{Saxpy serial (reference executable on Power8)}}
    \begin{itemize}
    \item \texttt{cd \$HOME/kokkos-tutorial/kokkos-tutorials/1-Day-Tutorial/Exercises/01\_AXPY/Serial}
    \item \texttt{make KOKKOS\_ARCH=Power8}
    \item Alternatively, we could have modify \texttt{Makefile} and change \texttt{SNB} into \texttt{Power8}
    \end{itemize}
  \item \textcolor{orange}{\textbf{Saxpy regular OpenMP (on Power8)}}
    \begin{itemize}
    \item \texttt{cd \$HOME/kokkos-tutorial/kokkos-tutorials/1-Day-Tutorial/Exercises/01\_AXPY/OpenMP}
    \item Rebuild: \texttt{make KOKKOS\_ARCH=Power8}; and observe performance
    \end{itemize}
  \end{itemize}

  \textbf{see also slides from SC2016, page 42(74).}
  
\end{frame}

%%%%%%%%%%%%%%%%%%%%%%%%%%%%%%%%%%%%%%%%%%%%%%%%%%%%%%%%%%%%%%%%%%%%%%%% 
%%%%%%%%%%%%%%%%%%%%%%%%%%%%%%%%%%%%%%%%%%%%%%%%%%%%%%%%%%%%%%%%%%%%%%%% 
\begin{frame}[fragile=singleslide]
  \frametitle{Hands-On 2 : SAXPY}

  \begin{itemize}
  \item \textcolor{violet}{\textbf{Saxpy Kokkos OpenMP (on Power8)}}    
    \begin{itemize}
    \item \texttt{cd \$HOME/kokkos-tutorial/kokkos-tutorials/1-Day-Tutorial/Exercises/01\_AXPY/Kokkos-Lambda}
    \item Add 3 lines in \texttt{saxpy.cpp} right after Kokkos initialization
      \begin{minted}{c++}
        std::ostringstream msg;
        Kokkos::OpenMP::print_configuration( msg );
        std::cout << msg.str();
      \end{minted}
    \item \texttt{make KOKKOS\_ARCH=Power8}
    \item Make sure all available CPU cores were used ($1\times 160 \times 1$)
    \item Change the number of OpenMP threads created by kokkos, e.g. :\\
      \texttt{./saxpy.host  --threads=20}
    \item Add again \texttt{KOKKOS\_USE\_TPLS="hwloc"} on the command line\\
      Rebuild and rerun, you should see that application uses \textbf{all the available numa domains}, and a strongly increased bandwidth usage !
    \end{itemize}
  \end{itemize}

\end{frame}

%%%%%%%%%%%%%%%%%%%%%%%%%%%%%%%%%%%%%%%%%%%%%%%%%%%%%%%%%%%%%%%%%%%%%%%% 
%%%%%%%%%%%%%%%%%%%%%%%%%%%%%%%%%%%%%%%%%%%%%%%%%%%%%%%%%%%%%%%%%%%%%%%% 
\begin{frame}[fragile=singleslide]
  \frametitle{Hands-On 2 : SAXPY}

  \begin{itemize}
  \item \textcolor{darkgreen}{\textbf{Saxpy CUDA (on Power8 + Nvidia K80/P100)}}
    \begin{itemize}
    \item \texttt{cd \$HOME/kokkos-tutorial/kokkos-tutorials/1-Day-Tutorial/Exercises/01\_AXPY/Kokkos-Lambda}
    \item \texttt{module load cuda/8.0}
    \end{itemize}
  \item Rebuild for K80, run on ouessant (front node):\\
    \texttt{make KOKKOS\_DEVICES="Cuda,OpenMP" KOKKOS\_ARCH="Kepler37,Power8" KOKKOS\_USE\_TPLS="hwloc"}
  \item Rebuild for P100, run on compute node using \texttt{submit\_ouessant.sh} (should see a strong difference):\\
    \texttt{make KOKKOS\_DEVICES="Cuda,OpenMP" KOKKOS\_ARCH="Pascal60,Power8" KOKKOS\_USE\_TPLS="hwloc"}\\
    Please note that \textbf{maximun bandwith is 732 GB/s for Pascal P100}, you can retrieve this number by examining \texttt{deviceQuery} example in CUDA/SDK.
  \end{itemize}
\end{frame}


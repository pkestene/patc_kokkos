%%%%%%%%%%%%%%%%%%%%%%%%%%%%%%%%%%%%%%%%%%%%%%%%%%%%%%%%%%%%%%%%%%%%%%%% 
%%%%%%%%%%%%%%%%%%%%%%%%%%%%%%%%%%%%%%%%%%%%%%%%%%%%%%%%%%%%%%%%%%%%%%%% 
\begin{frame}[fragile=singleslide]
  \frametitle{Kokkos data Container (1)}
  
  {\large \textcolor{blue}{\texttt{Kokkos::View<...>}} is \textbf{multidimensionnal data container} with \textbf{hardware adapted memory layout}}
  \begin{itemize}
  \item \textcolor{blue}{\texttt{Kokkos::View<double **> data("data",NX,NY);}} : 2D array with sizes known at runtime
  \item \textcolor{blue}{\texttt{Kokkos::View<double *[3]> data("data",NX);}} : 2D array with first size known at runtime ($NX$), and second known at compile time (3).
  \item \fbox{How do I access data ? $data(i,j)$ !} {\it a la Fortran}
  \item \textcolor{orange}{\bf Which memory space ?} By default, the default device memory space !\\
    Want to enforce in which memory space lives the view ? 
    \textcolor{blue}{\texttt{Kokkos::View<..., Device>}}: if a second template parameter is given, Kokkos expects a \texttt{Device} (e.g. \texttt{Kokkos::OpenMP}, \texttt{Kokkos::Cuda}, ...)
  \item \texttt{Kokkos::View} are \textbf{small}, designed as reference to allocated memory buffer
    \begin{itemize}
    \item \textcolor{orange}{View = pointer to data + metadata(array shape, layout, ... )}
    \item assignment is fast (shallow copy + increment ref counter)~\footnote{NB: same behaviour as in python for example}
    \end{itemize}
  \item \texttt{Kokkos::View} are designed to be pass by value to a function ({\bf no hard copy}).
  \end{itemize}

\end{frame}

%%%%%%%%%%%%%%%%%%%%%%%%%%%%%%%%%%%%%%%%%%%%%%%%%%%%%%%%%%%%%%%%%%%%%%%% 
%%%%%%%%%%%%%%%%%%%%%%%%%%%%%%%%%%%%%%%%%%%%%%%%%%%%%%%%%%%%%%%%%%%%%%%% 
\begin{frame}[fragile=singleslide]
  \frametitle{Kokkos data Container (2)}

  \begin{itemize}
  \item Concept of \textbf{memory layout:}
  \item \textbf{Memory layout is crucial for performance:}
    \begin{itemize}
    \item \textcolor{blue}{\textbf{LayoutLeft}}: $data(i,j,k)$ uses linearized index as $i + NX*j + NX*NY * k$ (column-major order)
    \item \textcolor{red}{\textbf{LayoutRight}}: $data(i,j,k)$ uses linearized index as $k + NZ*j + NZ*NY * i$ (raw-major order)
    \end{itemize}
  \item \textcolor{red}{\texttt{Kokkos::View<int**, Kokkos::OpenMP>} defaults with LayoutRight}; a single thread access contiguous entries of the array. Better for cache and avoid sharing cache lines between threads.
  \item \textcolor{blue}{\texttt{Kokkos::View<int**, Kokkos::Cuda>} defaults LayoutLeft} so that consecutive threads in the same warp access consecutive entries in memory; try to ensure memory coalescence constraint
  \item You can if you like, still enforce memory layout yourself (or just use 1D Views, and compute index yourself);\\
    We will see the 2 possibilities with the miniApp on the Fisher equation
  \end{itemize}

\end{frame}

%%%%%%%%%%%%%%%%%%%%%%%%%%%%%%%%%%%%%%%%%%%%%%%%%%%%%%%%%%%%%%%%%%%%%%%% 
%%%%%%%%%%%%%%%%%%%%%%%%%%%%%%%%%%%%%%%%%%%%%%%%%%%%%%%%%%%%%%%%%%%%%%%% 
\begin{frame}[fragile=singleslide]
  \frametitle{Kokkos data Container (3)}

  \begin{itemize}
  \item \textcolor{blue}{\texttt{Kokkos::View<...>}} are reference-counted
  \item \textcolor{orange}{\textbf{shallow copy}} is default behavior
    \begin{minted}{c++}
      Kokkos::View<int *>("a",10);
      Kokkos::View<int *>("b",10);
      a = b; // a now points to b (ref counter incremented by 1)
      // a destructor (memory deallocation) only actually happen
      // when ref counter reaches zero.
    \end{minted}
  \item \textcolor{orange}{\textbf{Deep copy}} must by explicit:
    \begin{minted}{c++}
      Kokkos::deep_copy(dest,src);
    \end{minted}
    \begin{itemize}
    \item \textbf{Usefull when copying data from a memory space to another}\\
      e.g. \textcolor{red}{from HostSpace to CudaSpace} replacing \texttt{cudaMemcpy}\\
      $\Rightarrow$ one API for all targets
    \item When \texttt{dest} and \texttt{src} are in the same memory space, it does nothing ! (usefull for portability, see example in miniapps later)
    \end{itemize}
  \end{itemize}

\end{frame}
%%%%%%%%%%%%%%%%%%%%%%%%%%%%%%%%%%%%%%%%%%%%%%%%%%%%%%%%%%%%%%%%%%%%%%%% 
%%%%%%%%%%%%%%%%%%%%%%%%%%%%%%%%%%%%%%%%%%%%%%%%%%%%%%%%%%%%%%%%%%%%%%%% 
\begin{frame}[fragile=singleslide]
  \frametitle{Kokkos data Container (4)}

  \begin{itemize}
  \item A verbose \textcolor{black}{\textbf{Kokkos::View}} declaration example of a 1D array of doubles:\\
    %\begin{minted}{c++}
    \textcolor{black}{\texttt{Kokkos::View}}<\textcolor{red}{double*},\textcolor{darkgreen}{Kokkos::LayoutLeft},\textcolor{magenta}{Kokkos::CudaSpace}> a;
    %\end{minted}
    \begin{itemize}
    \item \textcolor{red}{\textbf{What ?}} a data type
    \item \textcolor{darkgreen}{\textbf{How ?}} a memory layout
    \item \textcolor{magenta}{\textbf{Where ?}} a memory space
    \item the last two template parameters are optionnal (have default values)
    \item There is actually a 4th template parameter for Memory traits (e.g. atomic access)
    \end{itemize}
  \item \textcolor{blue}{\texttt{Kokkos::DualView<...>}} : usefull when porting an application incrementally, adata container on two different memory space.\\
    see \texttt{tutorial/Advanced\_Views/04\_dualviews/dual\_view.cpp}
  \item \textcolor{blue}{\texttt{Kokkos::UnorderedMap<...>}}
  \item Can also define \textbf{subview (array slicing, no deep copy)}. See exercice about Mandelbrot set.
  \end{itemize}
  
\end{frame}

%%%%%%%%%%%%%%%%%%%%%%%%%%%%%%%%%%%%%%%%%%%%%%%%%%%%%%%%%%%%%%%%%%%%%%%% 
%%%%%%%%%%%%%%%%%%%%%%%%%%%%%%%%%%%%%%%%%%%%%%%%%%%%%%%%%%%%%%%%%%%%%%%% 
\begin{frame}[fragile=singleslide]
  \frametitle{Kokkos data Container (5)}

  \begin{itemize}
  \item \textcolor{red}{\textbf{What types of data may a View contain ?}}\\
    C++ \myhref{http://en.cppreference.com/w/cpp/concept/PODType}{Plain Old Data} (POD), i.e. basically compatible with C language:
    \begin{itemize}
    \item Can be allocated with \texttt{std::malloc}
    \item Can be copied with \texttt{std::memmove}
    \end{itemize}
  \item POD in C++11: 
    \begin{itemize}
    \item a trivial type (no virtual member functions, no virtual base class)
    \item a standard layout type
    \end{itemize}
  \item C++11: How to check if a given class \texttt{A} is POD ?
  \end{itemize}
  \begin{minted}{c++}
    #include <type_traits>
    
    class A { ... }
    std::cout << "is class A POD ? " << std::is_pod<A>::value << "\n";
  \end{minted}
  
\end{frame}

%%%%%%%%%%%%%%%%%%%%%%%%%%%%%%%%%%%%%%%%%%%%%%%%%%%%%%%%%%%%%%%%%%%%%%%% 
%%%%%%%%%%%%%%%%%%%%%%%%%%%%%%%%%%%%%%%%%%%%%%%%%%%%%%%%%%%%%%%%%%%%%%%% 
\begin{frame}[fragile=singleslide]
  \frametitle{Kokkos data Container (6)}

  {\Large \bf Other interesting types}
  \begin{itemize}
  \item static \texttt{Kokkos::Array} (equivalent to \texttt{std::array})
  \item can be used inside a Kokkos kernel
  \item example
    \begin{minted}{c++}
      using vec = Kokkos::Array<double,3>;
    \end{minted}
  \end{itemize}

  {\Large \bf Interoperability with a legacy C++ API} (pointer based)

  \begin{itemize}
  \item \texttt{void legacyFunction(int * data, int size);}\\
    how to retrieve a raw pointer from a \texttt{Kokkos::View<int *> data}:\\
    \texttt{int *raw\_ptr = data.ptr\_on\_device()} \\
    This is not recommended. Only if you must (e.g. pass data to CuBLAS, ...).\\
    No more reference counting. Kokkos::View's are reference-counted
  \end{itemize}

\end{frame}

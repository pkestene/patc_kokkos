%%%%%%%%%%%%%%%%%%%%%%%%%%%%%%%%%%%%%%%%%%%%%%%%%%%%%%%%%%%%%%%%%%%%%%%% 
%%%%%%%%%%%%%%%%%%%%%%%%%%%%%%%%%%%%%%%%%%%%%%%%%%%%%%%%%%%%%%%%%%%%%%%% 
\begin{frame}[fragile=singleslide]
  \frametitle{Kokkos for Cuda users}

  From a pure software engineering point of view, how does {\bf Kokkos} manage to turn \textcolor{red}{\bf a pur C++ functor} into a \textcolor{darkgreen}{\bf CUDA kernel} ?
  
  \begin{enumerate}
    % explain from core/src/Kokkos_Parallel.hpp
    % explain from core/src/Cuda/Kokkos_Cuda_Parallel.hpp
  \item entry point of parallel computation is through \texttt{parallel\_for} (function call, templated by execution policy, functor, ...)
    \begin{minted}[autogobble=true]{c++}
      // parallel_for is defined in 
      // core/src/Kokkos_Parallel.hpp : line 200
      template< class FunctorType >
      inline
      void parallel_for( const size_t       work_count
                       , const FunctorType& functor
                       , const std::string& str = ""
                       )
      {
        // ...
        Impl::ParallelFor< FunctorType , policy > 
        closure( functor , policy(0,work_count) );
        // ...
      }
    \end{minted}
  \end{enumerate}

\end{frame}

%%%%%%%%%%%%%%%%%%%%%%%%%%%%%%%%%%%%%%%%%%%%%%%%%%%%%%%%%%%%%%%%%%%%%%%% 
%%%%%%%%%%%%%%%%%%%%%%%%%%%%%%%%%%%%%%%%%%%%%%%%%%%%%%%%%%%%%%%%%%%%%%%% 
\begin{frame}[fragile=singleslide]
  \frametitle{Kokkos for Cuda users}

  %From a pure software engineering point of view, how does {\bf Kokkos} manage to turn \textcolor{red}{\bf a pur C++ functor} into a \textcolor{darkgreen}{\bf CUDA kernel} ?
  
  \begin{enumerate}
    \setcounter{enumi}{1}
  \item \texttt{closure} is an instance of the {\bf driver} class \texttt{Kokkos::Impl::ParallelFor}; the precise object type created is off course Kokkos-backend dependent
    % Kokkos_Parallel.hpp : line 221
    % Impl::ParallelFor< FunctorType , policy > closure( functor , policy(0,work_count) );
  \item If CUDA backend is activated, the instantiated class \texttt{Kokkos::Impl::ParallelFor} is defined in \texttt{Cuda/Kokkos\_Cuda\_Parallel.hpp}; there are multiple specialization for the different execution policies (Range, multi-dimensional range, team policy, ...); e.g. for range
    {\scriptsize
      \begin{minted}[autogobble=true]{c++}
        template< class FunctorType , class ... Traits >
        class ParallelFor< FunctorType
                         , Kokkos::RangePolicy< Traits ... >
                         , Kokkos::Cuda
                         >
        {
          // this is where for a given iteration id, the functor is called
          // kind of generic cuda kernel work definition
          inline   __device__  void operator()(void) const { ... };
          
          // this is where the actual CUDA kernel run time config
          // is setup : block and grid dimension
          // then create a CudaParallelLaunch object
          inline void execute() const { ... };
        }

      \end{minted}
  }
  \end{enumerate}

\end{frame}

%%%%%%%%%%%%%%%%%%%%%%%%%%%%%%%%%%%%%%%%%%%%%%%%%%%%%%%%%%%%%%%%%%%%%%%% 
%%%%%%%%%%%%%%%%%%%%%%%%%%%%%%%%%%%%%%%%%%%%%%%%%%%%%%%%%%%%%%%%%%%%%%%% 
\begin{frame}[fragile=singleslide]
  \frametitle{Kokkos for Cuda users}

  %From a pure software engineering point of view, how does {\bf Kokkos} manage to turn \textcolor{red}{\bf a pur C++ functor} into a \textcolor{darkgreen}{\bf CUDA kernel} ?
  
  \begin{enumerate}
    \setcounter{enumi}{3}
  \item when \texttt{closure.execute()} is called, an object \texttt{CudaParallellaunch} is created
  \item struct \texttt{CudaParallellaunch} contains only a constructor, which only purpose is to actually launch the CUDA kernel (using the $<<<$ ... $>>>$ syntax)
  \item Copy closure (driver instance) to GPU memory (either constant, local or global) using Cuda API (e.g \texttt{cudaMemcpyToSymbolAsync} to copy constant memory space)
  \item finally the actual generated cuda kernel, using one of the static functions defined (e.g. \texttt{cuda\_parallel\_launch\_constant\_memory})
    % see Kokkos_Cuda_KernelLaunch.hpp  
  \end{enumerate}

\end{frame}

%%%%%%%%%%%%%%%%%%%%%%%%%%%%%%%%%%%%%%%%%%%%%%%%%%%%%%%%%%%%%%%%%%%%%%%% 
%%%%%%%%%%%%%%%%%%%%%%%%%%%%%%%%%%%%%%%%%%%%%%%%%%%%%%%%%%%%%%%%%%%%%%%% 
\begin{frame}
  \frametitle{Ouessant : IBM Power8 + Nvidia Pascal P100}
  
  \begin{itemize}
  \item {\bf \textcolor{violet}{\large About ouessant computing platform}}
    \begin{itemize}
    \item online ouessant user guide : \myurl{http://www.idris.fr/media/ouessant/ouessant-user_guide.pdf}
    \item Use material from IBM/NVidia~\footnote{Thanks to Nicolas Tallet and P. Vezolle (IBM)}, gives detail on the platform\\
      See file: \myurl{doc/ouessant/Introduction_ouessant_2017.pdf} in archive
    \item Minimal information about software environment, how to build and run an application, submit a job on machine \texttt{ouessant}\\
      See file: \myurl{doc/ouessant/Ouessant-Application_User_Guide-16-12-1.pdf}
    \item Examples of job submission scripts:\\
      \myurl{/pwrlocal/ibmccmpl/share/templates/lsf}\\
      You will need to understand the basics of task affinity on Power architecture (p8aff).\\
      \myhref{https://www.ibm.com/support/knowledgecenter/en/SSWRJV_10.1.0/lsf_admin/affinity_power8_lsf_submit.html}{affinity_power8_lsf_submit.html}
    \end{itemize}
  \end{itemize}
\end{frame}

%%%%%%%%%%%%%%%%%%%%%%%%%%%%%%%%%%%%%%%%%%%%%%%%%%%%%%%%%%%%%%%%%%%%%%%% 
%%%%%%%%%%%%%%%%%%%%%%%%%%%%%%%%%%%%%%%%%%%%%%%%%%%%%%%%%%%%%%%%%%%%%%%% 
\begin{frame}
  \frametitle{Kokkos training material}
  
  \begin{itemize}
  \item {\bf \textcolor{violet}{\large Official Kokkos documentation}}
    \begin{itemize}
    \item {\bf Wiki on github :} \myurl{https://github.com/kokkos/kokkos/wiki}
    \item {\bf Kokkos programming guide} in sources : \myurl{https://github.com/kokkos/kokkos/blob/master/doc/Kokkos_PG.pdf}
    \end{itemize}
  \item {\bf \textcolor{violet}{\large this PATC training material on Kokkos}}
    \begin{itemize}
    \item Last up-to-date archive for this training is on \texttt{ouessant}:\\
      \myurl{/pwrwork/workshops/patc-201805/kokkos/training_kokkos.tar.gz}
    \end{itemize}
  \end{itemize}

\end{frame}


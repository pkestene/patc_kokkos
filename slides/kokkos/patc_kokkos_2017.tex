\documentclass[9pt,hyperref={pdfpagemode=FullScreen,urlcolor=blue},xcolor=x11names]{beamer}

\mode<presentation>
{
  \usetheme{Warsaw}
  %\usetheme{Darmstadt}
  %\usetheme{Marburg}
  \setbeamertemplate{navigation symbols}{}

  %\usecolortheme{crane}
  %\usecolortheme{rose,sidebartab}

  \usecolortheme{beaver}
  %\usecolortheme{lily,sidebartab}
  %\usecolortheme{seahorse}

  \usefonttheme{serif}

  \setbeamertemplate{footline}[page number]
  \setbeamertemplate{sidebar canvas right}[vertical shading][top=palette
  primary.bg,%,middle=white,
  bottom=palette primary.bg]
  %\setbeamertemplate{sections/subsections in toc}[section numbered,subsection numbered]

  %\setbeamertemplate{itemize subitem}[circle]

  \setbeamercovered{transparent}

  %\beamertemplatenavigationsymbolsempty

  \useinnertheme{default}
}

\usepackage[utf8]{inputenc}
\usepackage[T1]{fontenc}
\usepackage{lmodern}
\usepackage{xspace}
\usepackage{amsmath,amssymb}
\usepackage[english]{babel}
%\usepackage[latin1]{inputenc}
%\usepackage[T1]{fontenc}
\usepackage{aeguill,fourier}

% souligne, barre
\usepackage{ulem}
%\usepackage[x11names]{xcolor}

\usepackage{pgf,pgfarrows,pgfnodes,pgfautomata,pgfheaps,pgfshade}


\usepackage{wasysym}
\usepackage{fancyvrb}
%\usepackage{verbatim}
\usepackage{marvosym}

\usepackage{colortbl}

\usepackage{pdftricks}
\begin{psinputs}
\usepackage{pstricks}
\usepackage{pst-bar}
\usepackage{pstricks-add}
\end{psinputs}

\usepackage{ulem}

\usepackage{ifdraft}
\usepackage{animate}
\usepackage{multimedia}

%\usepackage{texmath}

\usepackage{tikz}
\usetikzlibrary{calc}
\usetikzlibrary{patterns}   % for hatching
\usetikzlibrary{positioning}
\usetikzlibrary{decorations.pathreplacing}
\usetikzlibrary{decorations.pathmorphing}
\usetikzlibrary{arrows, decorations.markings}
\usetikzlibrary{shapes.geometric}
\newcommand{\warningsign}{\tikz[baseline=-.75ex] \node[shape=regular polygon, regular polygon sides=3, inner sep=0pt, draw, thick] {\textbf{!}};}
\newcommand{\reddanger}{\textcolor{red}{\danger}}


% the following is from 
% http://tex.stackexchange.com/questions/4811/make-first-row-of-table-all-bold
%\usepackage{array}
%\newcolumntype{$}{>{\global\let\currentrowstyle\relax}}
%\newcolumntype{^}{>{\currentrowstyle}}
%\newcommand{\rowstyle}[1]{\gdef\currentrowstyle{#1}%
%  #1\ignorespaces
%}

\usepackage{listings}
\usepackage{minted}

\usepackage{caption}


%%%%%%%%%%%%%%%%%%%
\hypersetup{%
  pdftitle={PATC-KOKKOS-2017},%
  pdfauthor={Pierre Kestener - CEA Saclay - MDLS - http://www.maisondelasimulation.fr},
  pdfsubject={Introdcution to Kokkos},
  pdfkeywords={KOKKOS, C++, GPU},
  pdfproducer={pdflatex avec la classe BEAMER},
  bookmarksopen=false,
  urlcolor=blue
}

%%%%%%%%%%%%%%%%%%%%%%%%%%%%%%%%%%%%%%%%%%%%%%%%%%%%%%%%%%%%%%%
%%%%%%%%%%%%%%%%%%%%%%%%%%%%%%%%%%%%%%%%%%%%%%%%%%%%%%%%%%%%%%%
%%%%%%%%%%%%%%%%%%%%%%%%%%%%%%%%%%%%%%%%%%%%%%%%%%%%%%%%%%%%%%%

\title{Kokkos, Modern C++, performance portability, ...}

\author
{
  \mbox{\underline{Pierre Kestener}}\inst{1}
}

\institute[mdls sap]{%
  \inst{1}%
  CEA Saclay, DSM, Maison de la Simulation
}

\date{PATC, January, 16-18th, 2017}

\pgfdeclareimage[height=0.5cm]{university-logo}{./images/Sigle-mdls}
\logo{\pgfuseimage{university-logo}}


%%%%%%%%%%%%%%%%%%%%%
\pgfdeclareimage[width=2.0cm]{sigle-cea}{./images/Sigle-mdls}
\pgfdeclareimage[width=2.0cm]{sigle-prace}{images/logo_prace}
\pgfdeclareimage[width=2.0cm]{sigle-nvidia}{images/NV_CUDA_Teaching_Center_3D.jpg}

\titlegraphic{
  % \pgfuseimage{sigle-prace}
  \hfill
  \pgfuseimage{sigle-cea}
  \hfill
  % \pgfuseimage{sigle-nvidia}
}



\begin{document}


\definecolor{green2}{rgb}{0.1,0.8,0.1} 
\definecolor{trust}{rgb}{0.71,0.14,0.07}
\definecolor{FancyPurple}{rgb}{0.5176, 0.1137, 0.2314}

\colorlet{redshaded}{red!25!bg}
\colorlet{shaded}{black!25!bg}
\colorlet{shadedshaded}{black!10!bg}
\colorlet{blackshaded}{black!40!bg}

\colorlet{darkred}{red!80!black}
\colorlet{darkblue}{blue!80!black}
\colorlet{darkgreen}{green!70!black}
\colorlet{greenshaded}{green!95!bg}
%\colorlet{coral}{Coral1!95!bg}

%red, green, blue, cyan, magenta, yellow, black, white, darkgray, gray,
%lightgray, brown, lime, olive, orange, pink, purple, teal, violet

\newcommand\myurl[1]{\textcolor{purple}{\underline{\url{#1}}}}
\newcommand\myhref[2]{\textcolor{purple}{\underline{\href{#1}{#2}}}}

\newcommand\mySmiley{\textcolor{darkgreen}{\Smiley{}}}
\newcommand\myFrowny{\textcolor{red}{\Frowny{}}}

%% Big-O notation.
\providecommand{\OO}[1]{\ensuremath{\operatorname{O}\bigl(#1\bigr)}}

% definition des couleurs pour affichage de code
\makeatletter
\def\PY@reset{\let\PY@it=\relax \let\PY@bf=\relax%
    \let\PY@ul=\relax \let\PY@tc=\relax%
    \let\PY@bc=\relax \let\PY@ff=\relax}
\def\PY@tok#1{\csname PY@tok@#1\endcsname}
\def\PY@toks#1+{\ifx\relax#1\empty\else%
    \PY@tok{#1}\expandafter\PY@toks\fi}
\def\PY@do#1{\PY@bc{\PY@tc{\PY@ul{%
    \PY@it{\PY@bf{\PY@ff{#1}}}}}}}
\def\PY#1#2{\PY@reset\PY@toks#1+\relax+\PY@do{#2}}

\def\PY@tok@gd{\def\PY@tc##1{\textcolor[rgb]{0.63,0.00,0.00}{##1}}}
\def\PY@tok@gu{\let\PY@bf=\textbf\def\PY@tc##1{\textcolor[rgb]{0.50,0.00,0.50}{##1}}}
\def\PY@tok@gt{\def\PY@tc##1{\textcolor[rgb]{0.00,0.25,0.82}{##1}}}
\def\PY@tok@gs{\let\PY@bf=\textbf}
\def\PY@tok@gr{\def\PY@tc##1{\textcolor[rgb]{1.00,0.00,0.00}{##1}}}
\def\PY@tok@cm{\let\PY@it=\textit\def\PY@tc##1{\textcolor[rgb]{0.25,0.50,0.50}{##1}}}
\def\PY@tok@vg{\def\PY@tc##1{\textcolor[rgb]{0.10,0.09,0.49}{##1}}}
\def\PY@tok@m{\def\PY@tc##1{\textcolor[rgb]{0.40,0.40,0.40}{##1}}}
\def\PY@tok@mh{\def\PY@tc##1{\textcolor[rgb]{0.40,0.40,0.40}{##1}}}
\def\PY@tok@go{\def\PY@tc##1{\textcolor[rgb]{0.50,0.50,0.50}{##1}}}
\def\PY@tok@ge{\let\PY@it=\textit}
\def\PY@tok@vc{\def\PY@tc##1{\textcolor[rgb]{0.10,0.09,0.49}{##1}}}
\def\PY@tok@il{\def\PY@tc##1{\textcolor[rgb]{0.40,0.40,0.40}{##1}}}
\def\PY@tok@cs{\let\PY@it=\textit\def\PY@tc##1{\textcolor[rgb]{0.25,0.50,0.50}{##1}}}
\def\PY@tok@cp{\def\PY@tc##1{\textcolor[rgb]{0.74,0.48,0.00}{##1}}}
\def\PY@tok@gi{\def\PY@tc##1{\textcolor[rgb]{0.00,0.63,0.00}{##1}}}
\def\PY@tok@gh{\let\PY@bf=\textbf\def\PY@tc##1{\textcolor[rgb]{0.00,0.00,0.50}{##1}}}
\def\PY@tok@ni{\let\PY@bf=\textbf\def\PY@tc##1{\textcolor[rgb]{0.60,0.60,0.60}{##1}}}
\def\PY@tok@nl{\def\PY@tc##1{\textcolor[rgb]{0.63,0.63,0.00}{##1}}}
\def\PY@tok@nn{\let\PY@bf=\textbf\def\PY@tc##1{\textcolor[rgb]{0.00,0.00,1.00}{##1}}}
\def\PY@tok@no{\def\PY@tc##1{\textcolor[rgb]{0.53,0.00,0.00}{##1}}}
\def\PY@tok@na{\def\PY@tc##1{\textcolor[rgb]{0.49,0.56,0.16}{##1}}}
\def\PY@tok@nb{\def\PY@tc##1{\textcolor[rgb]{0.00,0.50,0.00}{##1}}}
\def\PY@tok@nc{\let\PY@bf=\textbf\def\PY@tc##1{\textcolor[rgb]{0.00,0.00,1.00}{##1}}}
\def\PY@tok@nd{\def\PY@tc##1{\textcolor[rgb]{0.67,0.13,1.00}{##1}}}
\def\PY@tok@ne{\let\PY@bf=\textbf\def\PY@tc##1{\textcolor[rgb]{0.82,0.25,0.23}{##1}}}
\def\PY@tok@nf{\def\PY@tc##1{\textcolor[rgb]{0.00,0.00,1.00}{##1}}}
\def\PY@tok@si{\let\PY@bf=\textbf\def\PY@tc##1{\textcolor[rgb]{0.73,0.40,0.53}{##1}}}
\def\PY@tok@s2{\def\PY@tc##1{\textcolor[rgb]{0.73,0.13,0.13}{##1}}}
\def\PY@tok@vi{\def\PY@tc##1{\textcolor[rgb]{0.10,0.09,0.49}{##1}}}
\def\PY@tok@nt{\let\PY@bf=\textbf\def\PY@tc##1{\textcolor[rgb]{0.00,0.50,0.00}{##1}}}
\def\PY@tok@nv{\def\PY@tc##1{\textcolor[rgb]{0.10,0.09,0.49}{##1}}}
\def\PY@tok@s1{\def\PY@tc##1{\textcolor[rgb]{0.73,0.13,0.13}{##1}}}
\def\PY@tok@sh{\def\PY@tc##1{\textcolor[rgb]{0.73,0.13,0.13}{##1}}}
\def\PY@tok@sc{\def\PY@tc##1{\textcolor[rgb]{0.73,0.13,0.13}{##1}}}
\def\PY@tok@sx{\def\PY@tc##1{\textcolor[rgb]{0.00,0.50,0.00}{##1}}}
\def\PY@tok@bp{\def\PY@tc##1{\textcolor[rgb]{0.00,0.50,0.00}{##1}}}
\def\PY@tok@c1{\let\PY@it=\textit\def\PY@tc##1{\textcolor[rgb]{0.25,0.50,0.50}{##1}}}
\def\PY@tok@kc{\let\PY@bf=\textbf\def\PY@tc##1{\textcolor[rgb]{0.00,0.50,0.00}{##1}}}
\def\PY@tok@c{\let\PY@it=\textit\def\PY@tc##1{\textcolor[rgb]{0.25,0.50,0.50}{##1}}}
\def\PY@tok@mf{\def\PY@tc##1{\textcolor[rgb]{0.40,0.40,0.40}{##1}}}
\def\PY@tok@err{\def\PY@bc##1{\fcolorbox[rgb]{1.00,0.00,0.00}{1,1,1}{##1}}}
\def\PY@tok@kd{\let\PY@bf=\textbf\def\PY@tc##1{\textcolor[rgb]{0.00,0.50,0.00}{##1}}}
\def\PY@tok@ss{\def\PY@tc##1{\textcolor[rgb]{0.10,0.09,0.49}{##1}}}
\def\PY@tok@sr{\def\PY@tc##1{\textcolor[rgb]{0.73,0.40,0.53}{##1}}}
\def\PY@tok@mo{\def\PY@tc##1{\textcolor[rgb]{0.40,0.40,0.40}{##1}}}
\def\PY@tok@kn{\let\PY@bf=\textbf\def\PY@tc##1{\textcolor[rgb]{0.00,0.50,0.00}{##1}}}
\def\PY@tok@mi{\def\PY@tc##1{\textcolor[rgb]{0.40,0.40,0.40}{##1}}}
\def\PY@tok@gp{\let\PY@bf=\textbf\def\PY@tc##1{\textcolor[rgb]{0.00,0.00,0.50}{##1}}}
\def\PY@tok@o{\def\PY@tc##1{\textcolor[rgb]{0.40,0.40,0.40}{##1}}}
\def\PY@tok@kr{\let\PY@bf=\textbf\def\PY@tc##1{\textcolor[rgb]{0.00,0.50,0.00}{##1}}}
\def\PY@tok@s{\def\PY@tc##1{\textcolor[rgb]{0.73,0.13,0.13}{##1}}}
\def\PY@tok@kp{\def\PY@tc##1{\textcolor[rgb]{0.00,0.50,0.00}{##1}}}
\def\PY@tok@w{\def\PY@tc##1{\textcolor[rgb]{0.73,0.73,0.73}{##1}}}
\def\PY@tok@kt{\def\PY@tc##1{\textcolor[rgb]{0.69,0.00,0.25}{##1}}}
\def\PY@tok@ow{\let\PY@bf=\textbf\def\PY@tc##1{\textcolor[rgb]{0.67,0.13,1.00}{##1}}}
\def\PY@tok@sb{\def\PY@tc##1{\textcolor[rgb]{0.73,0.13,0.13}{##1}}}
\def\PY@tok@k{\let\PY@bf=\textbf\def\PY@tc##1{\textcolor[rgb]{0.00,0.50,0.00}{##1}}}
\def\PY@tok@se{\let\PY@bf=\textbf\def\PY@tc##1{\textcolor[rgb]{0.73,0.40,0.13}{##1}}}
\def\PY@tok@sd{\let\PY@it=\textit\def\PY@tc##1{\textcolor[rgb]{0.73,0.13,0.13}{##1}}}

\def\PYZbs{\char`\\}
\def\PYZus{\char`\_}
\def\PYZob{\char`\{}
\def\PYZcb{\char`\}}
\def\PYZca{\char`\^}
\def\PYZsh{\char`\#}
\def\PYZpc{\char`\%}
\def\PYZdl{\char`\$}
\def\PYZti{\char`\~}

\newcommand\lb{[}
\newcommand\rb{]}
\newcommand\PYbg[1]{\textcolor[rgb]{0.00,0.50,0.00}{\textbf{#1}}}
\newcommand\PYbf[1]{\textcolor[rgb]{0.73,0.40,0.53}{\textbf{#1}}}
\newcommand\PYbe[1]{\textcolor[rgb]{0.40,0.40,0.40}{#1}}
\newcommand\PYbd[1]{\textcolor[rgb]{0.73,0.13,0.13}{#1}}
\newcommand\PYbc[1]{\textcolor[rgb]{0.00,0.50,0.00}{\textbf{#1}}}
\newcommand\PYbb[1]{\textcolor[rgb]{0.40,0.40,0.40}{#1}}
\newcommand\PYba[1]{\textcolor[rgb]{0.00,0.00,0.50}{\textbf{#1}}}
\newcommand\PYaJ[1]{\textcolor[rgb]{0.73,0.13,0.13}{#1}}
\newcommand\PYaK[1]{\textcolor[rgb]{0.00,0.00,1.00}{#1}}
\newcommand\PYaH[1]{\fcolorbox[rgb]{1.00,0.00,0.00}{1,1,1}{#1}}
\newcommand\PYaI[1]{\textcolor[rgb]{0.69,0.00,0.25}{#1}}
\newcommand\PYaN[1]{\textcolor[rgb]{0.00,0.00,1.00}{\textbf{#1}}}
\newcommand\PYaO[1]{\textcolor[rgb]{0.00,0.00,0.50}{\textbf{#1}}}
\newcommand\PYaL[1]{\textcolor[rgb]{0.73,0.73,0.73}{#1}}
\newcommand\PYaM[1]{\textcolor[rgb]{0.74,0.48,0.00}{#1}}
\newcommand\PYaB[1]{\textcolor[rgb]{0.00,0.25,0.82}{#1}}
\newcommand\PYaC[1]{\textcolor[rgb]{0.67,0.13,1.00}{#1}}
\newcommand\PYaA[1]{\textcolor[rgb]{0.00,0.50,0.00}{#1}}
\newcommand\PYaF[1]{\textcolor[rgb]{1.00,0.00,0.00}{#1}}
\newcommand\PYaG[1]{\textcolor[rgb]{0.10,0.09,0.49}{#1}}
\newcommand\PYaD[1]{\textcolor[rgb]{0.25,0.50,0.50}{\textit{#1}}}
\newcommand\PYaE[1]{\textcolor[rgb]{0.63,0.00,0.00}{#1}}
\newcommand\PYaZ[1]{\textcolor[rgb]{0.00,0.50,0.00}{\textbf{#1}}}
\newcommand\PYaX[1]{\textcolor[rgb]{0.00,0.50,0.00}{#1}}
\newcommand\PYaY[1]{\textcolor[rgb]{0.73,0.13,0.13}{#1}}
\newcommand\PYaR[1]{\textcolor[rgb]{0.10,0.09,0.49}{#1}}
\newcommand\PYaS[1]{\textcolor[rgb]{0.25,0.50,0.50}{\textit{#1}}}
\newcommand\PYaP[1]{\textcolor[rgb]{0.49,0.56,0.16}{#1}}
\newcommand\PYaQ[1]{\textcolor[rgb]{0.40,0.40,0.40}{#1}}
\newcommand\PYaV[1]{\textcolor[rgb]{0.00,0.00,1.00}{\textbf{#1}}}
\newcommand\PYaW[1]{\textcolor[rgb]{0.73,0.13,0.13}{#1}}
\newcommand\PYaT[1]{\textcolor[rgb]{0.50,0.00,0.50}{\textbf{#1}}}
\newcommand\PYaU[1]{\textcolor[rgb]{0.82,0.25,0.23}{\textbf{#1}}}
\newcommand\PYaj[1]{\textcolor[rgb]{0.00,0.50,0.00}{#1}}
\newcommand\PYak[1]{\textcolor[rgb]{0.73,0.40,0.53}{#1}}
\newcommand\PYah[1]{\textcolor[rgb]{0.63,0.63,0.00}{#1}}
\newcommand\PYai[1]{\textcolor[rgb]{0.10,0.09,0.49}{#1}}
\newcommand\PYan[1]{\textcolor[rgb]{0.67,0.13,1.00}{\textbf{#1}}}
\newcommand\PYao[1]{\textcolor[rgb]{0.73,0.40,0.13}{\textbf{#1}}}
\newcommand\PYal[1]{\textcolor[rgb]{0.25,0.50,0.50}{\textit{#1}}}
\newcommand\PYam[1]{\textbf{#1}}
\newcommand\PYab[1]{\textit{#1}}
\newcommand\PYac[1]{\textcolor[rgb]{0.73,0.13,0.13}{#1}}
\newcommand\PYaa[1]{\textcolor[rgb]{0.50,0.50,0.50}{#1}}
\newcommand\PYaf[1]{\textcolor[rgb]{0.25,0.50,0.50}{\textit{#1}}}
\newcommand\PYag[1]{\textcolor[rgb]{0.40,0.40,0.40}{#1}}
\newcommand\PYad[1]{\textcolor[rgb]{0.73,0.13,0.13}{#1}}
\newcommand\PYae[1]{\textcolor[rgb]{0.40,0.40,0.40}{#1}}
\newcommand\PYaz[1]{\textcolor[rgb]{0.00,0.63,0.00}{#1}}
\newcommand\PYax[1]{\textcolor[rgb]{0.60,0.60,0.60}{\textbf{#1}}}
\newcommand\PYay[1]{\textcolor[rgb]{0.00,0.50,0.00}{\textbf{#1}}}
\newcommand\PYar[1]{\textcolor[rgb]{0.10,0.09,0.49}{#1}}
\newcommand\PYas[1]{\textcolor[rgb]{0.73,0.13,0.13}{\textit{#1}}}
\newcommand\PYap[1]{\textcolor[rgb]{0.00,0.50,0.00}{#1}}
\newcommand\PYaq[1]{\textcolor[rgb]{0.53,0.00,0.00}{#1}}
\newcommand\PYav[1]{\textcolor[rgb]{0.00,0.50,0.00}{\textbf{#1}}}
\newcommand\PYaw[1]{\textcolor[rgb]{0.40,0.40,0.40}{#1}}
\newcommand\PYat[1]{\textcolor[rgb]{0.10,0.09,0.49}{#1}}
\newcommand\PYau[1]{\textcolor[rgb]{0.40,0.40,0.40}{#1}}


% for compatibility with earlier versions
\def\PYZat{@}
\def\PYZlb{[}
\def\PYZrb{]}
\makeatother



%%%%%%%%%%%%%%%%%%%%%
% 1ere page
\begin{frame}[label=courant]
  \titlepage
\end{frame}

%%%%%%%%%%%%%%%%%%%%%%%%%%%%%%%%%%%%%%%%%%%%%%%%%%%%%%%%%%%%%%%%%%%%%%%% 
%%%%%%%%%%%%%%%%%%%%%%%%%%%%%%%%%%%%%%%%%%%%%%%%%%%%%%%%%%%%%%%%%%%%%%%% 
\begin{frame}
  \frametitle{Hands-On 0: Build kokkos (1)}
  
  \textbf{0. Kokkos is still experimental, but moving fast: use git sources}
  
  \textbf{1. Get Kokkos sources, development branch - don't try to build yet !}
  \begin{itemize}
  \item \textcolor{blue}{Practicals on \texttt{ouessant}:}\\
    \textcolor{darkgreen}{\texttt{1. mkdir \$HOME/kokkos-tutorial; cd \$HOME/kokkos-tutorial}}\\
    some kokkos tutorial examples have a Makefile configured for using that precise location.\\
    \textcolor{darkgreen}{\texttt{2. git clone https://github.com/kokkos/kokkos}}\\
    \textcolor{darkgreen}{\texttt{3. cd kokkos; git checkout develop}}
  \end{itemize}
  
\end{frame}

%%%%%%%%%%%%%%%%%%%%%%%%%%%%%%%%%%%%%%%%%%%%%%%%%%%%%%%%%%%%%%%%%%%%%%%% 
%%%%%%%%%%%%%%%%%%%%%%%%%%%%%%%%%%%%%%%%%%%%%%%%%%%%%%%%%%%%%%%%%%%%%%%% 
\begin{frame}
  \frametitle{Hands-On 0: Build kokkos (2)}

  \textbf{2. How to build and use}
  \begin{enumerate}
  \item \textcolor{red}{\textbf{As a regular library (standalone Makefile, installed library):}} \\
    \begin{itemize}
    \item {\bf not recommended} for production level (see below), {\bf OK for testing and building examples}
    \item use \texttt{generate\_makefile.bash}, then \texttt{make kokkoslib; make install}\\
      Then use a \textit{modulefile} to configure the environment\\
      Kokkos examples (inside source) can be built that way, as well as \myhref{https://github.com/kokkos/kokkos-tutorials}{Kokkos-tutorials}
    \end{itemize}
  \item \textcolor{blue}{\textbf{Embedded Kokkos source files in your application}}
    \begin{itemize}
    \item Why ?
    \item $\Rightarrow$ Kokkos by design has {\bf many different configurations possible} (hardware adaptability, heavily relies on C++ metaprograming - compile timing )
    \item $\Rightarrow$ best practice advice : better compile kokkos as part as the target application (same flags, same compiler, etc...)
    \item $\Rightarrow$ \textcolor{blue}{\bf recommended use}:
    \textcolor{darkgreen}{\bf standalone \myhref{https://cmake.org/}{cmake} + kokkos sources embedded in your application} (we'll see a skeleton app)
    \end{itemize}
  \item There exists another cmake-based build sytem, but relies on a third-party tools \myhref{https://tribits.org/}{TriBITS}. Right now this can only be used used when Kokkos is build inside \myhref{https://github.com/trilinos/Trilinos}{Trilinos} (heterogeneous distributed sparse and dense linear algebra package).
  \end{enumerate}
 
\end{frame}

%%%%%%%%%%%%%%%%%%%%%%%%%%%%%%%%%%%%%%%%%%%%%%%%%%%%%%%%%%%%%%%%%%%%%%%% 
%%%%%%%%%%%%%%%%%%%%%%%%%%%%%%%%%%%%%%%%%%%%%%%%%%%%%%%%%%%%%%%%%%%%%%%% 
\begin{frame}
  \frametitle{Hands-On 0: Build kokkos (3)}

  {\bf \textcolor{red}{About standalone Makefile} and environment variables settings for building on multiple architectures}

  \begin{itemize}
  \item The following variables are usefull when building some of the tutorial examples :
    \begin{itemize}
    \item \texttt{KOKKOS\_PATH}: path to Kokkos source dir
    \item \texttt{KOKKOS\_DEVICES}: define possible execution spaces: CUDA, OpenMP, Pthreads, Serial, ...
    \item \texttt{KOKKOS\_ARCH}: used to customize compiler flags; e.g. Power8, Kepler35, SNB, KNL, ARMv80, ROCm, ...
    \end{itemize}
  \item When building for \textcolor{darkgreen}{\bf CUDA device}, you'll need to use Kokkos' own compiler wrapper: \textcolor{darkgreen}{\texttt{\bf nvcc\_wrapper}} (included in Kokkos sources), not just \texttt{nvcc}
  \item \textcolor{red}{When building Kokkos and aiming at an installed Kokkos}, the same information (in a different form) is passed to script \texttt{generate\_makefile.bash}\\
    Just type \texttt{./generate\_makefile.bash \--\--help} at top-level Kokkos sources
  \item \textcolor{blue}{When using Kokkos embedded in your application}, these variables must be set on the \texttt{make} command line.
  \end{itemize}
  
\end{frame}
  
%%%%%%%%%%%%%%%%%%%%%%%%%%%%%%%%%%%%%%%%%%%%%%%%%%%%%%%%%%%%%%%%%%%%%%%% 
%%%%%%%%%%%%%%%%%%%%%%%%%%%%%%%%%%%%%%%%%%%%%%%%%%%%%%%%%%%%%%%%%%%%%%%% 
\begin{frame}
  \frametitle{Hands-On 0: Build kokkos (4)}

  \begin{itemize}
  \item \textcolor{blue}{\textbf{Example build configurations (for an installed Kokkos)}}
    \begin{itemize}
    \item For \texttt{ouessant}, see file \texttt{doc/readme\_build\_kokkos\_ouessant} in the provided archive
    \item Serial (mostly for testing)\\
      \texttt{../generate\_makefile.bash --with-serial --prefix=\$HOME/local/kokkos\_serial}
    \item \textbf{OpenMP}\\
      \texttt{../generate\_makefile.bash --with-openmp --prefix=\$HOME/local/kokkos\_openmp\_dev}
    \item \textbf{CUDA (+ OpenMP)}; typical configuration\\
      \texttt{../generate\_makefile.bash --with-cuda --arch=Pascal60 --prefix=\$HOME/local/kokkos\_cuda\_lambda\_openmp --with-cuda-options=enable\_lambda --with-openmp --with-hwloc=/usr}
    \end{itemize}
  \item \textcolor{darkgreen}{\textbf{After installation}} (\texttt{make kokkoslib; make install;}) the file \textbf{\texttt{Makefile.kokkos}} is created, and designed to be reused in your application build system.
  \item \textbf{Two choices for integrating Kokkos in your app:}
    \begin{itemize}
    \item Use an existing Makefile from Kokkos tutorial, examples, ...
    \item Use your own build system ({\bf cmake recommended}): there can be a quite large combinatorics of \texttt{DEVICES}, \texttt{ARCH}, compilers, compiler options, ...
    \end{itemize}
  \end{itemize}
  
\end{frame}

%%%%%%%%%%%%%%%%%%%%%%%%%%%%%%%%%%%%%%%%%%%%%%%%%%%%%%%%%%%%%%%%%%%%%%%% 
%%%%%%%%%%%%%%%%%%%%%%%%%%%%%%%%%%%%%%%%%%%%%%%%%%%%%%%%%%%%%%%%%%%%%%%% 
\begin{frame}
  \frametitle{Kokkos - Documentation}

  \begin{itemize}
  \item PDF documentation in kokkos source tree : \texttt{doc/Kokkos\_PG.pdf} (programming guide)
  \item \myhref{http://www.stack.nl/~dimitri/doxygen/}{Doxygen} can only be built from inside \myhref{https://github.com/trilinos/Trilinos}{Trilinos source tree}\\
    Version of the day can be browsed at \myurl{https://trilinos.org/docs/dev/packages/kokkos/doc/html/index.html}
  \item Kokkos source code itself, reading unit tests code is also helpful
  \end{itemize}

  Additionnal resources:

  \begin{itemize}
  \item Tutorial slides and codes: \\
    \myurl{https://github.com/kokkos/kokkos-tutorials}
  \end{itemize}
  
\end{frame}

%%%%%%%%%%%%%%%%%%%%%%%%%%%%%%%%%%%%%%%%%%%%%%%%%%%%%%%%%%%%%%%%%%%%%%%% 
%%%%%%%%%%%%%%%%%%%%%%%%%%%%%%%%%%%%%%%%%%%%%%%%%%%%%%%%%%%%%%%%%%%%%%%% 
\begin{frame}[fragile=singleslide]
  \frametitle{Kokkos - initialize / finalize}

  \begin{itemize}
  \item \texttt{Kokkos::initialize / finalize}
    %// introspection on configuration options
    {\small\begin{minted}{c++}
        #include <Kokkos_Macros.hpp>
        #include <Kokkos_Core.hpp>
        
        int main(int argc, char* argv[]) {
          // default: initialize the host exec space
          // What exactly gets initialized depends on how kokkos
          // was built, i.e. which options was passed to
          // generate_makefile.bash
          Kokkos::initialize();
          ...
          Kokkos::finalize();
        }
      \end{minted}
    }
    %
  \item \textcolor{red}{\textbf{What's happening inside \texttt{Kokkos::initialize}}}
    \begin{itemize}
    \item Defines \textcolor{blue}{\texttt{Default Device / DefaultExecutionSpace Default memory space}} (as specified when kokkos itself was built, by order of {\bf priority}: Cuda > OpenMP > Pthreads > Serial)\\
      e.g. if \texttt{\--\--with-cuda} was not pass to \texttt{generate\_makefile.bash}, but \texttt{\--\--with-openmp} was, then \texttt{DefaultExecutionSpace} is OpenMP
    \item You can activate several execution spaces (recommended)
    %\item Defines a \textcolor{blue}{default memory space}
    \item all this information provided at compile time will internally be used inside Kokkos sources as default (hidden) template parameters
    \end{itemize}
    % 
  \end{itemize}
  %
\end{frame}

%%%%%%%%%%%%%%%%%%%%%%%%%%%%%%%%%%%%%%%%%%%%%%%%%%%%%%%%%%%%%%%%%%%%%%%% 
%%%%%%%%%%%%%%%%%%%%%%%%%%%%%%%%%%%%%%%%%%%%%%%%%%%%%%%%%%%%%%%%%%%%%%%% 
\begin{frame}[fragile=singleslide]
  \frametitle{Kokkos - initialize / finalize}

  \begin{itemize}
  \item \texttt{Kokkos::initialize / finalize} (most of the time OK)
    % // introspection on configuration options
    {\small\begin{minted}{c++}
        #include <Kokkos_Macros.hpp>
        #include <Kokkos_Core.hpp>
        
        int main(int argc, char* argv[]) {
          // default: initialize the host exec space
          // What exactly gets initialized depends on how kokkos
          // was built, i.e. which options was passed to
          // generate_makefile.bash
          Kokkos::initialize();
          ...
          Kokkos::finalize();
        }
      \end{minted}
    }
    % 
  \item \textbf{Fine control of initialization:}
    \begin{itemize}
    \item \texttt{\bf Kokkos::initialize(argc, argv);}\\
      User can change/fix e.g. number OpenMP threads on the application's command line
    \item This is regular initialization. If available \textcolor{orange}{\textbf{\texttt{hwloc}}} library is available and activated, it provides default hardware locality:
      \begin{itemize}
      \item For OpenMP exec space: number of threads (default is all CPU cores)\\
        NB: usual environment variables (e.g. \texttt{OMP\_NUM\_THREADS}, \texttt{GOMP\_CPU\_AFFINITY} can (of course) also be used
      \item Mapping between GPUs and MPI task
      \end{itemize}
    \end{itemize}
    % 
  \end{itemize}
  
\end{frame}


%%%%%%%%%%%%%%%%%%%%%%%%%%%%%%%%%%%%%%%%%%%%%%%%%%%%%%%%%%%%%%%%%%%%%%%% 
%%%%%%%%%%%%%%%%%%%%%%%%%%%%%%%%%%%%%%%%%%%%%%%%%%%%%%%%%%%%%%%%%%%%%%%% 
\begin{frame}[fragile=singleslide]
  \frametitle{Kokkos - initialize / finalize}

  \begin{itemize}
  \item \textcolor{red}{\textbf{Advanced initialization}} with \textcolor{blue}{\textbf{OpenMP + CUDA}}\\
    \textbf{Needed/usefull to be able to execution computation on both HOST / GPU}
  \end{itemize}
  \begin{minted}{c++}
    #if defined( KOKKOS_ENABLE_CUDA )
    Kokkos::HostSpace::execution_space::initialize(teams*num_threads);
    Kokkos::Cuda::SelectDevice select_device(device);
    Kokkos::Cuda::initialize(select_device);
    #elif defined( KOKKOS_ENABLE_OPENMP )
    Kokkos::OpenMP::initialize(teams*num_threads);
    #elif defined( KOKKOS_ENABLE_PTHREAD )
    Kokkos::Threads::initialize(teams*num_threads);
    #endif
  \end{minted}
\end{frame}


%%%%%%%%%%%%%%%%%%%%%%%%%%%%%%%%%%%%%%%%%%%%%%%%%%%%%%%%%%%%%%%%%%%%%%%% 
%%%%%%%%%%%%%%%%%%%%%%%%%%%%%%%%%%%%%%%%%%%%%%%%%%%%%%%%%%%%%%%%%%%%%%%% 
\begin{frame}[fragile=singleslide]
  \frametitle{Kokkos - initialize / finalize with MPI}

  \begin{itemize}
  \item \textcolor{red}{\textbf{Advanced initialization}} with \textcolor{blue}{\textbf{MPI + Kokkos/CUDA}} \textcolor{violet}{\textbf{version 1 : implicit mapping}}\\
    Don't do anything special, let Kokkos through hwloc chose the GPU
    {\scriptsize
      \begin{minted}{c++}
        // Just checking how Kokkos+hwloc performed
        // the MPI rank - GPU mapping 
        int cudaDeviceId;
        cudaGetDevice(&cudaDeviceId);
        std::cout << "I'm MPI task #" << rank << " pinned to GPU #" << cudaDeviceId << "\n";
      \end{minted} 
    }
  \item \textcolor{red}{\textbf{Advanced initialization}} with \textcolor{blue}{\textbf{MPI + Kokkos/CUDA}} \textcolor{violet}{\textbf{version 2 : explicit mapping}}
    (we will come back into that with example code)
    {\scriptsize
      \begin{minted}{c++}
        
        // MPI initialized above
        
        // probe the number of CUDA device (i.e. GPUs)
        const int ngpu = Kokkos::Cuda::detect_device_count();
        
        // provide a mapping 1 MPI task <-> 1 GPU
        const int cuda_device_rank = pre_mpi_local_rank % ngpu ;
        
        // each MPI task initialize the selected device id
        Kokkos::Cuda::initialize(
        Kokkos::Cuda::SelectDevice( cuda_device_rank ) );
      \end{minted}
    }
  \item In any case, \textcolor{darkgreen}{\bf cross-check this information} with the job scheduler, e.g. \texttt{mpirun \--\--report-bindings}
  \end{itemize}
\end{frame}


\end{document}

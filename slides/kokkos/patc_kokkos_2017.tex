\documentclass[9pt,hyperref={pdfpagemode=FullScreen,urlcolor=blue},xcolor=x11names]{beamer}

\mode<presentation>
{
  \usetheme{Warsaw}
  %\usetheme{Darmstadt}
  %\usetheme{Marburg}
  \setbeamertemplate{navigation symbols}{}

  %\usecolortheme{crane}
  %\usecolortheme{rose,sidebartab}

  \usecolortheme{beaver}
  %\usecolortheme{lily,sidebartab}
  %\usecolortheme{seahorse}

  \usefonttheme{serif}

  \setbeamertemplate{footline}[page number]
  \setbeamertemplate{sidebar canvas right}[vertical shading][top=palette
  primary.bg,%,middle=white,
  bottom=palette primary.bg]
  %\setbeamertemplate{sections/subsections in toc}[section numbered,subsection numbered]

  %\setbeamertemplate{itemize subitem}[circle]

  \setbeamercovered{transparent}

  %\beamertemplatenavigationsymbolsempty

  \useinnertheme{default}
}

\usepackage[utf8]{inputenc}
\usepackage[T1]{fontenc}
\usepackage{lmodern}
\usepackage{xspace}
\usepackage{amsmath,amssymb}
\usepackage[english]{babel}
%\usepackage[latin1]{inputenc}
%\usepackage[T1]{fontenc}
\usepackage{aeguill,fourier}

% souligne, barre
\usepackage{ulem}
%\usepackage[x11names]{xcolor}

\usepackage{pgf,pgfarrows,pgfnodes,pgfautomata,pgfheaps,pgfshade}


\usepackage{wasysym}
\usepackage{fancyvrb}
%\usepackage{verbatim}
\usepackage{marvosym}

\usepackage{colortbl}

\usepackage{pdftricks}
\begin{psinputs}
\usepackage{pstricks}
\usepackage{pst-bar}
\usepackage{pstricks-add}
\end{psinputs}

\usepackage{ulem}

\usepackage{ifdraft}
\usepackage{animate}
\usepackage{multimedia}

%\usepackage{texmath}

\usepackage{tikz}
\usetikzlibrary{calc}
\usetikzlibrary{patterns}   % for hatching
\usetikzlibrary{positioning}
\usetikzlibrary{decorations.pathreplacing}
\usetikzlibrary{decorations.pathmorphing}
\usetikzlibrary{arrows, decorations.markings}
\usetikzlibrary{shapes.geometric}
\newcommand{\warningsign}{\tikz[baseline=-.75ex] \node[shape=regular polygon, regular polygon sides=3, inner sep=0pt, draw, thick] {\textbf{!}};}
\newcommand{\reddanger}{\textcolor{red}{\danger}}


% the following is from 
% http://tex.stackexchange.com/questions/4811/make-first-row-of-table-all-bold
%\usepackage{array}
%\newcolumntype{$}{>{\global\let\currentrowstyle\relax}}
%\newcolumntype{^}{>{\currentrowstyle}}
%\newcommand{\rowstyle}[1]{\gdef\currentrowstyle{#1}%
%  #1\ignorespaces
%}

\usepackage{listings}
\usepackage{minted}

\usepackage{caption}


%%%%%%%%%%%%%%%%%%%
\hypersetup{%
  pdftitle={PATC-KOKKOS-2017},%
  pdfauthor={Pierre Kestener - CEA Saclay - MDLS - http://www.maisondelasimulation.fr},
  pdfsubject={Introdcution to Kokkos},
  pdfkeywords={KOKKOS, C++, GPU},
  pdfproducer={pdflatex avec la classe BEAMER},
  bookmarksopen=false,
  urlcolor=blue
}

%%%%%%%%%%%%%%%%%%%%%%%%%%%%%%%%%%%%%%%%%%%%%%%%%%%%%%%%%%%%%%%
%%%%%%%%%%%%%%%%%%%%%%%%%%%%%%%%%%%%%%%%%%%%%%%%%%%%%%%%%%%%%%%
%%%%%%%%%%%%%%%%%%%%%%%%%%%%%%%%%%%%%%%%%%%%%%%%%%%%%%%%%%%%%%%

\title{Kokkos, Modern C++, performance portability, ...}

\author
{
  \mbox{\underline{Pierre Kestener}}\inst{1}
}

\institute[mdls sap]{%
  \inst{1}%
  CEA Saclay, DSM, Maison de la Simulation
}

\date{PATC, January, 16-18th, 2017}

\pgfdeclareimage[height=0.5cm]{university-logo}{./images/Sigle-mdls}
\logo{\pgfuseimage{university-logo}}


%%%%%%%%%%%%%%%%%%%%%
\pgfdeclareimage[width=2.0cm]{sigle-cea}{./images/Sigle-mdls}
\pgfdeclareimage[width=2.0cm]{sigle-prace}{images/logo_prace}
\pgfdeclareimage[width=2.0cm]{sigle-nvidia}{images/NV_CUDA_Teaching_Center_3D.jpg}

\titlegraphic{
  % \pgfuseimage{sigle-prace}
  \hfill
  \pgfuseimage{sigle-cea}
  \hfill
  % \pgfuseimage{sigle-nvidia}
}



\begin{document}


\definecolor{green2}{rgb}{0.1,0.8,0.1} 
\definecolor{trust}{rgb}{0.71,0.14,0.07}
\definecolor{FancyPurple}{rgb}{0.5176, 0.1137, 0.2314}

\colorlet{redshaded}{red!25!bg}
\colorlet{shaded}{black!25!bg}
\colorlet{shadedshaded}{black!10!bg}
\colorlet{blackshaded}{black!40!bg}

\colorlet{darkred}{red!80!black}
\colorlet{darkblue}{blue!80!black}
\colorlet{darkgreen}{green!70!black}
\colorlet{greenshaded}{green!95!bg}
%\colorlet{coral}{Coral1!95!bg}

%red, green, blue, cyan, magenta, yellow, black, white, darkgray, gray,
%lightgray, brown, lime, olive, orange, pink, purple, teal, violet

\newcommand\myurl[1]{\textcolor{purple}{\underline{\url{#1}}}}
\newcommand\myhref[2]{\textcolor{purple}{\underline{\href{#1}{#2}}}}

\newcommand\mySmiley{\textcolor{darkgreen}{\Smiley{}}}
\newcommand\myFrowny{\textcolor{red}{\Frowny{}}}

%% Big-O notation.
\providecommand{\OO}[1]{\ensuremath{\operatorname{O}\bigl(#1\bigr)}}

% definition des couleurs pour affichage de code
\makeatletter
\def\PY@reset{\let\PY@it=\relax \let\PY@bf=\relax%
    \let\PY@ul=\relax \let\PY@tc=\relax%
    \let\PY@bc=\relax \let\PY@ff=\relax}
\def\PY@tok#1{\csname PY@tok@#1\endcsname}
\def\PY@toks#1+{\ifx\relax#1\empty\else%
    \PY@tok{#1}\expandafter\PY@toks\fi}
\def\PY@do#1{\PY@bc{\PY@tc{\PY@ul{%
    \PY@it{\PY@bf{\PY@ff{#1}}}}}}}
\def\PY#1#2{\PY@reset\PY@toks#1+\relax+\PY@do{#2}}

\def\PY@tok@gd{\def\PY@tc##1{\textcolor[rgb]{0.63,0.00,0.00}{##1}}}
\def\PY@tok@gu{\let\PY@bf=\textbf\def\PY@tc##1{\textcolor[rgb]{0.50,0.00,0.50}{##1}}}
\def\PY@tok@gt{\def\PY@tc##1{\textcolor[rgb]{0.00,0.25,0.82}{##1}}}
\def\PY@tok@gs{\let\PY@bf=\textbf}
\def\PY@tok@gr{\def\PY@tc##1{\textcolor[rgb]{1.00,0.00,0.00}{##1}}}
\def\PY@tok@cm{\let\PY@it=\textit\def\PY@tc##1{\textcolor[rgb]{0.25,0.50,0.50}{##1}}}
\def\PY@tok@vg{\def\PY@tc##1{\textcolor[rgb]{0.10,0.09,0.49}{##1}}}
\def\PY@tok@m{\def\PY@tc##1{\textcolor[rgb]{0.40,0.40,0.40}{##1}}}
\def\PY@tok@mh{\def\PY@tc##1{\textcolor[rgb]{0.40,0.40,0.40}{##1}}}
\def\PY@tok@go{\def\PY@tc##1{\textcolor[rgb]{0.50,0.50,0.50}{##1}}}
\def\PY@tok@ge{\let\PY@it=\textit}
\def\PY@tok@vc{\def\PY@tc##1{\textcolor[rgb]{0.10,0.09,0.49}{##1}}}
\def\PY@tok@il{\def\PY@tc##1{\textcolor[rgb]{0.40,0.40,0.40}{##1}}}
\def\PY@tok@cs{\let\PY@it=\textit\def\PY@tc##1{\textcolor[rgb]{0.25,0.50,0.50}{##1}}}
\def\PY@tok@cp{\def\PY@tc##1{\textcolor[rgb]{0.74,0.48,0.00}{##1}}}
\def\PY@tok@gi{\def\PY@tc##1{\textcolor[rgb]{0.00,0.63,0.00}{##1}}}
\def\PY@tok@gh{\let\PY@bf=\textbf\def\PY@tc##1{\textcolor[rgb]{0.00,0.00,0.50}{##1}}}
\def\PY@tok@ni{\let\PY@bf=\textbf\def\PY@tc##1{\textcolor[rgb]{0.60,0.60,0.60}{##1}}}
\def\PY@tok@nl{\def\PY@tc##1{\textcolor[rgb]{0.63,0.63,0.00}{##1}}}
\def\PY@tok@nn{\let\PY@bf=\textbf\def\PY@tc##1{\textcolor[rgb]{0.00,0.00,1.00}{##1}}}
\def\PY@tok@no{\def\PY@tc##1{\textcolor[rgb]{0.53,0.00,0.00}{##1}}}
\def\PY@tok@na{\def\PY@tc##1{\textcolor[rgb]{0.49,0.56,0.16}{##1}}}
\def\PY@tok@nb{\def\PY@tc##1{\textcolor[rgb]{0.00,0.50,0.00}{##1}}}
\def\PY@tok@nc{\let\PY@bf=\textbf\def\PY@tc##1{\textcolor[rgb]{0.00,0.00,1.00}{##1}}}
\def\PY@tok@nd{\def\PY@tc##1{\textcolor[rgb]{0.67,0.13,1.00}{##1}}}
\def\PY@tok@ne{\let\PY@bf=\textbf\def\PY@tc##1{\textcolor[rgb]{0.82,0.25,0.23}{##1}}}
\def\PY@tok@nf{\def\PY@tc##1{\textcolor[rgb]{0.00,0.00,1.00}{##1}}}
\def\PY@tok@si{\let\PY@bf=\textbf\def\PY@tc##1{\textcolor[rgb]{0.73,0.40,0.53}{##1}}}
\def\PY@tok@s2{\def\PY@tc##1{\textcolor[rgb]{0.73,0.13,0.13}{##1}}}
\def\PY@tok@vi{\def\PY@tc##1{\textcolor[rgb]{0.10,0.09,0.49}{##1}}}
\def\PY@tok@nt{\let\PY@bf=\textbf\def\PY@tc##1{\textcolor[rgb]{0.00,0.50,0.00}{##1}}}
\def\PY@tok@nv{\def\PY@tc##1{\textcolor[rgb]{0.10,0.09,0.49}{##1}}}
\def\PY@tok@s1{\def\PY@tc##1{\textcolor[rgb]{0.73,0.13,0.13}{##1}}}
\def\PY@tok@sh{\def\PY@tc##1{\textcolor[rgb]{0.73,0.13,0.13}{##1}}}
\def\PY@tok@sc{\def\PY@tc##1{\textcolor[rgb]{0.73,0.13,0.13}{##1}}}
\def\PY@tok@sx{\def\PY@tc##1{\textcolor[rgb]{0.00,0.50,0.00}{##1}}}
\def\PY@tok@bp{\def\PY@tc##1{\textcolor[rgb]{0.00,0.50,0.00}{##1}}}
\def\PY@tok@c1{\let\PY@it=\textit\def\PY@tc##1{\textcolor[rgb]{0.25,0.50,0.50}{##1}}}
\def\PY@tok@kc{\let\PY@bf=\textbf\def\PY@tc##1{\textcolor[rgb]{0.00,0.50,0.00}{##1}}}
\def\PY@tok@c{\let\PY@it=\textit\def\PY@tc##1{\textcolor[rgb]{0.25,0.50,0.50}{##1}}}
\def\PY@tok@mf{\def\PY@tc##1{\textcolor[rgb]{0.40,0.40,0.40}{##1}}}
\def\PY@tok@err{\def\PY@bc##1{\fcolorbox[rgb]{1.00,0.00,0.00}{1,1,1}{##1}}}
\def\PY@tok@kd{\let\PY@bf=\textbf\def\PY@tc##1{\textcolor[rgb]{0.00,0.50,0.00}{##1}}}
\def\PY@tok@ss{\def\PY@tc##1{\textcolor[rgb]{0.10,0.09,0.49}{##1}}}
\def\PY@tok@sr{\def\PY@tc##1{\textcolor[rgb]{0.73,0.40,0.53}{##1}}}
\def\PY@tok@mo{\def\PY@tc##1{\textcolor[rgb]{0.40,0.40,0.40}{##1}}}
\def\PY@tok@kn{\let\PY@bf=\textbf\def\PY@tc##1{\textcolor[rgb]{0.00,0.50,0.00}{##1}}}
\def\PY@tok@mi{\def\PY@tc##1{\textcolor[rgb]{0.40,0.40,0.40}{##1}}}
\def\PY@tok@gp{\let\PY@bf=\textbf\def\PY@tc##1{\textcolor[rgb]{0.00,0.00,0.50}{##1}}}
\def\PY@tok@o{\def\PY@tc##1{\textcolor[rgb]{0.40,0.40,0.40}{##1}}}
\def\PY@tok@kr{\let\PY@bf=\textbf\def\PY@tc##1{\textcolor[rgb]{0.00,0.50,0.00}{##1}}}
\def\PY@tok@s{\def\PY@tc##1{\textcolor[rgb]{0.73,0.13,0.13}{##1}}}
\def\PY@tok@kp{\def\PY@tc##1{\textcolor[rgb]{0.00,0.50,0.00}{##1}}}
\def\PY@tok@w{\def\PY@tc##1{\textcolor[rgb]{0.73,0.73,0.73}{##1}}}
\def\PY@tok@kt{\def\PY@tc##1{\textcolor[rgb]{0.69,0.00,0.25}{##1}}}
\def\PY@tok@ow{\let\PY@bf=\textbf\def\PY@tc##1{\textcolor[rgb]{0.67,0.13,1.00}{##1}}}
\def\PY@tok@sb{\def\PY@tc##1{\textcolor[rgb]{0.73,0.13,0.13}{##1}}}
\def\PY@tok@k{\let\PY@bf=\textbf\def\PY@tc##1{\textcolor[rgb]{0.00,0.50,0.00}{##1}}}
\def\PY@tok@se{\let\PY@bf=\textbf\def\PY@tc##1{\textcolor[rgb]{0.73,0.40,0.13}{##1}}}
\def\PY@tok@sd{\let\PY@it=\textit\def\PY@tc##1{\textcolor[rgb]{0.73,0.13,0.13}{##1}}}

\def\PYZbs{\char`\\}
\def\PYZus{\char`\_}
\def\PYZob{\char`\{}
\def\PYZcb{\char`\}}
\def\PYZca{\char`\^}
\def\PYZsh{\char`\#}
\def\PYZpc{\char`\%}
\def\PYZdl{\char`\$}
\def\PYZti{\char`\~}

\newcommand\lb{[}
\newcommand\rb{]}
\newcommand\PYbg[1]{\textcolor[rgb]{0.00,0.50,0.00}{\textbf{#1}}}
\newcommand\PYbf[1]{\textcolor[rgb]{0.73,0.40,0.53}{\textbf{#1}}}
\newcommand\PYbe[1]{\textcolor[rgb]{0.40,0.40,0.40}{#1}}
\newcommand\PYbd[1]{\textcolor[rgb]{0.73,0.13,0.13}{#1}}
\newcommand\PYbc[1]{\textcolor[rgb]{0.00,0.50,0.00}{\textbf{#1}}}
\newcommand\PYbb[1]{\textcolor[rgb]{0.40,0.40,0.40}{#1}}
\newcommand\PYba[1]{\textcolor[rgb]{0.00,0.00,0.50}{\textbf{#1}}}
\newcommand\PYaJ[1]{\textcolor[rgb]{0.73,0.13,0.13}{#1}}
\newcommand\PYaK[1]{\textcolor[rgb]{0.00,0.00,1.00}{#1}}
\newcommand\PYaH[1]{\fcolorbox[rgb]{1.00,0.00,0.00}{1,1,1}{#1}}
\newcommand\PYaI[1]{\textcolor[rgb]{0.69,0.00,0.25}{#1}}
\newcommand\PYaN[1]{\textcolor[rgb]{0.00,0.00,1.00}{\textbf{#1}}}
\newcommand\PYaO[1]{\textcolor[rgb]{0.00,0.00,0.50}{\textbf{#1}}}
\newcommand\PYaL[1]{\textcolor[rgb]{0.73,0.73,0.73}{#1}}
\newcommand\PYaM[1]{\textcolor[rgb]{0.74,0.48,0.00}{#1}}
\newcommand\PYaB[1]{\textcolor[rgb]{0.00,0.25,0.82}{#1}}
\newcommand\PYaC[1]{\textcolor[rgb]{0.67,0.13,1.00}{#1}}
\newcommand\PYaA[1]{\textcolor[rgb]{0.00,0.50,0.00}{#1}}
\newcommand\PYaF[1]{\textcolor[rgb]{1.00,0.00,0.00}{#1}}
\newcommand\PYaG[1]{\textcolor[rgb]{0.10,0.09,0.49}{#1}}
\newcommand\PYaD[1]{\textcolor[rgb]{0.25,0.50,0.50}{\textit{#1}}}
\newcommand\PYaE[1]{\textcolor[rgb]{0.63,0.00,0.00}{#1}}
\newcommand\PYaZ[1]{\textcolor[rgb]{0.00,0.50,0.00}{\textbf{#1}}}
\newcommand\PYaX[1]{\textcolor[rgb]{0.00,0.50,0.00}{#1}}
\newcommand\PYaY[1]{\textcolor[rgb]{0.73,0.13,0.13}{#1}}
\newcommand\PYaR[1]{\textcolor[rgb]{0.10,0.09,0.49}{#1}}
\newcommand\PYaS[1]{\textcolor[rgb]{0.25,0.50,0.50}{\textit{#1}}}
\newcommand\PYaP[1]{\textcolor[rgb]{0.49,0.56,0.16}{#1}}
\newcommand\PYaQ[1]{\textcolor[rgb]{0.40,0.40,0.40}{#1}}
\newcommand\PYaV[1]{\textcolor[rgb]{0.00,0.00,1.00}{\textbf{#1}}}
\newcommand\PYaW[1]{\textcolor[rgb]{0.73,0.13,0.13}{#1}}
\newcommand\PYaT[1]{\textcolor[rgb]{0.50,0.00,0.50}{\textbf{#1}}}
\newcommand\PYaU[1]{\textcolor[rgb]{0.82,0.25,0.23}{\textbf{#1}}}
\newcommand\PYaj[1]{\textcolor[rgb]{0.00,0.50,0.00}{#1}}
\newcommand\PYak[1]{\textcolor[rgb]{0.73,0.40,0.53}{#1}}
\newcommand\PYah[1]{\textcolor[rgb]{0.63,0.63,0.00}{#1}}
\newcommand\PYai[1]{\textcolor[rgb]{0.10,0.09,0.49}{#1}}
\newcommand\PYan[1]{\textcolor[rgb]{0.67,0.13,1.00}{\textbf{#1}}}
\newcommand\PYao[1]{\textcolor[rgb]{0.73,0.40,0.13}{\textbf{#1}}}
\newcommand\PYal[1]{\textcolor[rgb]{0.25,0.50,0.50}{\textit{#1}}}
\newcommand\PYam[1]{\textbf{#1}}
\newcommand\PYab[1]{\textit{#1}}
\newcommand\PYac[1]{\textcolor[rgb]{0.73,0.13,0.13}{#1}}
\newcommand\PYaa[1]{\textcolor[rgb]{0.50,0.50,0.50}{#1}}
\newcommand\PYaf[1]{\textcolor[rgb]{0.25,0.50,0.50}{\textit{#1}}}
\newcommand\PYag[1]{\textcolor[rgb]{0.40,0.40,0.40}{#1}}
\newcommand\PYad[1]{\textcolor[rgb]{0.73,0.13,0.13}{#1}}
\newcommand\PYae[1]{\textcolor[rgb]{0.40,0.40,0.40}{#1}}
\newcommand\PYaz[1]{\textcolor[rgb]{0.00,0.63,0.00}{#1}}
\newcommand\PYax[1]{\textcolor[rgb]{0.60,0.60,0.60}{\textbf{#1}}}
\newcommand\PYay[1]{\textcolor[rgb]{0.00,0.50,0.00}{\textbf{#1}}}
\newcommand\PYar[1]{\textcolor[rgb]{0.10,0.09,0.49}{#1}}
\newcommand\PYas[1]{\textcolor[rgb]{0.73,0.13,0.13}{\textit{#1}}}
\newcommand\PYap[1]{\textcolor[rgb]{0.00,0.50,0.00}{#1}}
\newcommand\PYaq[1]{\textcolor[rgb]{0.53,0.00,0.00}{#1}}
\newcommand\PYav[1]{\textcolor[rgb]{0.00,0.50,0.00}{\textbf{#1}}}
\newcommand\PYaw[1]{\textcolor[rgb]{0.40,0.40,0.40}{#1}}
\newcommand\PYat[1]{\textcolor[rgb]{0.10,0.09,0.49}{#1}}
\newcommand\PYau[1]{\textcolor[rgb]{0.40,0.40,0.40}{#1}}


% for compatibility with earlier versions
\def\PYZat{@}
\def\PYZlb{[}
\def\PYZrb{]}
\makeatother



%%%%%%%%%%%%%%%%%%%%%
% 1ere page
\begin{frame}[label=courant]
  \titlepage
\end{frame}

\section{Build Kokkos}
%%%%%%%%%%%%%%%%%%%%%%%%%%%%%%%%%%%%%%%%%%%%%%%%%%%%%%%%%%%%%%%%%%%%%%%% 
%%%%%%%%%%%%%%%%%%%%%%%%%%%%%%%%%%%%%%%%%%%%%%%%%%%%%%%%%%%%%%%%%%%%%%%% 
\begin{frame}
  \frametitle{Build kokkos}

  \begin{itemize}
  \item \textbf{Get sources, development branch}
    \begin{itemize}
    \item \texttt{git clone https://github.com/kokkos/kokkos}
    \item \texttt{git checkout develop}
    \item \textcolor{blue}{Practicals on \texttt{ouessant}:}\\
      \texttt{mkdir \$HOME/kokkos-tutorial}\\
      perform \texttt{git clone} operation from \texttt{\$HOME/kokkos-tutorial}\\
      some tutorial examples are already configured for using that precise location.
    \end{itemize}
  \item \textbf{Build configuration}
    \begin{itemize}
    \item \texttt{cd \$KOKKOS\_SOURCES; mkdir build; cd build}
    \item About build system, several ways to use Kokkos
      \begin{enumerate}
      \item Only when Kokkos is build inside \myhref{https://github.com/trilinos/Trilinos}{Trilinos}, \myhref{https://cmake.org/}{CMake} is used
      \item Standalone utilization: use \texttt{generate\_makefile.bash}, then \texttt{make kokkoslib; make install}
      \item Embedded Kokkos source files in your application.
      \end{enumerate}
      % \item We will use 2. and 3.
    \end{itemize}
  \end{itemize}
 
\end{frame}

%%%%%%%%%%%%%%%%%%%%%%%%%%%%%%%%%%%%%%%%%%%%%%%%%%%%%%%%%%%%%%%%%%%%%%%% 
%%%%%%%%%%%%%%%%%%%%%%%%%%%%%%%%%%%%%%%%%%%%%%%%%%%%%%%%%%%%%%%%%%%%%%%% 
\begin{frame}
  \frametitle{Build kokkos (2)}

  \begin{itemize}
  \item Example build configurations
    \begin{itemize}
    \item Serial (mostly for testing)\\
      \texttt{../generate\_makefile.bash --with-serial --prefix=\$HOME/local/kokkos\_serial}
    \item \textbf{OpenMP}\\
      \texttt{../generate\_makefile.bash --with-openmp --prefix=\$HOME/local/kokkos\_openmp\_dev}
    \item \textbf{CUDA (+ OpenMP)}\\
      \texttt{../generate\_makefile.bash --with-cuda --arch=Pascal60 --prefix=\$HOME/local/kokkos\_cuda\_lambda\_openmp --with-cuda-options=enable\_lambda --with-openmp --with-hwloc=/usr}
    \end{itemize}
  \end{itemize}
  
\end{frame}

\section{Tutorial Kokkos}
%%%%%%%%%%%%%%%%%%%%%%%%%%%%%%%%%%%%%%%%%%%%%%%%%%%%%%%%%%%%%%%%%%%%%%%% 
%%%%%%%%%%%%%%%%%%%%%%%%%%%%%%%%%%%%%%%%%%%%%%%%%%%%%%%%%%%%%%%%%%%%%%%% 
\begin{frame}
  \frametitle{Tutorial Kokkos}

  \begin{itemize}
  \item We will first re-use material from Kokkos github repository.
  \item On your home, on \texttt{oeussant}: 
    \begin{enumerate}
    \item \texttt{mkdir kokkos-tutorial; cd kokkos-tutorial}
    \item \texttt{git clone https://github.com/kokkos/kokkos.git} \\
      \# \textbf{Don't try to build kokkos here (for now)}
    \item \texttt{git clone https://github.com/kokkos/kokkos-tutorials.git}
    \item \texttt{cd kokkos-tutorials/1-Day-Tutorial}\\
      \# 1 Day tutorial exercice are routed to \textbf{build kokkos for you}
    \end{enumerate}
  \end{itemize}

\end{frame}

%%%%%%%%%%%%%%%%%%%%%%%%%%%%%%%%%%%%%%%%%%%%%%%%%%%%%%%%%%%%%%%%%%%%%%%% 
%%%%%%%%%%%%%%%%%%%%%%%%%%%%%%%%%%%%%%%%%%%%%%%%%%%%%%%%%%%%%%%%%%%%%%%% 
\begin{frame}[fragile=singleslide]
  \frametitle{1Day Tutorial: SAXPY}

  \begin{itemize}
  \item \textbf{Proposed activity}
  \item \textcolor{red}{\textbf{Saxpy serial (reference executable on Power8)}}
    \begin{itemize}
    \item \texttt{cd \$HOME/kokkos-tutorial/kokkos-tutorials/1-Day-Tutorial/Exercises/01\_AXPY/Serial}
    \item Open \texttt{Makefile} and change \texttt{SNB} into \texttt{Power8}
    \item \texttt{make}
    \end{itemize}
  \item \textcolor{orange}{\textbf{Saxpy OpenMP (on Power8)}}
    \begin{itemize}
    \item \texttt{cd \$HOME/kokkos-tutorial/kokkos-tutorials/1-Day-Tutorial/Exercises/01\_AXPY/Kokkos-Lambda}
    \item Change again \texttt{Makefile}
    \item Add 3 lines in \texttt{saxpy.cpp} right after Kokkos initialization
      \begin{minted}{c++}
        std::ostringstream msg;
        Kokkos::OpenMP::print_configuration( msg );
        std::cout << msg.str();
      \end{minted}
    \item Make sure all available CPU cores were used ($1\times 160 \times 1$)
    \item Change the number of OpenMP threads created by kokkos, e.g. :\\
      \texttt{./saxpy.host  --threads=20}
    \item We need to change \texttt{Makefile} and use hwloc\\
      add \texttt{KOKKOS\_USE\_TPLS="hwloc"} right after \texttt{KOKKOS\_DEVICES} in \texttt{Makefile}\\
      Rebuild and rerun, you should see that application uses the available numa domains, and increasee bandwidth use !
    \end{itemize}
  \end{itemize}

\end{frame}

%%%%%%%%%%%%%%%%%%%%%%%%%%%%%%%%%%%%%%%%%%%%%%%%%%%%%%%%%%%%%%%%%%%%%%%% 
%%%%%%%%%%%%%%%%%%%%%%%%%%%%%%%%%%%%%%%%%%%%%%%%%%%%%%%%%%%%%%%%%%%%%%%% 
\begin{frame}[fragile=singleslide]
  \frametitle{1Day Tutorial: SAXPY}

  \begin{itemize}
  \item \textbf{Proposed activity}
  \item \textcolor{darkgreen}{\textbf{Saxpy CUDA (on Power8 + Nvidia K80/P100)}}
    \begin{itemize}
    \item \texttt{cd \$HOME/kokkos-tutorial/kokkos-tutorials/1-Day-Tutorial/Exercises/01\_AXPY/Kokkos-Lambda}
    \item \texttt{module load cuda/8.0}
    \item Change again \texttt{Makefile}: \\
      add CUDA to variable KOKKOS\_DEVICES = "Cuda,OpenMP"\\
      Kepler35 $\Rightarrow$ Kepler37 (for Nvidia K80)\\
      Kepler35 $\Rightarrow$ Pascal60 (for Nvidia P100)
    \end{itemize}
  \item Rebuild for K80, run on ouessant (front node)
  \item Rebuild for P100, run on compute node using \texttt{submit\_ouessant.sh} (should see a strong difference)
  \end{itemize}

\end{frame}


%%%%%%%%%%%%%%%%%%%%%%%%%%%%%%%%%%%%%%%%%%%%%%%%%%%%%%%%%%%%%%%%%%%%%%%% 
%%%%%%%%%%%%%%%%%%%%%%%%%%%%%%%%%%%%%%%%%%%%%%%%%%%%%%%%%%%%%%%%%%%%%%%% 
\begin{frame}[fragile=singleslide]
  \frametitle{Kokkos data Container}

  \begin{itemize}
  \item \texttt{Kokkos::View<...>}; replacement for \texttt{std::vector} with multidimensionnal feature and hardware adapted memory layout\\
    \begin{itemize}
    \item \texttt{Kokkos::View<double **>}
    \end{itemize}

  \item \texttt{Kokkos::DualView<...>} : usefull when porting an application incrementally, adata container on two different memory space.\\
    see \texttt{tutorial/Advanced\_Views/04\_dualviews/dual\_view.cpp}
  \item \texttt{Kokkos::UnorderedMap<...>}
  \end{itemize}

\end{frame}

%%%%%%%%%%%%%%%%%%%%%%%%%%%%%%%%%%%%%%%%%%%%%%%%%%%%%%%%%%%%%%%%%%%%%%%% 
%%%%%%%%%%%%%%%%%%%%%%%%%%%%%%%%%%%%%%%%%%%%%%%%%%%%%%%%%%%%%%%%%%%%%%%% 
\begin{frame}[fragile=singleslide]
  \frametitle{Kokkos compute Kernels}

  \begin{itemize}
  \item How to specify a compute kernel in Kokkos ?
    \begin{enumerate}
    \item \textcolor{blue}{\textbf{Use Lambda functions.}}\\
      NB: a lambda in c++11 is an unnamed function object capable of capturing variables in scope.
      \begin{minted}{c++}
        Kokkos::parallel_for (100, KOKKOS_LAMBDA (const int i) {
          data(i) = 2*i;
        });
      \end{minted}
    \item \textcolor{darkgreen}{\textbf{Use a C++ functor class.}}\\
      A functor is a class containing a function to execute in parallel.
      \begin{minted}{c++}
        class FunctorType {
          public:
          KOKKOS_INLINE_FUNCTION
          void operator() ( const int i ) const ;
        };
      \end{minted}
    \end{enumerate}
  \end{itemize}

\end{frame}


%%%%%%%%%%%%%%%%%%%%%%%%%%%%%%%%%%%%%%%%%%%%%%%%%%%%%%%%%%%%%%%%%%%%%%%% 
%%%%%%%%%%%%%%%%%%%%%%%%%%%%%%%%%%%%%%%%%%%%%%%%%%%%%%%%%%%%%%%%%%%%%%%% 
\begin{frame}[fragile=singleslide]
  \frametitle{Kokkos compute Kernels}

  \textbf{Lambda or Functor: which one to use in Kokkos ? Both !}
  \begin{enumerate}
  \item \textcolor{blue}{\textbf{Use Lambda functions.}}\\
    \begin{itemize}
    \item easy way for small compute kernels
    \item For GPU, requires Cuda 7.5 (8.0 is current and latest CUDA version)
    \end{itemize}
  \item \textcolor{darkgreen}{\textbf{Use a C++ functor class.}}\\
    \begin{itemize}
    \item More flexible, allow to design more complex kernel
    \end{itemize}
  \end{enumerate}
\end{frame}

\end{document}

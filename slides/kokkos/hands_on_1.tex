%%%%%%%%%%%%%%%%%%%%%%%%%%%%%%%%%%%%%%%%%%%%%%%%%%%%%%%%%%%%%%%%%%%%%%%% 
%%%%%%%%%%%%%%%%%%%%%%%%%%%%%%%%%%%%%%%%%%%%%%%%%%%%%%%%%%%%%%%%%%%%%%%% 
\begin{frame}
  \frametitle{Hands-On 1a : Cuda device\_query - job submission}

  {\large\textcolor{red}{\textbf{Purpose:} just make sure you are able to launch a job on Ouessant}}

  \begin{itemize}
  \item We will use a cuda sample
  \item In your home on \texttt{ouessant}: 
    \begin{itemize}
    \item \texttt{cp -a /usr/local/cuda-9.2/samples ./samples-9.2}
    \item \texttt{cd samples-9.2/1\_Utilities/deviceQuery}
    \item \texttt{module load at/10.0 cuda/9.2}
    \item \texttt{make}
    \item You have an executable named \texttt{deviceQuery}
    \item You can run it on a \textcolor{red}{\bf Ouessant login node:} \fbox{\texttt{./deviceQuery}}
    \item You can run it on a \textcolor{darkgreen}{\bf Ouessant compute node} using the script \texttt{submit\_ouessant.sh} launched like this:\\
      \fbox{\texttt{bsub < submit\_ouessant.sh}}\\
      The submission script is located in the training archive (\texttt{code/handson/1a/submit\_ouessant.sh})
    \item What differences can you see between the two executions ?
    \end{itemize}
  \end{itemize}

\end{frame}

%%%%%%%%%%%%%%%%%%%%%%%%%%%%%%%%%%%%%%%%%%%%%%%%%%%%%%%%%%%%%%%%%%%%%%%% 
%%%%%%%%%%%%%%%%%%%%%%%%%%%%%%%%%%%%%%%%%%%%%%%%%%%%%%%%%%%%%%%%%%%%%%%% 
\begin{frame}
  \frametitle{Hands-On 1b : Kokkos query\_device with hwloc}

  \hypertarget{handson1}{}
  {\large\textcolor{red}{\textbf{Purpose:} just cross-checking Kokkos/Hwloc is working OK}}

  \begin{itemize}
  \item We will first re-use material from Kokkos github repository.
  \item On your home, on \texttt{ouessant}: 
    \begin{enumerate}
    \item \texttt{mkdir kokkos-tutorial; cd kokkos-tutorial}
    \item \texttt{git clone https://github.com/kokkos/kokkos.git} \\
      \# \textbf{Don't try to build kokkos here (for now)}
    %\item \texttt{git clone https://github.com/kokkos/kokkos-tutorials.git}
    %\item \texttt{cd kokkos-tutorials/Intro-Full/SNL2015/Exercises/}\\
    %  \# 1 Day tutorial exercice are routed to \textbf{build kokkos for you}
    \end{enumerate}
  \end{itemize}

\end{frame}

%%%%%%%%%%%%%%%%%%%%%%%%%%%%%%%%%%%%%%%%%%%%%%%%%%%%%%%%%%%%%%%%%%%%%%%% 
%%%%%%%%%%%%%%%%%%%%%%%%%%%%%%%%%%%%%%%%%%%%%%%%%%%%%%%%%%%%%%%%%%%%%%%% 
\begin{frame}[fragile=singleslide]
  \frametitle{Hands-On 1b : Kokkos query\_device with hwloc}

  {\large\textcolor{red}{\textbf{Purpose:}}}
  \begin{itemize}
  \item \textcolor{red}{just cross-checking Kokkos/Hwloc is working OK}
  \item \textcolor{red}{On login nodes only for now}
  \end{itemize}
    
  {\bf TO DO:}
  \begin{itemize}
  \item Kokkos sources will be built by the application Makefile
  \item \texttt{cd \$HOME/kokkos-tutorial/kokkos/example/query\_device}
  \item open \texttt{query\_device.cpp} to have a look; no computations, it just prints hardware information
  \item \textbf{Take some time to have a look at the Makefile.}\\
    Note that latter when using an installed kokkos library, we won't need to set architecture or device related variables on the command line .
  \end{itemize}

\end{frame}

%%%%%%%%%%%%%%%%%%%%%%%%%%%%%%%%%%%%%%%%%%%%%%%%%%%%%%%%%%%%%%%%%%%%%%%% 
%%%%%%%%%%%%%%%%%%%%%%%%%%%%%%%%%%%%%%%%%%%%%%%%%%%%%%%%%%%%%%%%%%%%%%%% 
\begin{frame}[fragile=singleslide]
  \frametitle{Hands-On 1b : Kokkos query\_device with hwloc}

  \begin{enumerate}
  \item \textbf{Serial build (default, with hwloc):}\\
    \fbox{\texttt{make KOKKOS\_USE\_TPLS="hwloc"}}\\
    How many NUMA / Cores / Hyperthreads on power8 CPU ?\\
    \underline{Cross check:} what is the current SMT mode on a \texttt{ouessant} login node ?
    \begin{itemize}
    \item \texttt{ppc64\_cpu \--\--smt}
    \item \texttt{ppc64\_cpu \--\--info} 
    \end{itemize}
  \item environment:~\footnote{module at/10.0 provides gnu (advanced) toolchain, version 6.3.1 (at/11.0 not compatible with cuda 9.2)} \texttt{module load at/10.0 cuda/9.2}
  \item \textbf{OpenMP build (with hwloc):}\\
    \fbox{\texttt{make KOKKOS\_USE\_TPLS="hwloc" KOKKOS\_DEVICES=OpenMP}} (off course, exact same information obtained as with serial execution)
  \item \textbf{CUDA/OpenMP build (with hwloc):}\\
    \fbox{\texttt{make KOKKOS\_USE\_TPLS="hwloc" KOKKOS\_DEVICES=Cuda,OpenMP}} rerun and you should get information about the CPU+GPU configuration
  \end{enumerate}
  
\end{frame}

% https://devtalk.nvidia.com/default/topic/1023776/-request-add-nvcc-compatibility-with-glibc-2-26/
  
%%%%%%%%%%%%%%%%%%%%%%%%%%%%%%%%%%%%%%%%%%%%%%%%%%%%%%%%%%%%%%%%%%%%%%%% 
%%%%%%%%%%%%%%%%%%%%%%%%%%%%%%%%%%%%%%%%%%%%%%%%%%%%%%%%%%%%%%%%%%%%%%%% 
\begin{frame}[fragile=singleslide]
  \frametitle{Hands-On 1b : Kokkos query\_device without hwloc}

  {\large\textcolor{orange}{\bf Question: What happens if hwloc is not activated ?}}

  \begin{itemize}
  \item Edit file \texttt{query\_device.cpp} and do the following modification:
    \begin{enumerate}
    \item Add \texttt{Kokkos::initialize(argc, argv);} after \texttt{MPI\_Init}
    \item Add \texttt{Kokkos::finalize();} before \texttt{MPI\_Finalize}
    \item Rebuild and run \texttt{./query\_device.host \--\--help}
    \item run \texttt{./query\_device.host \--\--kokkos-threads=12} (alternatively, you can use regular OpenMP environment variables)
    \item change\\
      {\small
        \begin{minted}{c++}
          #if defined( KOKKOS_ENABLE_CUDA )
          Kokkos::Cuda::print_configuration( msg );
          #else
          Kokkos::OpenMP::print_configuration( msg );
          #endif
          //Kokkos::print_configuration( msg ); // how this app was build
        \end{minted}
      }
    \end{enumerate}
  \item {\small Rebuild 1 \textcolor{red}{without HWLOC:}\\
      \texttt{make KOKKOS\_DEVICES=OpenMP}}
    % {\small
    %   \begin{minted}{bash}
    %     Kokkos::OpenMP thread_pool_topology[ 1 x 80 x 1 ]
    %   \end{minted}
    % }
  \item {\small Rebuild 2 \textcolor{darkgreen}{with HWLOC:}\\
      \texttt{make KOKKOS\_DEVICES=OpenMP KOKKOS\_USE\_TPLS="hwloc"}}
    % {\small 
    %   \begin{minted}{bash}
    %     hwloc( NUMA[2] x CORE[10] x HT[4] )
    %     Kokkos::OpenMP hwloc[2x10x4] hwloc_binding_enabled thread_pool_topology[ 2 x 10 x 4 ]      
    %   \end{minted}
    % }
  %\item add \texttt{Kokkos::print\_configuration} to cross-check at run-time executable was built with the right options
  \item processor affinity is important to performance; you can/must configure OpenMP environment.
  \end{itemize}

\end{frame}

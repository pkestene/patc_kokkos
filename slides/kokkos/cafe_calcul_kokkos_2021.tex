\documentclass[9pt,hyperref={pdfpagemode=FullScreen,urlcolor=blue},usenames,xcolor=dvipsnames]{beamer}

\mode<presentation>
{
  \usetheme{Warsaw}
  %\usetheme{Darmstadt}
  %\usetheme{Marburg}
  \setbeamertemplate{navigation symbols}{}

  %\usecolortheme{crane}
  %\usecolortheme{rose,sidebartab}

  \usecolortheme{beaver}
  %\usecolortheme{lily,sidebartab}
  %\usecolortheme{seahorse}

  \usefonttheme{serif}

  \setbeamertemplate{footline}[page number]
  \setbeamertemplate{sidebar canvas right}[vertical shading][top=palette
  primary.bg,%,middle=white,
  bottom=palette primary.bg]
  %\setbeamertemplate{sections/subsections in toc}[section numbered,subsection numbered]

  %\setbeamertemplate{itemize subitem}[circle]

  \setbeamercovered{transparent}

  %\beamertemplatenavigationsymbolsempty

  \useinnertheme{default}
  \setbeamertemplate{enumerate items}[default]
}

\usepackage[utf8]{inputenc}
\usepackage[T1]{fontenc}
\usepackage{lmodern}
\usepackage{xspace}
\usepackage{amsmath,amssymb}
\usepackage[english]{babel}
%\usepackage[latin1]{inputenc}
%\usepackage[T1]{fontenc}
\usepackage{aeguill,fourier}

% souligne, barre
\usepackage{ulem}
% \usepackage[x11names]{xcolor}

\usepackage{pgf,pgfarrows,pgfnodes,pgfautomata,pgfheaps,pgfshade}


\usepackage{wasysym}
\usepackage{fancyvrb}
%\usepackage{verbatim}
\usepackage{marvosym}

\usepackage{colortbl}

\usepackage{pdftricks}
\begin{psinputs}
\usepackage{pstricks}
\usepackage{pst-bar}
\usepackage{pstricks-add}
\end{psinputs}

\usepackage{ulem}

\usepackage{ifdraft}
\usepackage{animate}
\usepackage{multimedia}

%\usepackage{texmath}

\usepackage{tikz}
\usetikzlibrary{calc}
\usetikzlibrary{patterns}   % for hatching
\usetikzlibrary{positioning}
\usetikzlibrary{decorations.pathreplacing}
\usetikzlibrary{decorations.pathmorphing}
\usetikzlibrary{arrows, decorations.markings}
\usetikzlibrary{shapes.geometric}
\newcommand{\warningsign}{\tikz[baseline=-.75ex] \node[shape=regular polygon, regular polygon sides=3, inner sep=0pt, draw, thick] {\textbf{!}};}
\newcommand{\reddanger}{\textcolor{red}{\danger}}


% the following is from
% http://tex.stackexchange.com/questions/4811/make-first-row-of-table-all-bold
%\usepackage{array}
%\newcolumntype{$}{>{\global\let\currentrowstyle\relax}}
%\newcolumntype{^}{>{\currentrowstyle}}
%\newcommand{\rowstyle}[1]{\gdef\currentrowstyle{#1}%
%  #1\ignorespaces
%}

\usepackage{listings}
\usepackage{minted}

\usepackage{caption}


%%%%%%%%%%%%%%%%%%%
\hypersetup{%
  pdftitle={CAF-CALCUL-2021},%
  pdfauthor={Pierre Kestener - CEA Saclay - IRFU/LILAS},
  pdfsubject={Introduction to Kokkos},
  pdfkeywords={KOKKOS, C++, GPU},
  pdfproducer={pdflatex avec la classe BEAMER},
  bookmarksopen=false,
  urlcolor=blue
}

%%%%%%%%%%%%%%%%%%%%%%%%%%%%%%%%%%%%%%%%%%%%%%%%%%%%%%%%%%%%%%%
%%%%%%%%%%%%%%%%%%%%%%%%%%%%%%%%%%%%%%%%%%%%%%%%%%%%%%%%%%%%%%%
%%%%%%%%%%%%%%%%%%%%%%%%%%%%%%%%%%%%%%%%%%%%%%%%%%%%%%%%%%%%%%%

\title{Pourquoi (j'aime bien) Kokkos ?\\
  Modern C++, portabilité de performance, ...}

\author
{
  \textcolor{purple}{\underline{\href{https://github.com/pkestene}{Pierre Kestener}}}
}

\institute{%
  %\inst{1}%
  \myhref{http://www.cea.fr/}{CEA Saclay}, \myhref{http://www.cea.fr/drf/Pages/Accueil.aspx}{DRF}, \myhref{https://irfu.cea.fr/Phocea/Vie_des_labos/Ast/ast_service.php?id_unit=5}{IRFU/DEDIP/LILAS}
}

\date{Café CALCUL - 22 octobre 2021}

%\pgfdeclareimage[height=0.5cm]{university-logo}{./images/Sigle-mdls}
\pgfdeclareimage[height=0.7cm]{university-logo}{./images/cea2}

\logo{\pgfuseimage{university-logo}}


%%%%%%%%%%%%%%%%%%%%%
\pgfdeclareimage[width=1.0cm]{sigle-cea}{./images/Sigle-mdls}
\pgfdeclareimage[width=2.0cm]{sigle-prace}{images/logo_prace}
\pgfdeclareimage[width=2.0cm]{sigle-nvidia}{images/NV_CUDA_Teaching_Center_3D.jpg}

\titlegraphic{
  % \pgfuseimage{sigle-prace}
  \hfill
  %\pgfuseimage{sigle-cea}
  \hfill
  % \pgfuseimage{sigle-nvidia}
}



\begin{document}


\definecolor{green2}{rgb}{0.1,0.8,0.1}
\definecolor{trust}{rgb}{0.71,0.14,0.07}
\definecolor{FancyPurple}{rgb}{0.5176, 0.1137, 0.2314}

\colorlet{redshaded}{red!25!bg}
\colorlet{shaded}{black!25!bg}
\colorlet{shadedshaded}{black!10!bg}
\colorlet{blackshaded}{black!40!bg}

\colorlet{darkred}{red!80!black}
\colorlet{darkblue}{blue!80!black}
\colorlet{darkgreen}{green!70!black}
\colorlet{greenshaded}{green!95!bg}
%\colorlet{coral}{Coral1!95!bg}

%red, green, blue, cyan, magenta, yellow, black, white, darkgray, gray,
%lightgray, brown, lime, olive, orange, pink, purple, teal, violet

\newcommand\myurl[1]{\textcolor{purple}{\underline{\url{#1}}}}
\newcommand\myhref[2]{\textcolor{purple}{\underline{\href{#1}{#2}}}}

\newcommand\mySmiley{\textcolor{darkgreen}{\Smiley{}}}
\newcommand\myFrowny{\textcolor{red}{\Frowny{}}}

%% Big-O notation.
\providecommand{\OO}[1]{\ensuremath{\operatorname{O}\bigl(#1\bigr)}}

% definition des couleurs pour affichage de code
\makeatletter
\def\PY@reset{\let\PY@it=\relax \let\PY@bf=\relax%
    \let\PY@ul=\relax \let\PY@tc=\relax%
    \let\PY@bc=\relax \let\PY@ff=\relax}
\def\PY@tok#1{\csname PY@tok@#1\endcsname}
\def\PY@toks#1+{\ifx\relax#1\empty\else%
    \PY@tok{#1}\expandafter\PY@toks\fi}
\def\PY@do#1{\PY@bc{\PY@tc{\PY@ul{%
    \PY@it{\PY@bf{\PY@ff{#1}}}}}}}
\def\PY#1#2{\PY@reset\PY@toks#1+\relax+\PY@do{#2}}

\def\PY@tok@gd{\def\PY@tc##1{\textcolor[rgb]{0.63,0.00,0.00}{##1}}}
\def\PY@tok@gu{\let\PY@bf=\textbf\def\PY@tc##1{\textcolor[rgb]{0.50,0.00,0.50}{##1}}}
\def\PY@tok@gt{\def\PY@tc##1{\textcolor[rgb]{0.00,0.25,0.82}{##1}}}
\def\PY@tok@gs{\let\PY@bf=\textbf}
\def\PY@tok@gr{\def\PY@tc##1{\textcolor[rgb]{1.00,0.00,0.00}{##1}}}
\def\PY@tok@cm{\let\PY@it=\textit\def\PY@tc##1{\textcolor[rgb]{0.25,0.50,0.50}{##1}}}
\def\PY@tok@vg{\def\PY@tc##1{\textcolor[rgb]{0.10,0.09,0.49}{##1}}}
\def\PY@tok@m{\def\PY@tc##1{\textcolor[rgb]{0.40,0.40,0.40}{##1}}}
\def\PY@tok@mh{\def\PY@tc##1{\textcolor[rgb]{0.40,0.40,0.40}{##1}}}
\def\PY@tok@go{\def\PY@tc##1{\textcolor[rgb]{0.50,0.50,0.50}{##1}}}
\def\PY@tok@ge{\let\PY@it=\textit}
\def\PY@tok@vc{\def\PY@tc##1{\textcolor[rgb]{0.10,0.09,0.49}{##1}}}
\def\PY@tok@il{\def\PY@tc##1{\textcolor[rgb]{0.40,0.40,0.40}{##1}}}
\def\PY@tok@cs{\let\PY@it=\textit\def\PY@tc##1{\textcolor[rgb]{0.25,0.50,0.50}{##1}}}
\def\PY@tok@cp{\def\PY@tc##1{\textcolor[rgb]{0.74,0.48,0.00}{##1}}}
\def\PY@tok@gi{\def\PY@tc##1{\textcolor[rgb]{0.00,0.63,0.00}{##1}}}
\def\PY@tok@gh{\let\PY@bf=\textbf\def\PY@tc##1{\textcolor[rgb]{0.00,0.00,0.50}{##1}}}
\def\PY@tok@ni{\let\PY@bf=\textbf\def\PY@tc##1{\textcolor[rgb]{0.60,0.60,0.60}{##1}}}
\def\PY@tok@nl{\def\PY@tc##1{\textcolor[rgb]{0.63,0.63,0.00}{##1}}}
\def\PY@tok@nn{\let\PY@bf=\textbf\def\PY@tc##1{\textcolor[rgb]{0.00,0.00,1.00}{##1}}}
\def\PY@tok@no{\def\PY@tc##1{\textcolor[rgb]{0.53,0.00,0.00}{##1}}}
\def\PY@tok@na{\def\PY@tc##1{\textcolor[rgb]{0.49,0.56,0.16}{##1}}}
\def\PY@tok@nb{\def\PY@tc##1{\textcolor[rgb]{0.00,0.50,0.00}{##1}}}
\def\PY@tok@nc{\let\PY@bf=\textbf\def\PY@tc##1{\textcolor[rgb]{0.00,0.00,1.00}{##1}}}
\def\PY@tok@nd{\def\PY@tc##1{\textcolor[rgb]{0.67,0.13,1.00}{##1}}}
\def\PY@tok@ne{\let\PY@bf=\textbf\def\PY@tc##1{\textcolor[rgb]{0.82,0.25,0.23}{##1}}}
\def\PY@tok@nf{\def\PY@tc##1{\textcolor[rgb]{0.00,0.00,1.00}{##1}}}
\def\PY@tok@si{\let\PY@bf=\textbf\def\PY@tc##1{\textcolor[rgb]{0.73,0.40,0.53}{##1}}}
\def\PY@tok@s2{\def\PY@tc##1{\textcolor[rgb]{0.73,0.13,0.13}{##1}}}
\def\PY@tok@vi{\def\PY@tc##1{\textcolor[rgb]{0.10,0.09,0.49}{##1}}}
\def\PY@tok@nt{\let\PY@bf=\textbf\def\PY@tc##1{\textcolor[rgb]{0.00,0.50,0.00}{##1}}}
\def\PY@tok@nv{\def\PY@tc##1{\textcolor[rgb]{0.10,0.09,0.49}{##1}}}
\def\PY@tok@s1{\def\PY@tc##1{\textcolor[rgb]{0.73,0.13,0.13}{##1}}}
\def\PY@tok@sh{\def\PY@tc##1{\textcolor[rgb]{0.73,0.13,0.13}{##1}}}
\def\PY@tok@sc{\def\PY@tc##1{\textcolor[rgb]{0.73,0.13,0.13}{##1}}}
\def\PY@tok@sx{\def\PY@tc##1{\textcolor[rgb]{0.00,0.50,0.00}{##1}}}
\def\PY@tok@bp{\def\PY@tc##1{\textcolor[rgb]{0.00,0.50,0.00}{##1}}}
\def\PY@tok@c1{\let\PY@it=\textit\def\PY@tc##1{\textcolor[rgb]{0.25,0.50,0.50}{##1}}}
\def\PY@tok@kc{\let\PY@bf=\textbf\def\PY@tc##1{\textcolor[rgb]{0.00,0.50,0.00}{##1}}}
\def\PY@tok@c{\let\PY@it=\textit\def\PY@tc##1{\textcolor[rgb]{0.25,0.50,0.50}{##1}}}
\def\PY@tok@mf{\def\PY@tc##1{\textcolor[rgb]{0.40,0.40,0.40}{##1}}}
\def\PY@tok@err{\def\PY@bc##1{\fcolorbox[rgb]{1.00,0.00,0.00}{1,1,1}{##1}}}
\def\PY@tok@kd{\let\PY@bf=\textbf\def\PY@tc##1{\textcolor[rgb]{0.00,0.50,0.00}{##1}}}
\def\PY@tok@ss{\def\PY@tc##1{\textcolor[rgb]{0.10,0.09,0.49}{##1}}}
\def\PY@tok@sr{\def\PY@tc##1{\textcolor[rgb]{0.73,0.40,0.53}{##1}}}
\def\PY@tok@mo{\def\PY@tc##1{\textcolor[rgb]{0.40,0.40,0.40}{##1}}}
\def\PY@tok@kn{\let\PY@bf=\textbf\def\PY@tc##1{\textcolor[rgb]{0.00,0.50,0.00}{##1}}}
\def\PY@tok@mi{\def\PY@tc##1{\textcolor[rgb]{0.40,0.40,0.40}{##1}}}
\def\PY@tok@gp{\let\PY@bf=\textbf\def\PY@tc##1{\textcolor[rgb]{0.00,0.00,0.50}{##1}}}
\def\PY@tok@o{\def\PY@tc##1{\textcolor[rgb]{0.40,0.40,0.40}{##1}}}
\def\PY@tok@kr{\let\PY@bf=\textbf\def\PY@tc##1{\textcolor[rgb]{0.00,0.50,0.00}{##1}}}
\def\PY@tok@s{\def\PY@tc##1{\textcolor[rgb]{0.73,0.13,0.13}{##1}}}
\def\PY@tok@kp{\def\PY@tc##1{\textcolor[rgb]{0.00,0.50,0.00}{##1}}}
\def\PY@tok@w{\def\PY@tc##1{\textcolor[rgb]{0.73,0.73,0.73}{##1}}}
\def\PY@tok@kt{\def\PY@tc##1{\textcolor[rgb]{0.69,0.00,0.25}{##1}}}
\def\PY@tok@ow{\let\PY@bf=\textbf\def\PY@tc##1{\textcolor[rgb]{0.67,0.13,1.00}{##1}}}
\def\PY@tok@sb{\def\PY@tc##1{\textcolor[rgb]{0.73,0.13,0.13}{##1}}}
\def\PY@tok@k{\let\PY@bf=\textbf\def\PY@tc##1{\textcolor[rgb]{0.00,0.50,0.00}{##1}}}
\def\PY@tok@se{\let\PY@bf=\textbf\def\PY@tc##1{\textcolor[rgb]{0.73,0.40,0.13}{##1}}}
\def\PY@tok@sd{\let\PY@it=\textit\def\PY@tc##1{\textcolor[rgb]{0.73,0.13,0.13}{##1}}}

\def\PYZbs{\char`\\}
\def\PYZus{\char`\_}
\def\PYZob{\char`\{}
\def\PYZcb{\char`\}}
\def\PYZca{\char`\^}
\def\PYZsh{\char`\#}
\def\PYZpc{\char`\%}
\def\PYZdl{\char`\$}
\def\PYZti{\char`\~}

\newcommand\lb{[}
\newcommand\rb{]}
\newcommand\PYbg[1]{\textcolor[rgb]{0.00,0.50,0.00}{\textbf{#1}}}
\newcommand\PYbf[1]{\textcolor[rgb]{0.73,0.40,0.53}{\textbf{#1}}}
\newcommand\PYbe[1]{\textcolor[rgb]{0.40,0.40,0.40}{#1}}
\newcommand\PYbd[1]{\textcolor[rgb]{0.73,0.13,0.13}{#1}}
\newcommand\PYbc[1]{\textcolor[rgb]{0.00,0.50,0.00}{\textbf{#1}}}
\newcommand\PYbb[1]{\textcolor[rgb]{0.40,0.40,0.40}{#1}}
\newcommand\PYba[1]{\textcolor[rgb]{0.00,0.00,0.50}{\textbf{#1}}}
\newcommand\PYaJ[1]{\textcolor[rgb]{0.73,0.13,0.13}{#1}}
\newcommand\PYaK[1]{\textcolor[rgb]{0.00,0.00,1.00}{#1}}
\newcommand\PYaH[1]{\fcolorbox[rgb]{1.00,0.00,0.00}{1,1,1}{#1}}
\newcommand\PYaI[1]{\textcolor[rgb]{0.69,0.00,0.25}{#1}}
\newcommand\PYaN[1]{\textcolor[rgb]{0.00,0.00,1.00}{\textbf{#1}}}
\newcommand\PYaO[1]{\textcolor[rgb]{0.00,0.00,0.50}{\textbf{#1}}}
\newcommand\PYaL[1]{\textcolor[rgb]{0.73,0.73,0.73}{#1}}
\newcommand\PYaM[1]{\textcolor[rgb]{0.74,0.48,0.00}{#1}}
\newcommand\PYaB[1]{\textcolor[rgb]{0.00,0.25,0.82}{#1}}
\newcommand\PYaC[1]{\textcolor[rgb]{0.67,0.13,1.00}{#1}}
\newcommand\PYaA[1]{\textcolor[rgb]{0.00,0.50,0.00}{#1}}
\newcommand\PYaF[1]{\textcolor[rgb]{1.00,0.00,0.00}{#1}}
\newcommand\PYaG[1]{\textcolor[rgb]{0.10,0.09,0.49}{#1}}
\newcommand\PYaD[1]{\textcolor[rgb]{0.25,0.50,0.50}{\textit{#1}}}
\newcommand\PYaE[1]{\textcolor[rgb]{0.63,0.00,0.00}{#1}}
\newcommand\PYaZ[1]{\textcolor[rgb]{0.00,0.50,0.00}{\textbf{#1}}}
\newcommand\PYaX[1]{\textcolor[rgb]{0.00,0.50,0.00}{#1}}
\newcommand\PYaY[1]{\textcolor[rgb]{0.73,0.13,0.13}{#1}}
\newcommand\PYaR[1]{\textcolor[rgb]{0.10,0.09,0.49}{#1}}
\newcommand\PYaS[1]{\textcolor[rgb]{0.25,0.50,0.50}{\textit{#1}}}
\newcommand\PYaP[1]{\textcolor[rgb]{0.49,0.56,0.16}{#1}}
\newcommand\PYaQ[1]{\textcolor[rgb]{0.40,0.40,0.40}{#1}}
\newcommand\PYaV[1]{\textcolor[rgb]{0.00,0.00,1.00}{\textbf{#1}}}
\newcommand\PYaW[1]{\textcolor[rgb]{0.73,0.13,0.13}{#1}}
\newcommand\PYaT[1]{\textcolor[rgb]{0.50,0.00,0.50}{\textbf{#1}}}
\newcommand\PYaU[1]{\textcolor[rgb]{0.82,0.25,0.23}{\textbf{#1}}}
\newcommand\PYaj[1]{\textcolor[rgb]{0.00,0.50,0.00}{#1}}
\newcommand\PYak[1]{\textcolor[rgb]{0.73,0.40,0.53}{#1}}
\newcommand\PYah[1]{\textcolor[rgb]{0.63,0.63,0.00}{#1}}
\newcommand\PYai[1]{\textcolor[rgb]{0.10,0.09,0.49}{#1}}
\newcommand\PYan[1]{\textcolor[rgb]{0.67,0.13,1.00}{\textbf{#1}}}
\newcommand\PYao[1]{\textcolor[rgb]{0.73,0.40,0.13}{\textbf{#1}}}
\newcommand\PYal[1]{\textcolor[rgb]{0.25,0.50,0.50}{\textit{#1}}}
\newcommand\PYam[1]{\textbf{#1}}
\newcommand\PYab[1]{\textit{#1}}
\newcommand\PYac[1]{\textcolor[rgb]{0.73,0.13,0.13}{#1}}
\newcommand\PYaa[1]{\textcolor[rgb]{0.50,0.50,0.50}{#1}}
\newcommand\PYaf[1]{\textcolor[rgb]{0.25,0.50,0.50}{\textit{#1}}}
\newcommand\PYag[1]{\textcolor[rgb]{0.40,0.40,0.40}{#1}}
\newcommand\PYad[1]{\textcolor[rgb]{0.73,0.13,0.13}{#1}}
\newcommand\PYae[1]{\textcolor[rgb]{0.40,0.40,0.40}{#1}}
\newcommand\PYaz[1]{\textcolor[rgb]{0.00,0.63,0.00}{#1}}
\newcommand\PYax[1]{\textcolor[rgb]{0.60,0.60,0.60}{\textbf{#1}}}
\newcommand\PYay[1]{\textcolor[rgb]{0.00,0.50,0.00}{\textbf{#1}}}
\newcommand\PYar[1]{\textcolor[rgb]{0.10,0.09,0.49}{#1}}
\newcommand\PYas[1]{\textcolor[rgb]{0.73,0.13,0.13}{\textit{#1}}}
\newcommand\PYap[1]{\textcolor[rgb]{0.00,0.50,0.00}{#1}}
\newcommand\PYaq[1]{\textcolor[rgb]{0.53,0.00,0.00}{#1}}
\newcommand\PYav[1]{\textcolor[rgb]{0.00,0.50,0.00}{\textbf{#1}}}
\newcommand\PYaw[1]{\textcolor[rgb]{0.40,0.40,0.40}{#1}}
\newcommand\PYat[1]{\textcolor[rgb]{0.10,0.09,0.49}{#1}}
\newcommand\PYau[1]{\textcolor[rgb]{0.40,0.40,0.40}{#1}}


% for compatibility with earlier versions
\def\PYZat{@}
\def\PYZlb{[}
\def\PYZrb{]}
\makeatother



%%%%%%%%%%%%%%%%%%%%%
% 1ere page
\begin{frame}[label=courant]
  \titlepage
\end{frame}

%%%%%%%%%%%%%%%%%%%%%%%%%%%%%%%%%%%%%%%%%%%%%%%%%%%%%%%%%%%%%%%%%%%%%%%%
%%%%%%%%%%%%%%%%%%%%%%%%%%%%%%%%%%%%%%%%%%%%%%%%%%%%%%%%%%%%%%%%%%%%%%%%
\begin{frame}
  \frametitle{Plan}

  \begin{itemize}
  \item 8 avril 2021 : \myhref{https://www.canal-u.tv/video/groupe_calcul/pourquoi_julia.60773}{Café Calcul - Pourquoi Julia ?}
  \item le problème des deux languages (script + bas niveau) $\Rightarrow$ \textcolor{violet}{\bf Julia}
  \item le problème des deux (en fait plus) modèles de programmation (en mémoire partagée) $\Rightarrow$ Portabilité de performance
  \end{itemize}

  \begin{center}
    \includegraphics[width=4cm]{julia/pourquoi_julia1}
    \includegraphics[width=4cm]{julia/pourquoi_julia2}
  \end{center}

\end{frame}


\section{Introduction - Kokkos concepts}

%%%%%%%%%%%%%%%%%%%%%%%%%%%%%%%%%%%%%%%%%%%%%%%%%%%%%%%%%%%%%%%%%%%%
%%%%%%%%%%%%%%%%%%%%%%%%%%%%%%%%%%%%%%%%%%%%%%%%%%%%%%%%%%%%%%%%%%%
%%%%%%%%%%%%%%%%%%%%%%%%%%%%%%%%%%%%%%%%%%%%%%%%%%%%%%%%%%%%%%%%%%%
% \begin{frame}
%   \frametitle{Kokkos: a programming model for performance portability}

%   \only<1>{
%     \begin{itemize}
%     \item \textcolor{blue}{\textbf{Kokkos}} is a \textbf{C++ library} with \textcolor{red}{\textbf{parallel algorithmic patterns}} AND \textcolor{red}{\textbf{data containers}} for \textcolor{blue}{\textbf{node-level parallelism}}.
%     \item Implementation relies heavily on \textbf{meta-programing} to derive native {\bf low-level code (OpenMP, Pthreads, CUDA, ...)} and adapt data structure {\bf memory layout} at compile-time
%     \item Core developers at \textcolor{violet}{\bf SANDIA NL} (\textbf{H.C. Edwards, C. Trott})
%     \end{itemize}
%   }
%   \only<2>{
%     \begin{itemize}
%     \item \textcolor{darkgreen}{\textbf{Open source}}, \myurl{https://github.com/kokkos/kokkos}
%     \item Primarily developped as a base building layer for \textbf{generic high-performance parallel linear algebra} in \myhref{https://github.com/trilinos/Trilinos}{Trilinos}
%     \item Also used in molecular dynamics code, e.g. \myhref{http://lammps.sandia.gov/}{LAMMPS}
%     \item Goal: \textcolor{orange}{\textbf{ISO/C++ 2020 Standard}} subsumes Kokkos abstractions~\footnote{see mdspan proposal \myurl{https://github.com/kokkos/array_ref}}
%     \end{itemize}
%   }
%   \begin{center}
%     \includegraphics<1-2>[width=7cm]{doc/perf_portability/kokkos_summary}
%   \end{center}

% \end{frame}

%%%%%%%%%%%%%%%%%%%%%%%%%%%%%%%%%%%%%%%%%%%%%%%%%%%%%%%%%%%%%%%%%%%
%%%%%%%%%%%%%%%%%%%%%%%%%%%%%%%%%%%%%%%%%%%%%%%%%%%%%%%%%%%%%%%%%%%
%%%%%%%%%%%%%%%%%%%%%%%%%%%%%%%%%%%%%%%%%%%%%%%%%%%%%%%%%%%%%%%%%%%
\begin{frame}
  \frametitle{Kokkos: a programming model for perf. portability}

  \only<1>{
    \begin{itemize}
    \item \textcolor{blue}{\textbf{Kokkos}} is a \textbf{C++ library} for \textcolor{violet}{\textbf{node-level parallelism}} (i.e. \textcolor{violet}{\bf shared memory}) providing:

      \begin{itemize}
      \item \textcolor{red}{\textbf{parallel algorithmic patterns}}
      \item \textcolor{red}{\textbf{data containers}}
      \end{itemize}
    \item Implementation relies heavily on \textbf{meta-programing} to derive native low-level code (OpenMP, Pthreads, CUDA, ...) and adapt data structure memory layout at compile-time
    \item Core developers at \textcolor{violet}{\textbf{SANDIA NL}} (\textbf{H.C. Edwards~\footnote{now hired @Nvidia}, C. Trott})
    \end{itemize}
  }
  \only<2>{
    \begin{itemize}
    \item \textcolor{darkgreen}{\textbf{Open source}}, \myurl{https://github.com/kokkos/kokkos}
    \item Primarily developped as a base building layer for \textbf{generic high-performance parallel linear algebra} in \myhref{https://github.com/trilinos/Trilinos}{Trilinos}
    \item Also used in
      \begin{itemize}
      \item \myhref{http://lammps.sandia.gov/}{LAMMPS} (molecular dynamics code),
      \item \myhref{https://github.com/NaluCFD/Nalu}{NALU CFD} (low-Mach wind flow),
      \item \myhref{https://sparta.sandia.gov/}{SPARTA/DSMC} (rarefied gas flow), \myhref{}{SPARC} (CFD, RANS, LES, hypersonic flow)
      \item \myhref{https://github.com/SNLComputation/Albany}{Albany} (fluid/solid,...)
      \item \myhref{http://uintah.utah.edu/}{Uintah} (structured AMR, combustion, radiation)
      \end{itemize}

    %\item Goal: \textcolor{orange}{\textbf{ISO/C++ 2020 Standard}} subsumes Kokkos abstractions~\footnote{\scriptsize see mdspan proposal \myurl{https://github.com/kokkos/array_ref}}
    \end{itemize}
  }
  \begin{columns}
    \begin{column}{0.34\linewidth}
      Goal: \textcolor{orange}{\textbf{ISO/C++ 2020 Standard}} subsumes Kokkos abstractions
    \end{column}
    \begin{column}{0.65\linewidth}
      \begin{center}
        \includegraphics<1-2>[width=5cm]{doc/perf_portability/kokkos_summary}
      \end{center}
    \end{column}
  \end{columns}
  \vfill
  {\scriptsize see mdspan proposal \myurl{https://github.com/kokkos/array_ref}}
\end{frame}

%%%%%%%%%%%%%%%%%%%%%%%%%%%%%%%%%%%%%%%%%%%%%%%%%%%%%%%%%%%%%%%%%%%
%%%%%%%%%%%%%%%%%%%%%%%%%%%%%%%%%%%%%%%%%%%%%%%%%%%%%%%%%%%%%%%%%%%
%%%%%%%%%%%%%%%%%%%%%%%%%%%%%%%%%%%%%%%%%%%%%%%%%%%%%%%%%%%%%%%%%%%
\begin{frame}
  \frametitle{Kokkos: a programming model for performance portability}

  \begin{center}
    \includegraphics[width=8.0cm]{../intro/images/Kokkos-Multi-CoE_slide3}
  \end{center}

  {\small reference: \myurl{https://cfwebprod.sandia.gov/cfdocs/CompResearch/docs/Kokkos-Multi-CoE.pdf}}

\end{frame}

%%%%%%%%%%%%%%%%%%%%%%%%%%%%%%%%%%%%%%%%%%%%%%%%%%%%%%%%%%%%%%%%%%%
%%%%%%%%%%%%%%%%%%%%%%%%%%%%%%%%%%%%%%%%%%%%%%%%%%%%%%%%%%%%%%%%%%%
%%%%%%%%%%%%%%%%%%%%%%%%%%%%%%%%%%%%%%%%%%%%%%%%%%%%%%%%%%%%%%%%%%%
\begin{frame}
  \frametitle{Kokkos: a programming model for performance portability}

  \begin{center}
    \includegraphics<1>[width=8.0cm]{./images/kokkos_timeline}
    \includegraphics<2>[width=8.0cm]{./images/kokkos_ecosystem}
  \end{center}

  {\small reference: \myurl{https://cfwebprod.sandia.gov/cfdocs/CompResearch/docs/Kokkos-Needs-Of-Apps.pdf}}

\end{frame}



% remind the difference between CPU and GPU +
% the need for Kokkos
%%%%%%%%%%%%%%%%%%%%%%%%%%%%%%%%%%%%%%%%%%%%%%%%
%%%%%%%%%%%%%%%%%%%%%%%%%%%%%%%%%%%%%%%%%%%%%%%%
\begin{frame}
  \frametitle{(Pre-)Exascale machines - architecture diversity !}

  \begin{itemize}
  \item \textcolor{red}{\bf \large US}: Summit , Sierra $\Rightarrow$ mostly OpenPower (IBM P9 + Nvidia V100), GPU-based architecture, \#2 and \#3 @top500; exascale machines announced
    \begin{itemize}
     \item \myhref{https://www.nextplatform.com/2019/03/18/intel-to-take-on-openpower-for-exascale-dominance-with-aurora/}{Aurora} (Argonne NL, 2022): Intel \myhref{https://www.nextplatform.com/2018/12/16/intel-unfolds-roadmaps-for-future-cpus-and-gpus/}{Xe GPU}
     \item \myhref{https://www.nextplatform.com/2019/05/07/cray-amd-tag-team-on-1-5-exaflops-frontier-supercomputer/}{Frontier} (Oak Ridge NL, 2021 ?): AMD EPYC + Radeon Instinct GPU
    \end{itemize}
  \item \textcolor{blue}{\bf \large China:}
    \begin{itemize}
    \item Phytium FT2000/64 ARM chips + Matrix2000 GPDSP accelerators $\Rightarrow$ \#6 @top500, Tianhe-2A, 61 PFlops
    \item 260-core Shenwei, \textcolor{blue}{\bf homegrow technology} hardware + software (C++/fortran compiler + OpenACC) $\Rightarrow$ \#4 @top500 , Sunway TaihuLight, 105 PFlops
    \item Dhyana, AMD-licenced x86 multicore (300 M\$), identical to AMD EPYC
      % https://www.top500.org/news/china-reveals-third-exascale-prototype/
    \end{itemize}

  \item \textcolor{violet}{\bf \large Japan:} \myhref{https://postk-web.r-ccs.riken.jp/spec.html}{\bf Fugaku}(Fujitsu, ARM, RIKEN)  A64FX ARM (\textcolor{violet}{\bf home grown}, started in 2014, \textcolor{violet}{\bf \#1 @top500 (Nov. 2020)}, 900 M\$), GPU, etc ...
    % https://www.hpcwire.com/2018/09/05/no-go-for-glofo-at-7nm-and-the-fujitsu-a64fx-post-k-cpu/
    % https://www.top500.org/news/fujitsu-reveals-details-of-processor-that-will-power-post-k-supercomputer/
    % http://www.fujitsu.com/jp/Images/20180821hotchips30.pdf
    % https://www.theregister.co.uk/2018/08/22/fujitsu_post_k_a64fx/
  \item \textcolor{darkgreen}{\bf \large Europe} : new organization EuroHPC (2018), EC H2020 budget ($\sim$ 500 M\euro{} per year)\\
    \textcolor{darkgreen}{\bf home grown} (EPI) ARM and RISC-V architecture, early stage%just starting development
    % https://www.top500.org/news/european-program-to-develop-supercomputing-chips-begins-to-take-shape/
    % mettre une EPI roadmap
    % EuroHPC (Nov. 2018 - 2026)
    % EuroHPC, Exascale not before 2022 : https://www.youtube.com/watch?v=y7_VvcIKJnI
    % EuroHPC 2 exascale machines (500 M€)
    % EuroHPC 2 pre-exascale machines in 2021 (240 M€)
  \end{itemize}

  %{\tiny \myurl{https://www.nextplatform.com/2016/07/11/chinas-triple-play-pre-exascale-systems/}}

\end{frame}

%%%%%%%%%%%%%%%%%%%%%%%%%%%%%%%%%%%
%%%%%%%%%%%%%%%%%%%%%%%%%%%%%%%%%%%
\begin{frame}
  \frametitle{Motivations for performance portability}

  \begin{itemize}
  \item {\Large What is performance portability ?}
    \begin{itemize}
    \item \textcolor{violet}{\large \bf (Re)write your code once, (try to) run {\it efficiently} everywhere}
    \item By everywhere, we mean : Multicore Intel/ARM and Nvidia/AMD GPUs
    \item \textcolor{violet}{\bf High-level approach:} as much as possible (if possible) hide hardware details to the (physicist / applied math) software developer
    \item \myhref{https://performanceportability.org}{https://performanceportability.org}
    \item \myhref{https://asc.llnl.gov/doe-coe-mtg-2016}{1st annual DOE Performance Portability Meeting (2016)}
    \end{itemize}
  \item {\large Is that \textcolor{darkgreen}{\bf possible} ?}\\ How ?\\ Which programming model ?\\ Which language ?\\ Which compiler ? $\Rightarrow$ large combinatorics
  \item for the rest of this talk, i'll focus on the \myhref{https://github.com/kokkos/kokkos}{kokkos/C++} library
  \end{itemize}

\end{frame}

%%%%%%%%%%%%%%%%%%%%%%%%%%%%%%%%%%%%%%%%%%%%%%%%
%%%%%%%%%%%%%%%%%%%%%%%%%%%%%%%%%%%%%%%%%%%%%%%%
\begin{frame}
  \frametitle{Parallel programming models landscape}

  \begin{minipage}{0.73\linewidth}
    \begin{itemize}
    \item \textcolor{red}{\textbf{Low-level native language:}} \myhref{https://www.khronos.org/opencl/}{OpenCL}, \myhref{https://developer.nvidia.com/cuda-downloads}{CUDA}, \myhref{https://rocmdocs.amd.com/en/latest/Programming_Guides/Programming-Guides.html}{HIP}
    \item \textcolor{orange}{\textbf{Directive approach (code annotations)}} for multicore/GPU, ...:
      \begin{itemize}
      \item \myhref{http://www.openmp.org/}{OpenMP} 5.1 (Clang, PGI, GNU, ...), \myhref{https://pm.bsc.es/ompss-2}{OmpSs-2}
      \item \myhref{http://www.openacc.org/}{OpenACC} 2.7 (PGI, GNU, ...) \textcolor{blue}{$\Rightarrow$ Fortran codes.}
      \end{itemize}
    \item \textcolor{darkgreen}{\textbf{Other high-level library-based approaches}}:
      {
        \scriptsize
        \begin{itemize}
        \item \framebox{\myhref{https://github.com/kokkos/kokkos}{Kokkos}}, \myhref{https://github.com/LLNL/RAJA}{RAJA}, \myhref{https://github.com/alpaka-group/alpaka}{Alpaka}, \myhref{https://github.com/STEllAR-GROUP/hpx}{HPX}, \myhref{https://github.com/GridTools/gridtools}{GridTools}, \myhref{https://arrayfire.com/}{ArrayFire}...
        \item \myhref{https://www.khronos.org/sycl}{SYCL} (Khronos Group \textit{standard}), C++ high-level layer on top of OpenCL. %\textcolor{red}{\bf Still $\sim$experimental}\\
          \myhref{https://software.intel.com/content/www/us/en/develop/tools/oneapi.html}{Intel OneAPI/DPCPP} (Intel CPU/GPU/FPGA, Nvidia GPUs),\\
          \myhref{https://www.codeplay.com/products/computesuite/computecpp}{CodePlay},
          \myhref{https://github.com/illuhad/hipSYCL}{AMD and Nvidia GPUs},
          \myhref{https://github.com/keryell/triSYCL}{Keryell/Xilinx}
        \item {\bf C++-17 built-in parallelism for multicore and GPUs}, e.g.:
          \begin{itemize}
          \item Nvidia's \myhref{https://developer.nvidia.com/hpc-sdk}{hpc-sdk} (May 2020)
          \item \myhref{https://github.com/oneapi-src/oneTBB}{Intel OneAPI/TBB}
          \end{itemize}
        \end{itemize}
      }
    \end{itemize}
  \end{minipage}
  %
  \begin{minipage}{0.26\linewidth}
    \includegraphics[width=1.3\linewidth]{./images/exascale/kokkos_backends}

    \includegraphics[width=1.3\linewidth]{./tikz/sycl}

    %\textcolor{blue}{\bf \scriptsize All these programming solutions are interconnected and inter-dependent !}

    %\begin{framed}
       {
          \scriptsize
          {\bf additionnal features:}\\
          \textcolor{violet}{\bf memory management,}\\
          \textcolor{violet}{\bf data containers}, ...
       }
    %\end{framed}

  \end{minipage}

\end{frame}

%%%%%%%%%%%%%%%%%%%%%%%%%%%%%%%%%%%%%%%%%%%%%%%%%%%%%%%%%%%%%%%%%%%
%%%%%%%%%%%%%%%%%%%%%%%%%%%%%%%%%%%%%%%%%%%%%%%%%%%%%%%%%%%%%%%%%%%
%%%%%%%%%%%%%%%%%%%%%%%%%%%%%%%%%%%%%%%%%%%%%%%%%%%%%%%%%%%%%%%%%%%
\begin{frame}[fragile=singleslide]
  \frametitle{Kokkos (2010), before C++-11 and lambdas}

  \begin{itemize}
  \item Before 2010, starts as a refactoring of Trilinos (10.4), \textcolor{darkgreen}{abstract concept of {\tt Node}} (generic for SerialNode, TBBNode or CUDANode)\\
    conf paper: \myhref{https://www.researchgate.net/publication/221392413_A_Light-weight_API_for_Portable_Multicore_Programming}{A light-weight API for portable Multicore Programming}
    {\small
    \begin{minted}[autogobble=true]{c++}
      // data-work struct
      template <class Node>
      class AxpyOp {
        Node::buffer x,y;
        double alpha,beta;
        void execute(int i);
      };

      template <>
      void SomeNode::parallel_for<AxpyOp>(int begin, int end, AxpyOp wd) {
        // node specific implementation
        // if SomeNode == TBB, then call TBB API
        // if SomeNode == CUDA, then call Cuda thrust::for_each
        // ...
      }
    \end{minted}
  }
  \end{itemize}
\end{frame}

%%%%%%%%%%%%%%%%%%%%%%%%%%%%%%%%%%%%%%%%%%%%%%%%%%%%%%%%%%%%%%%%%%%
%%%%%%%%%%%%%%%%%%%%%%%%%%%%%%%%%%%%%%%%%%%%%%%%%%%%%%%%%%%%%%%%%%%
%%%%%%%%%%%%%%%%%%%%%%%%%%%%%%%%%%%%%%%%%%%%%%%%%%%%%%%%%%%%%%%%%%%
\begin{frame}
  \frametitle{Kokkos: a programming model for perf. portability}

  \only<1>{
    \begin{itemize}
    \item \textcolor{blue}{\textbf{Kokkos}} is a \textbf{C++ library} for \textcolor{violet}{\textbf{node-level parallelism}} (i.e. \textcolor{violet}{\bf shared memory}) providing abstractions for {\bf harware-aware}:

      \begin{itemize}
      \item \textcolor{red}{\textbf{parallel algorithmic patterns}}
      \item \textcolor{red}{\textbf{data containers}}
      \end{itemize}
    \item \myurl{https://kokkos.org/}
    \item Implementation relies heavily on \textbf{C++ meta-programing} to derive native low-level code (OpenMP, CUDA, HIP, SYCL...) and adapt data structure memory layout at compile-time
    \item Developped at \textcolor{violet}{\textbf{Sandia NL}} (core, CUDA, OpenMP), \textcolor{violet}{\textbf{ORNL}} (HIP, SYCL), ...
    \end{itemize}

    \textcolor{darkgreen}{\bf Goal:} {\bf write one implementation which:}
    \begin{itemize}
    \item compiles and \textcolor{blue}{\bf run on multiple archs},
    \item obtains \textcolor{blue}{\bf performant memory access pattern} across archs,
    \item can leverage \textcolor{blue}{\bf arch-specific features} where possible.
    \end{itemize}

  }
  \only<2>{
    \begin{itemize}
    \item \textcolor{darkgreen}{\textbf{Open source}}, \myurl{https://github.com/kokkos/kokkos} %, since $\sim 2012$
    \item Primarily developped as a base building layer for \textbf{generic high-performance parallel linear algebra} in \myhref{https://github.com/trilinos/Trilinos}{Trilinos}
    \item Used in, e.g.:
      \begin{itemize}
      \item \myhref{http://lammps.sandia.gov/}{LAMMPS} (molecular dynamics code),
        %\myhref{https://github.com/ECP-copa/ExaMiniMD}{ExaMiniMD}
      \item \myhref{https://github.com/NaluCFD/Nalu}{NALU CFD} (low-Mach wind flow),
      \item \myhref{https://sparta.sandia.gov/}{SPARTA/DSMC} (rarefied gas flow), \myhref{}{SPARC} (CFD, RANS, LES, hypersonic flow)
      \item \myhref{https://github.com/SNLComputation/Albany}{Albany} (fluid/solid,...)
      \item \myhref{http://uintah.utah.edu/}{Uintah} (structured AMR, combustion, radiation)
        % \item list of codes using Kokkos: \myurl{https://github.com/kokkos/kokkos/issues/1950}
      %\item \myhref{https://github.com/kokkos/kokkos/issues/1950}{list of codes using Kokkos}
      \end{itemize}

      % \item Goal: \textcolor{orange}{\textbf{ISO/C++ 2020 Standard}} subsumes Kokkos abstractions~\footnote{\scriptsize see mdspan proposal \myurl{https://github.com/kokkos/array_ref}}
    \end{itemize}
  }

  \only<2>{
  \begin{columns}
    \begin{column}{0.34\linewidth}
      Strong involvement in \textcolor{orange}{\textbf{ISO/C++ 2020 Standard}}\\
      Make Kokkos a sliding window of future c++ features
    \end{column}
    \begin{column}{0.65\linewidth}
      \begin{center}
        \includegraphics<1-2>[width=5cm]{images/perf_portability/kokkos_summary}
      \end{center}
    \end{column}
  \end{columns}
  \vfill
  {
    \scriptsize see mdspan proposal \myurl{https://github.com/kokkos/mdspan}\\
    \myurl{https://arxiv.org/abs/2010.06474}
  }
}

\end{frame}

%%%%%%%%%%%%%%%%%%%%%%%%%%%%%%%%%%%%%%%%%%%%%%%%%%%%%%%%%%%%%%%%%%%
%%%%%%%%%%%%%%%%%%%%%%%%%%%%%%%%%%%%%%%%%%%%%%%%%%%%%%%%%%%%%%%%%%%
%%%%%%%%%%%%%%%%%%%%%%%%%%%%%%%%%%%%%%%%%%%%%%%%%%%%%%%%%%%%%%%%%%%
\begin{frame}
  \frametitle{Kokkos: a programming model for perf. portability}

  \begin{center}
    \includegraphics[width=0.8\linewidth]{images/kokkos/trott1}
  \end{center}

  { \tiny
    \begin{itemize}
    \item \myhref{https://github.com/kokkos/kokkos-kernels}{Kokkos-kernels} (many dense/sparse BLAS problems, ...), \myhref{https://github.com/kokkos/simd-math}{simd-math}, \myhref{https://github.com/ECP-copa/Cabana}{Cabana} (for particle-based codes)
    \item \myhref{https://github.com/kokkos/kokkos-fortran-interop}{Fortran compatibility layer} (REX code \myhref{https://ecpannualmeeting.com/assets/overview/sessions/XGC_ECP_2020.pptx}{XGC-Cabana}, Plasma physics, Gyrocinetics, particle-in-cell)
    \item \myhref{https://github.com/kokkos/pykokkos-base}{pykokkos-base} (\myhref{https://github.com/pybind/pybind11}{pybind11}-based API mapping + memory, numpy/cupy interoperability), \myhref{https://github.com/kokkos/pykokkos}{pykokkos} (decorator + python to C++ translation)
    \item \myhref{https://prod-ng.sandia.gov/techlib-noauth/access-control.cgi/2017/1710464.pdf}{Task-DAG parallelism (CPU / GPU)}
    \end{itemize}
  }

  source: C. Trott, DOE Performance Portability Meeting, April 2019
\end{frame}

%%%%%%%%%%%%%%%%%%%%%%%%%%%%%%%%%%%
%%%%%%%%%%%%%%%%%%%%%%%%%%%%%%%%%%%
% \begin{frame}
%   \frametitle{C++ Kokkos library summary}

%   \begin{itemize}
%   %\item See GTC2017 session \textcolor{violet}{S7344 - Kokkos ? The C++ Performance Portability Programming Model} (C. Trott and H.C. Edwards).
%   \item Framework for efficient \textcolor{RedOrange}{\bf node-level parallelism (CPU, GPU, ...)}
%   \item Provides
%     \begin{itemize}
%     \item \textcolor{blue}{\bf Computationnal parallel patterns} (for, reduce, scan, ...)
%     \item \textcolor{violet}{\bf Hardware aware memory containers}: e.g. {\bf A multi-dimensionnal data container with hardware adapted memory layout}
%     \item Support for multiple backends:
%       {\scriptsize
%         \begin{itemize}
%         \item OpenMP (x86, ARM, IBM, ...)
%         \item pthreads
%         \item OpenMP target (GPU, ...),
%         \item CUDA (Nvidia GPU, ...),
%         \item HIP (AMD and Nvidia GPU),
%         \item SYCL (Intel CPU and GPU, Nvidia, ...)
%         \item HPX
%         \end{itemize}
%       }
%     \item Additionnal sub-projects: \myhref{https://github.com/kokkos/kokkos-kernels}{kokkos-kernels} (BLAS, sparse BLAS, Graph), \myhref{https://github.com/kokkos/simd-math}{Kokkos::simd}, \myhref{https://github.com/kokkos/kokkos-python}{python bindings}, ...

%     \end{itemize}
%   % \item Provides some {\bf \textcolor{RedOrange}{high-level (abstract) concepts}} as template C++ classes:
%   %   \begin{itemize}
%   %   \item A \textcolor{red}{\bf kokkos device:} \texttt{Kokkos::Cuda, Kokkos::OpenMP, Kokkos::Pthreads, Kokkos::Serial},...
%   %   \item concepts controlled by C++ template meta-programing: \textcolor{darkgreen}{\bf execution space, memory space, memory layout, ...}
%   %   \item \textcolor{blue}{\bf Computationnal parallel patterns} (for, reduce, scan, ...) controlled with a {\bf execution policy} (i.e. how many iterations, teams, ...)
%   %   \end{itemize}
%   % \item \textcolor{violet}{\bf \texttt{Kokkos::View}}: {\bf A multi-dimensionnal data container with hardware adapted memory layout} \\
%   %   % with ability to mirror view between memory spaces, ...
%   %   {\small
%   %     -\;\textcolor{violet}{\texttt{Kokkos::View<double **> data("data",NX,NY);}} // 2D array with sizes known at runtime\\
%   %     -\;\framebox{{\bf How do I access data ?} \texttt{data(i,j)} !}
%   %   }
%   \item Mostly a header library (C++ metaprogramming)
%   %\item build system / module load / etc ... (?)
%   \end{itemize}

% \end{frame}

%%%%%%%%%%%%%%%%%%%%%%%%%%%%%%%%%%%
%%%%%%%%%%%%%%%%%%%%%%%%%%%%%%%%%%%
% \begin{frame}
%   \frametitle{C++ Kokkos library summary}

%   % I started using Kokkos in June 2016, shortly developped 2D Hydro MUSCL-Hancok 2nd order scheme.

%   \begin{itemize}
%   \item {\bf What does it mean hardware aware memory containers ?}
%   \item Most commonly in a C/C++, {\bf multi-dimensionnal array access} is done through {\bf \textcolor{red}{index linearization}} (row or column-major in 2D):
%     $$ \mathbf{index} = \mathbf{i} + nx * \mathbf{j}  + nx*ny * \mathbf{k}$$
%   \item {\bf Fortran} (column-major format) vs {\bf C/C++} (row-major format)
%   \item There is no reason to favour one layout versus the other
%     \begin{itemize}
%     \item column-major is better for vectorization on CPU architecture
%     \item row-major is better for high througput architecture e.g. GPU (memory coalescence)
%     \item $\Rightarrow$ \textcolor{orange}{\bf Kokkos allows to chose memory layout at compile time}
%     \end{itemize}
%   \item In Kokkos, one should/must avoid this index linearization at the user level, let \texttt{Kokkos::View} do this job (\textcolor{darkgreen}{\bf decided at compile-time, hardware adapted})
%     $$ \text{data}(i,j,k) $$
%     % \begin{itemize}
%     % \item 1D \texttt{Kokkos::View} with user-defined index linearization + 1D Iteration range
%     % \item \textcolor{blue}{2D \texttt{Kokkos::View}  + 1D flat Iteration range} {\bf (used in this work)}
%     % \item 2D \texttt{Kokkos::View}  + 2D (\texttt{Kokkos::MDRange} Kernel policy) : still an experimental feature
%     % \end{itemize}
%   %\item \texttt{Kokkos::MDRange} is functional, but was generating kernels with some performance loss, will surely be solved shortly by Kokkos core developpers.
%   %\item See also new developpement on hierarchical task-data parallelism, session S7253 (Monday 8th, room 211B).
%   \end{itemize}

%   % Some personnal views / experience as an external Kokkos user

% \note[item]{Let me brievely add a comment on memory layout, which is an important feature. With kokkos, array linearization is a low-level implementation detail, that must be abstracted away, if I may say. This way, the end-programmer only access data using a multi-index set, linearization is done in an hardware-aware way by kokkos at compile time, once the target device has been specified.}

% \end{frame}

%%%%%%%%%%%%%%%%%%%%%%%%%%%%%%%%%%%%%%%%%%%%%%%%%%%%%%%%%%%%%%%%%%%%%%%%
%%%%%%%%%%%%%%%%%%%%%%%%%%%%%%%%%%%%%%%%%%%%%%%%%%%%%%%%%%%%%%%%%%%%%%%%
\begin{frame}
  \frametitle{Kokkos - Documentation}

  \begin{itemize}
  %\item PDF documentation in kokkos source tree : \texttt{doc/Kokkos\_PG.pdf} (programming guide)
  %\item \myhref{http://www.stack.nl/~dimitri/doxygen/}{Doxygen} can only be built from inside \myhref{https://github.com/trilinos/Trilinos}{Trilinos source tree}\\
   % Version of the day can be browsed at \myurl{https://trilinos.org/docs/dev/packages/kokkos/doc/html/index.html}
  \item \myhref{https://github.com/kokkos/kokkos-tutorials/wiki/Kokkos-Lecture-Series}{Kokkos video lectures + slides} : \myurl{https://github.com/kokkos/kokkos-tutorials/wiki/Kokkos-Lecture-Series}
  \item \myhref{https://github.com/kokkos/kokkos-tutorials}{Kokkos tutorial} : \myurl{https://github.com/kokkos/kokkos-tutorials}
  \item Kokkos source code itself, reading unit tests code is also very helpful
  \end{itemize}

\end{frame}


% %%%%%%%%%%%%%%%%%%%%%%%%%%%%%%%%%%%
% %%%%%%%%%%%%%%%%%%%%%%%%%%%%%%%%%%%
% \begin{frame}
%   \frametitle{Main difference between CPU and GPU}

%   \begin{center}
%     \textbf{\textcolor{blue}{Architecture design differences between manycore GPUs and general purpose multicore CPU ?}}
    
%     \only<1-4>{\includegraphics[height=3.5cm]{images/cpu_gpu_comparison}}

%   \end{center}

% \begin{itemize}

%   \only<1>{
%   \item \textbf{Different goals produce different designs:}
%     \begin{itemize}
%     \item \textcolor{red}{\textbf{CPU}}  must be good at everything, parallel or not
%     \item \textcolor{blue}{\textbf{GPU}} assumes work load is highly parallel
%     \end{itemize}
%   }
%   \only<2>{
%   \item \textcolor{red}{\textbf{CPU}} design goal : optimize architecture for sequential code
%     performance : \textcolor{red}{minimize latency experienced by \textbf{1 thread}}
%     % latence = nb de cycle d'horloge pour acheminer des donnees de la memoire centrale jusqu'au coeur de calcul (ALU)
%     % donner l'illusion d'une memoire tres rapide
%     % exploiter la localite spatiale et temporelle de la memoire
%     % s'il n'y avait pas de cache, les coeurs de proc serait sans arret en "stall"
%     % voir l'article: http://dl.acm.org/citation.cfm?id=2692965.2682585
%     % Rethinking caches for throughput processors: technical perspective
%     % by Stephen W. Keckler

%     \begin{itemize}
%     \item \textcolor{red}{sophisticated} (i.e. large chip area) \textcolor{red}{control logic} for instruction-level parallelism
%       (branch prediction, out-of-order instruction, etc...)
%     \item \textcolor{red}{CPU have large cache memory} to reduce the instruction and
%       data access latency
%     \end{itemize}
%   }

%   \only<3>{
%   \item \textcolor{blue}{\textbf{GPU}} design goal : \textcolor{blue}{maximize throughput of \textbf{all threads}}
%     \begin{itemize}
%     \item \# threads in flight limited by resources => lots of
%       resources (registers, bandwidth, etc.)
%     \item  multithreading can \textcolor{blue}{hide latency} => skip the big caches
%     \item \textcolor{blue}{share control logic} across many threads
%     \end{itemize}
%   }
%   \only<4>{
%   \item \textcolor{red}{\bf CPU} : {\bf 1 thread $\Leftrightarrow$ 1 core}
%     \hspace{0.2\textwidth}\textcolor{blue}{\bf GPU}: {\bf \# threads $\gg$ \# cores}
%   }
% \end{itemize}

% %\only<4>{
% %\item \textcolor{red}{\bf CPU} : {\bf 1 thread $\Leftrightarrow$ 1 core}
% %\item \textcolor{blue}{\bf GPU}: {\bf \# threads $\gg$ \# cores}

%   % \begin{colums}
%   %   \begin{colum}{0.48\textwidth}
%   %     \textcolor{red}{\bf CPU} : {\bf 1 thread $\Leftrightarrow$ 1 core}
%   %   \end{colum}
%   %   \begin{colum}{0.48\textwidth}
%   %     \textcolor{blue}{\bf GPU}: {\bf \# threads $\gg$ \# cores}
%   %   \end{colum}
%   % \end{colums}
% %}
  
% \end{frame}

%%%%%%%%%%%%%%%%%%%%%%%%%%%%%%%%%%%%%%%%%%%%%%%%
%%%%%%%%%%%%%%%%%%%%%%%%%%%%%%%%%%%%%%%%%%%%%%%%
% \begin{frame}
%   \frametitle{Kokkos portability: leveraging hardware specific features}

%   \begin{center}
%     \textcolor{violet}{\Large Memory layouts / index linearization: e.g. 2D}
%   \end{center}

%   \begin{columns}
%     \begin{column}{0.5\textwidth}
%       \begin{center}
%         \textcolor{red}{\large row-major}

%         \includegraphics[width=5cm]{tikz/row-major}

%         $\text{index} = i + n_x j$,\\
%         \textcolor{red}{left layout}\\
%         fast index on the left
%       \end{center}
%     \end{column}
%     %
%     \begin{column}{0.5\textwidth}
%       \begin{center}
%         \textcolor{blue}{\large column-major}

%         \includegraphics[width=5cm]{tikz/col-major}

%         $\text{index} = j + n_y i$,\\
%         \textcolor{blue}{right layout}\\
%         fast index on the right
%       \end{center}
%     \end{column}
%   \end{columns}

% \end{frame}

%%%%%%%%%%%%%%%%%%%%%%%%%%%%%%%%%%%%%%%%%%%%%%%%
%%%%%%%%%%%%%%%%%%%%%%%%%%%%%%%%%%%%%%%%%%%%%%%%
\begin{frame}[fragile=singleslide]
  %\frametitle{Kokkos portability: leveraging hardware specific features}
  \frametitle{Illustrating portability with Kokkos}

  % \begin{block}{}
  %   {\large \textcolor{Red}{\bf Question:}
  %   Assuming 2d data, \textcolor{red}{left layout}, which loop would you prefer to parallelize (inner or outer) ?}
  % \end{block}

  \begin{center}
    \begin{minipage}{0.5\textwidth}
      \begin{minted}[autogobble=true]{c++}
    for(int j=0; j<ny; ++j)
      for(int i=0; i<nx; ++i)
        data[i+nx*j] += 42;
      \end{minted}
    \end{minipage}
  \end{center}

  \begin{columns}
    \begin{column}{0.45\textwidth}
      \begin{block}{}
        { \small
          \textcolor{Red}{\bf Question:}
          Assuming 2d data with \textcolor{red}{left layout}, but only 1 loop to parallelize, which one would you prefer to parallelize (inner or outer) ?}
      \end{block}

      \begin{center}
        \textcolor{red}{left-layout $=$ row-major}

        \includegraphics[width=5cm]{tikz/row-major}

        %$\text{index} = i + n_x j$,\\
        %\textcolor{red}{left layout}\\
        %fast index on the left
      \end{center}
    \end{column}
    %\hspace{-1.5cm}
    \begin{column}{0.49\textwidth}
\begin{block}{}
  \textcolor{blue}{\bf Answer:}\\
  {\bf Optimize memory access pattern !}
\end{block}
{
  \scriptsize
  \begin{itemize}
  \item \textcolor{violet}{maximize cache usage + SIMD for CPU}
  \item \textcolor{darkgreen}{maximize memory coalescence on GPU}
  \end{itemize}
}
\begin{center}
  \begin{block}{}
    {\bf Different hardware $\Rightarrow$}\\
    \textcolor{orange}{\bf Different parallelization strategies}
  \end{block}

  \includegraphics[width=5cm]{images/perf_portability/cpu_gpu_comparison}
\end{center}
\end{column}
    \hfill
  \end{columns}
\end{frame}

%%%%%%%%%%%%%%%%%%%%%%%%%%%%%%%%%%%%%%%%%%%%%%%%
%%%%%%%%%%%%%%%%%%%%%%%%%%%%%%%%%%%%%%%%%%%%%%%%
\begin{frame}[fragile=singleslide]
  %\frametitle{Multidimensional array and data parallelism}
  \frametitle{Illustrating portability with Kokkos}

  \begin{block}{}
    {\textcolor{Red}{\bf Question:}
    Assuming 2d data, \textcolor{red}{left layout}, which loop would you prefer to parallelize (inner or outer) ?}
  \end{block}

  \begin{columns}
    \begin{column}{0.48\textwidth}
      \begin{center}
        \textcolor{violet}{\bf OpenMP // outer loop}
        %each OpenMP thread handles {\bf 1 or more row(s)}

        \begin{minipage}{1.0\textwidth}
      {
        \scriptsize
        \begin{minted}[autogobble=true]{c++}
#pragma omp parallel
{
  #pragma omp for
  for(int j=0; j<ny; ++j)
     #pragma omp simd ivdep
    for(int i=0; i<nx; ++i)
      data[i+nx*j] += 42;
}
        \end{minted}
      }
      \end{minipage}
      \end{center}

      \begin{center}
        %\textcolor{red}{\large row-major}

        \includegraphics[width=3.0cm]{tikz/row-major-openmp}

        %$\text{index} = i + n_x j$,\\
        %\textcolor{red}{left layout}\\
        %fast index on the left
      \end{center}
    \end{column}
    %\hspace{-1.0cm}
    \begin{column}{0.48\textwidth}
      \begin{center}
        \textcolor{darkgreen}{\bf CUDA // inner loop}
        %each CUDA thread handles {\bf1 or more col(s)}\\
        %memory coalescence
      \end{center}
      {\tiny
        \begin{minted}[autogobble=true]{c++}
__global__ void compute(int *data)
{
  // adjacent memory cells
  // computed by adjacent threads
  int i = threadIdx.x + blockIdx.x*blockDim.x;

  for(int j=0; j<ny; ++j)
    data[i+nx*j] += 42;
}
\end{minted}
}
      \begin{center}
        %\textcolor{red}{\large row-major}

        \includegraphics[width=3cm]{tikz/row-major-cuda}

        %$\text{index} = i + n_x j$, \textcolor{red}{left layout}\\
        %fast index on the left
      \end{center}

    \end{column}
    \hfill
  \end{columns}
\end{frame}
%%%%%%%%%%%%%%%%%%%%%%%%%%%%%%%%%%%%%%%%%%%%%%%%
%%%%%%%%%%%%%%%%%%%%%%%%%%%%%%%%%%%%%%%%%%%%%%%%
% \begin{frame}[fragile=singleslide]
%   %\frametitle{Multidimensional array and data parallelism}
%   \frametitle{Illustrating portability with Kokkos}

%   \begin{block}{Conclusion:}
%     {\Large Don't assume \textcolor{red}{layout}, let's \textcolor{red}{chose} at compile-time !}
%   \end{block}

%   \begin{itemize}
%   \item {\bf First conclusion:}\\
%     if we keep the same memory layout, \textcolor{violet}{\bf OpenMP} and \textcolor{darkgreen}{\bf CUDA} \textcolor{red}{\bf disagree} on which loop should be parallelized to optimize for their respective hardware target.
%   \item \textcolor{blue}{\bf How can we make portable code ?}
%   \item Note that swapping memory layout and {\tt for loops} is {\bf involutive}
%   %\item instead of swapping loops, swap memory layout
%   \item \textcolor{orange}{\bf Kokkos answer:} make memory layout abstract (since a good memory layout is hardware dependent), fixed at compile-time\\
%     access $data(i,j)$
%     \begin{itemize}
%     \item On \textcolor{violet}{\bf OpenMP} $data(i,j)$ actually means accessing $dataPtr[Ny*i+j]$
%     \item On \textcolor{darkgreen}{\bf Cuda} $data(i,j)$ actually means accessing $dataPtr[i+Nx*j]$
%     \end{itemize}
%   \end{itemize}

% \end{frame}

%%%%%%%%%%%%%%%%%%%%%%%%%%%%%%%%%%%%%%%%%%%%%%%%
%%%%%%%%%%%%%%%%%%%%%%%%%%%%%%%%%%%%%%%%%%%%%%%%
\begin{frame}[fragile=singleslide]
  %\frametitle{Multidimensional array and data parallelism}
  \frametitle{Illustrating portability with Kokkos}

  \begin{center}
    {\large Let's \textcolor{red}{chose} memory layout at compile-time\\
    Make it hardware aware.}
  \end{center}

  \begin{columns}
    \begin{column}{0.45\textwidth}
      \begin{center}
        \textcolor{black}{\large left layout / CUDA}\\
        \includegraphics[width=2.8cm]{tikz/row-major-cuda}

        \textcolor{black}{\large right layout / OpenMP}\\
        \includegraphics[width=2.8cm]{tikz/col-major-kokkos-openmp}

      \end{center}
    \end{column}
    %\hspace{-1.5cm}
    \begin{column}{0.45\textwidth}
      \begin{center}
        \textcolor{orange}{\bf Kokkos single parallel version (CUDA+OpenMP)}\\
        {
          \scriptsize
          Kokkos/CUDA defaults to \textcolor{darkgreen}{\bf left-layout}\\
          Kokkos/OpenMP defaults to \textcolor{violet}{\bf right-layout}
        }
      \end{center}
      \begin{minted}{c++}
Kokkos::parallel_for(nx,
   KOKKOS_LAMBDA(int i) {
     for (int j=0; j<ny; ++j)
       data(i,j) += 42;
   }
);
      \end{minted}
    \end{column}
    \hfill
  \end{columns}
\end{frame}

%%%%%%%%%%%%%%%%%%%%%%%%%%%%%%%%%%%%%%%%%%%%%%%%
%%%%%%%%%%%%%%%%%%%%%%%%%%%%%%%%%%%%%%%%%%%%%%%%
% \begin{frame}
%   %\frametitle{Multidimensional array and data parallelism}
%   \frametitle{Kokkos summary}

%   \begin{center}
%     \includegraphics<1>[width=0.8\linewidth]{./images/kokkos/kokkos_pattern1}
%     \includegraphics<2>[width=0.8\linewidth]{./images/kokkos/kokkos_pattern2}
%   \end{center}

% \end{frame}


%%%%%%%%%%%%%%%%%%%%%%%%%%%%%%%%%%%%%%%%%%%%%%%%%%%%%%%%%%%%%%%%%%% 
%%%%%%%%%%%%%%%%%%%%%%%%%%%%%%%%%%%%%%%%%%%%%%%%%%%%%%%%%%%%%%%%%%% 
%%%%%%%%%%%%%%%%%%%%%%%%%%%%%%%%%%%%%%%%%%%%%%%%%%%%%%%%%%%%%%%%%%% 
\begin{frame}
  \frametitle{Kokkos Concepts (1) - the abstract machine model}

  \begin{itemize}
  \item Kokkos defines an abstract machine model for future large shared-memory nodes made of 
    \begin{itemize}
    \item \textcolor{blue}{\textbf{latency-oriented cores}} (contemporary CPU core)
    \item \textcolor{orange}{\textbf{throughput-oriented cores}} (GPU, ...)
    \end{itemize}
  \end{itemize}

  \begin{center}
    \begin{figure}
      \includegraphics[width=5cm]{images/kokkos_machine_model}
      \caption{Conceptual model of a future HPC node. (Kokkos User's Guide).}
      \end{figure}
  \end{center}

\end{frame}


%%%%%%%%%%%%%%%%%%%%%%%%%%%%%%%%%%%%%%%%%%%%%%%%%%%%%%%%%%%%%%%%%%% 
%%%%%%%%%%%%%%%%%%%%%%%%%%%%%%%%%%%%%%%%%%%%%%%%%%%%%%%%%%%%%%%%%%% 
%%%%%%%%%%%%%%%%%%%%%%%%%%%%%%%%%%%%%%%%%%%%%%%%%%%%%%%%%%%%%%%%%%% 
\begin{frame}[fragile=singleslide]
  \frametitle{Kokkos Concepts (2) - What is a device ?}

  \begin{itemize}
  \item Kokkos defines several {\bf c++ class} for representing a \textcolor{red}{device} in \texttt{core/src}, e.g.
    \begin{itemize}
    \item Kokkos::Cuda, Kokkos::OpenMP, Kokkos::Pthreads, Kokkos::Serial
    \item \textcolor{red}{\bf device = execution space + memory space}
    \end{itemize}
  \item Each \textit{Kokkos device} pre-defines some types
  \item Example \textcolor{red}{\textbf{Kokkos device}} (not required for a user, only Kokkos developper), e.g.\\
    {\tiny
      \begin{minted}{c++}
        class Cuda {
          public:
          // Tag this class as a kokkos execution space
          typedef Cuda                  execution_space ;
          
          #if defined( KOKKOS_USE_CUDA_UVM )
          // This execution space's preferred memory space.
          typedef CudaUVMSpace          memory_space ;
          #else
          // This execution space's preferred memory space.
          typedef CudaSpace             memory_space ;
          #endif
          
          // This execution space preferred device_type
          typedef Kokkos::Device<execution_space,memory_space> device_type;
          
          // The size_type best suited for this execution space.
          typedef memory_space::size_type  size_type ;
          
          // This execution space's preferred array layout.
          typedef LayoutLeft            array_layout ;
          ...
        } // end class Cuda
        \end{minted}
      }
  \end{itemize}

\end{frame}

%%%%%%%%%%%%%%%%%%%%%%%%%%%%%%%%%%%%%%%%%%%%%%%%%%%%%%%%%%%%%%%%%%% 
%%%%%%%%%%%%%%%%%%%%%%%%%%%%%%%%%%%%%%%%%%%%%%%%%%%%%%%%%%%%%%%%%%% 
%%%%%%%%%%%%%%%%%%%%%%%%%%%%%%%%%%%%%%%%%%%%%%%%%%%%%%%%%%%%%%%%%%% 
\begin{frame}
  \frametitle{Kokkos Concepts (3) - execution space, memory space}

  \begin{itemize}
  \item \textcolor{blue}{\textbf{Execution space:}} Where should a parallel contruct (\texttt{parallel\_for}, \texttt{parallel\_reduce}, ...) be executed\\
    \begin{itemize}
    \item Special case: \texttt{class HostSpace}, special device (always defined) where execution space is either (Serial, Pthread or OpenMP).
    \item Each execution space is equipped with a \texttt{fence}: \texttt{Kokkos::Cuda::fence()}
    \end{itemize}
  \item \textcolor{blue}{\textbf{Memory space:}} Where / how data are allocated in memory (HostSpace, CudaSpace, CudaUVMSpace, CudaHostPinnedSpace, HBWSpace, ...)
  \item \textcolor{blue}{\textbf{Memory layout}} (we will come back later on that)
  \item Other concepts:
    \begin{itemize}
    \item Execution policy: used to modify a parallel thread dispatch
    \end{itemize}
  \item \textcolor{red}{Multiple execution / memory space} can be used in a single application\\
    See for example in Kokkos sources \texttt{example/tutorial/Advanced\_View/07\_Overlapping\_DeepCopy}\\
    Though, take care that currently, Cuda stream are not completely mapped into Kokkos API~\footnote{To be implemented.}; meanwhile Cuda streams can be used directly (but looses portability); 
  \end{itemize}

\end{frame}



\subsection{Build Kokkos}
%%%%%%%%%%%%%%%%%%%%%%%%%%%%%%%%%%%%%%%%%%%%%%%%%%%%%%%%%%%%%%%%%%%%%%%% 
%%%%%%%%%%%%%%%%%%%%%%%%%%%%%%%%%%%%%%%%%%%%%%%%%%%%%%%%%%%%%%%%%%%%%%%% 
\begin{frame}
  \frametitle{Build kokkos}

  \begin{itemize}
  \item \textbf{Get sources, development branch}
    \begin{itemize}
    \item \texttt{git clone https://github.com/kokkos/kokkos}
    \item \texttt{git checkout develop}
    \item \textcolor{blue}{Practicals on \texttt{ouessant}:}\\
      \texttt{mkdir \$HOME/kokkos-tutorial}\\
      perform \texttt{git clone} operation from \texttt{\$HOME/kokkos-tutorial}\\
      some tutorial examples are already configured for using that precise location.
    \end{itemize}
  \item \textbf{Build configuration}
    \begin{itemize}
    \item \texttt{cd \$KOKKOS\_SOURCES; mkdir build; cd build}
    \item About build system, several ways to use Kokkos
      \begin{enumerate}
      \item Only when Kokkos is build inside \myhref{https://github.com/trilinos/Trilinos}{Trilinos}, \myhref{https://cmake.org/}{CMake} is used
      \item Standalone utilization: use \texttt{generate\_makefile.bash}, then \texttt{make kokkoslib; make install}
      \item Embedded Kokkos source files in your application.
      \end{enumerate}
      % \item We will use 2. and 3.
    \end{itemize}
  \end{itemize}
 
\end{frame}

%%%%%%%%%%%%%%%%%%%%%%%%%%%%%%%%%%%%%%%%%%%%%%%%%%%%%%%%%%%%%%%%%%%%%%%% 
%%%%%%%%%%%%%%%%%%%%%%%%%%%%%%%%%%%%%%%%%%%%%%%%%%%%%%%%%%%%%%%%%%%%%%%% 
\begin{frame}
  \frametitle{Build kokkos (2)}

  \begin{itemize}
  \item Example build configurations
    \begin{itemize}
    \item Serial (mostly for testing)\\
      \texttt{../generate\_makefile.bash --with-serial --prefix=\$HOME/local/kokkos\_serial}
    \item \textbf{OpenMP}\\
      \texttt{../generate\_makefile.bash --with-openmp --prefix=\$HOME/local/kokkos\_openmp\_dev}
    \item \textbf{CUDA (+ OpenMP)}\\
      \texttt{../generate\_makefile.bash --with-cuda --arch=Pascal60 --prefix=\$HOME/local/kokkos\_cuda\_lambda\_openmp --with-cuda-options=enable\_lambda --with-openmp --with-hwloc=/usr}
    \end{itemize}
  \end{itemize}
  
\end{frame}

% device query
%%%%%%%%%%%%%%%%%%%%%%%%%%%%%%%%%%%%%%%%%%%%%%%%%%%%%%%%%%%%%%%%%%%%%%%%% 
%%%%%%%%%%%%%%%%%%%%%%%%%%%%%%%%%%%%%%%%%%%%%%%%%%%%%%%%%%%%%%%%%%%%%%%% 
\begin{frame}
  \frametitle{Hands-On 1a : Cuda device\_query - job submission}

  {\large\textcolor{red}{\textbf{Purpose:} just make sure you are able to launch a job on Ouessant}}

  \begin{itemize}
  \item We will use a cuda sample
  \item In your home on \texttt{ouessant}: 
    \begin{itemize}
    \item \texttt{cp -a /usr/local/cuda-9.0/samples .}
    \item \texttt{cd samples/1\_Utilities/deviceQuery}
    \item \texttt{module load at/10.0 cuda/9.0}
    \item \texttt{make}
    \item You have an executable named \texttt{deviceQuery}
    \item You can run it on a \textcolor{red}{\bf Ouessant login node:} \fbox{\texttt{./deviceQuery}}
    \item You can run it on a \textcolor{darkgreen}{\bf Ouessant compute node} using the script \texttt{submit\_ouessant.sh} launched like this:\\
      \fbox{\texttt{bsub < submit\_ouessant.sh}}\\
      The submission script is located in the training archive (\texttt{code/handson/1a/submit\_ouessant.sh})
    \item What differences can you see between the two executions ?
    \end{itemize}
  \end{itemize}

\end{frame}

%%%%%%%%%%%%%%%%%%%%%%%%%%%%%%%%%%%%%%%%%%%%%%%%%%%%%%%%%%%%%%%%%%%%%%%% 
%%%%%%%%%%%%%%%%%%%%%%%%%%%%%%%%%%%%%%%%%%%%%%%%%%%%%%%%%%%%%%%%%%%%%%%% 
\begin{frame}
  \frametitle{Hands-On 1b : Kokkos query\_device with hwloc}

  \hypertarget{handson1}{}
  {\large\textcolor{red}{\textbf{Purpose:} just cross-checking Kokkos/Hwloc is working OK}}

  \begin{itemize}
  \item We will first re-use material from Kokkos github repository.
  \item On your home, on \texttt{ouessant}: 
    \begin{enumerate}
    \item \texttt{mkdir kokkos-tutorial; cd kokkos-tutorial}
    \item \texttt{git clone https://github.com/kokkos/kokkos.git} \\
      \# \textbf{Don't try to build kokkos here (for now)}
    %\item \texttt{git clone https://github.com/kokkos/kokkos-tutorials.git}
    %\item \texttt{cd kokkos-tutorials/Intro-Full/SNL2015/Exercises/}\\
    %  \# 1 Day tutorial exercice are routed to \textbf{build kokkos for you}
    \end{enumerate}
  \end{itemize}

\end{frame}

%%%%%%%%%%%%%%%%%%%%%%%%%%%%%%%%%%%%%%%%%%%%%%%%%%%%%%%%%%%%%%%%%%%%%%%% 
%%%%%%%%%%%%%%%%%%%%%%%%%%%%%%%%%%%%%%%%%%%%%%%%%%%%%%%%%%%%%%%%%%%%%%%% 
\begin{frame}[fragile=singleslide]
  \frametitle{Hands-On 1b : Kokkos query\_device with hwloc}

  {\large\textcolor{red}{\textbf{Purpose:}}}
  \begin{itemize}
  \item \textcolor{red}{just cross-checking Kokkos/Hwloc is working OK}
  \item \textcolor{red}{On login nodes only for now}
  \end{itemize}
    
  {\bf TO DO:}
  \begin{itemize}
  \item Kokkos sources will be built by the application Makefile
  \item \texttt{cd \$HOME/kokkos-tutorial/kokkos/example/query\_device}
  \item open \texttt{query\_device.cpp} to have a look; no computations, it just prints hardware information
  \item \textbf{Take some time to have a look at the Makefile.}\\
    Note that latter when using an installed kokkos library, we won't need to set architecture or device related variables on the command line .
  \end{itemize}

\end{frame}

%%%%%%%%%%%%%%%%%%%%%%%%%%%%%%%%%%%%%%%%%%%%%%%%%%%%%%%%%%%%%%%%%%%%%%%% 
%%%%%%%%%%%%%%%%%%%%%%%%%%%%%%%%%%%%%%%%%%%%%%%%%%%%%%%%%%%%%%%%%%%%%%%% 
\begin{frame}[fragile=singleslide]
  \frametitle{Hands-On 1b : Kokkos query\_device with hwloc}

  \begin{enumerate}
  \item \textbf{Default serial build (with hwloc):} \texttt{make KOKKOS\_USE\_TPLS="hwloc"}\\
    How many NUMA / Cores / Hyperthreads on power8 CPU ?\\
    What is the current SMT mode on a \texttt{ouessant} login node ?
    \begin{itemize}
    \item \texttt{ppc64\_cpu \--\--smt}
    \item \texttt{ppc64\_cpu --info} 
    \end{itemize}
  \item environment: \texttt{module load at/10.0 cuda/9.0}
  \item \textbf{OpenMP build (with hwloc):}\\
    \fbox{\texttt{make KOKKOS\_USE\_TPLS="hwloc" KOKKOS\_DEVICES=OpenMP}} (off course, exact same information obtained as with serial execution)
  \item \textbf{CUDA/OpenMP build (with hwloc):}\\
    \fbox{\texttt{make KOKKOS\_USE\_TPLS="hwloc" KOKKOS\_DEVICES=Cuda,OpenMP}} rerun and you should get information about the CPU+GPU configuration
  \end{enumerate}
  
\end{frame}

  
%%%%%%%%%%%%%%%%%%%%%%%%%%%%%%%%%%%%%%%%%%%%%%%%%%%%%%%%%%%%%%%%%%%%%%%% 
%%%%%%%%%%%%%%%%%%%%%%%%%%%%%%%%%%%%%%%%%%%%%%%%%%%%%%%%%%%%%%%%%%%%%%%% 
\begin{frame}[fragile=singleslide]
  \frametitle{Hands-On 1b : Kokkos query\_device without hwloc}

  {\large\textcolor{orange}{\bf Question: What happens if hwloc is not activated ?}}

  \begin{itemize}
  \item Edit file \texttt{query\_device.cpp} and do the following modification:
    \begin{enumerate}
    \item Add \texttt{Kokkos::initialize(argc, argv);} after \texttt{MPI\_Init}
    \item Add \texttt{Kokkos::finalize();} before \texttt{MPI\_Finalize}
    \item Rebuild and run \texttt{./query\_device.host \--\--help}
    \item run \texttt{./query\_device.host \--\--kokkos-threads=12} (alternatively, you can use regular OpenMP environment variables)
    \item change\\
      {\small
        \begin{minted}{c++}
          #if defined( KOKKOS_ENABLE_CUDA )
          Kokkos::Cuda::print_configuration( msg );
          #else
          Kokkos::OpenMP::print_configuration( msg );
          #endif
          //Kokkos::print_configuration( msg ); // how this app was build
        \end{minted}
      }
    \end{enumerate}
  \item {\small Rebuild 1 \textcolor{red}{without HWLOC:}\\
      \texttt{make KOKKOS\_DEVICES=OpenMP}}
    % {\small
    %   \begin{minted}{bash}
    %     Kokkos::OpenMP thread_pool_topology[ 1 x 80 x 1 ]
    %   \end{minted}
    % }
  \item {\small Rebuild 2 \textcolor{darkgreen}{with HWLOC:}\\
      \texttt{make KOKKOS\_DEVICES=OpenMP KOKKOS\_USE\_TPLS="hwloc"}}
    % {\small 
    %   \begin{minted}{bash}
    %     hwloc( NUMA[2] x CORE[10] x HT[4] )
    %     Kokkos::OpenMP hwloc[2x10x4] hwloc_binding_enabled thread_pool_topology[ 2 x 10 x 4 ]      
    %   \end{minted}
    % }
  %\item add \texttt{Kokkos::print\_configuration} to cross-check at run-time executable was built with the right options
  \item processor affinity is important to performance; you can/must configure OpenMP environment.
  \end{itemize}

\end{frame}


\section{Kokkos - data containers and threads dispatch}
%%%%%%%%%%%%%%%%%%%%%%%%%%%%%%%%%%%%%%%%%%%%%%%%%%%%%%%%%%%%%%%%%%%%%%%% 
%%%%%%%%%%%%%%%%%%%%%%%%%%%%%%%%%%%%%%%%%%%%%%%%%%%%%%%%%%%%%%%%%%%%%%%% 
\begin{frame}[fragile=singleslide]
  \frametitle{Kokkos data Container}

  \begin{itemize}
  \item \textcolor{blue}{\texttt{Kokkos::View<...>}} replacement for \textcolor{red}{\texttt{std::vector}} with \textbf{multidimensionnal feature} and \textbf{hardware adapted memory layout}\\
    \begin{itemize}
    \item \texttt{Kokkos::View<double **> data("data",NX,NY);} : 2D array with sizes known at runtime
    \item \texttt{Kokkos::View<double *[3]> data("data",NX);} : 2D array with first size known at runtime ($NX$), and second known at compile time (3).
    \item How do I access data ? $data(i,j)$ !
    \item Need to talk about \textbf{memory layout:}\\
      \begin{itemize}
      \item \textbf{LayoutLeft}: $data(i,j,k)$ uses linearized index as $i + NX*j + NX*NY * k$ (column-major order)
      \item \textbf{LayoutRight}: $data(i,j,k)$ uses linearized index as $k + NZ*j + NZ*NY * i$ (raw-major order)
      \item You can if you like, still enforce memory layout yourself (just use 1D Views, and compute index yourself);\\
        We will see the 2 possibilities with the miniApp on the Fisher equation
      \end{itemize}
    \item \textcolor{orange}{Which memory space ?} By default, the default device memory space !\\
      Want to enforce in which memory space lives the view ? 
      \textcolor{blue}{\texttt{Kokkos::View<..., Device>}}: if a second template parameter is given, Kokkos expects a \texttt{Device} (e.g. \texttt{Kokkos::OpenMP}, \texttt{Kokkos::Cuda}, ...)
    \end{itemize}
  \end{itemize}

\end{frame}

%%%%%%%%%%%%%%%%%%%%%%%%%%%%%%%%%%%%%%%%%%%%%%%%%%%%%%%%%%%%%%%%%%%%%%%% 
%%%%%%%%%%%%%%%%%%%%%%%%%%%%%%%%%%%%%%%%%%%%%%%%%%%%%%%%%%%%%%%%%%%%%%%% 
\begin{frame}[fragile=singleslide]
  \frametitle{Kokkos data Container (2)}

  \begin{itemize}
  \item \textcolor{blue}{\texttt{Kokkos::View<...>}} are reference-counted
  \item By default do a \textcolor{orange}{\textbf{shallow copy}}
    \begin{minted}{c++}
      Kokkos::View<int *>("a",10);
      Kokkos::View<int *>("b",10);
      a = b; // a now points to b (ref counter incremented by 1)
    \end{minted}
  \item \textcolor{orange}{\textbf{Deep copy}} must by explicit:
    \begin{minted}{c++}
      Kokkos::deep_copy(dest,src);
    \end{minted}
    \begin{itemize}
    \item \textbf{Usefull when copying data from one memory space to another}\\
      e.g. \textcolor{red}{from HostSpace to CudaSpace}
    \item When \texttt{dest} and \texttt{src} are in the same memory space, it does nothing ! (usefull for portability, see example in miniapps later)
    \end{itemize}
  \end{itemize}

\end{frame}
%%%%%%%%%%%%%%%%%%%%%%%%%%%%%%%%%%%%%%%%%%%%%%%%%%%%%%%%%%%%%%%%%%%%%%%% 
%%%%%%%%%%%%%%%%%%%%%%%%%%%%%%%%%%%%%%%%%%%%%%%%%%%%%%%%%%%%%%%%%%%%%%%% 
\begin{frame}[fragile=singleslide]
  \frametitle{Kokkos data Container (3)}

  \begin{itemize}
  \item \textcolor{orange}{A verbose Kokkos::View declaration} example:
    \begin{minted}{c++}
      Kokkos::View<double*,Kokkos::LayoutLeft,Kokkos::CudaSpace> a;
    \end{minted}
    \begin{itemize}
    \item \textcolor{orange}{\textbf{What ?}} a data type
    \item \textcolor{orange}{\textbf{How ?}} a memory layout
    \item \textcolor{orange}{\textbf{Where ?}} a memory space
    \item the last two template parameters are optionnal (have default values)
    \end{itemize}
  \item \textcolor{blue}{\texttt{Kokkos::DualView<...>}} : usefull when porting an application incrementally, adata container on two different memory space.\\
    see \texttt{tutorial/Advanced\_Views/04\_dualviews/dual\_view.cpp}
  \item \textcolor{blue}{\texttt{Kokkos::UnorderedMap<...>}}
  \item Can also define \textbf{subview (array slicing, no deep copy)}. See exercice about Mandelbrot set.
  \end{itemize}
  
\end{frame}

%%%%%%%%%%%%%%%%%%%%%%%%%%%%%%%%%%%%%%%%%%%%%%%%%%%%%%%%%%%%%%%%%%%%%%%% 
%%%%%%%%%%%%%%%%%%%%%%%%%%%%%%%%%%%%%%%%%%%%%%%%%%%%%%%%%%%%%%%%%%%%%%%% 
\begin{frame}[fragile=singleslide]
  \frametitle{Kokkos data Container (4)}

  \begin{itemize}
  \item \textcolor{red}{\textbf{What types of data may a View contain ?}}\\
    C++ \myhref{http://en.cppreference.com/w/cpp/concept/PODType}{Plain Old Data} (POD), i.e. basically compatible with C language:
    \begin{itemize}
    \item Can be allocated with \texttt{std::malloc}
    \item Can be copied with \texttt{std::memmove}
    \end{itemize}
  \item POD in C++11: 
    \begin{itemize}
    \item a trivial type (no virtual member functions, no virtual base class)
    \item a standard layout type
    \end{itemize}
  \item C++11: How to check if a given class \texttt{A} is POD ?
  \end{itemize}
  \begin{minted}{c++}
    #include <type_traits>
    
    class A { ... }
    std::cout << "is class A POD ? " << std::is_pod<A>::value << "\n";
  \end{minted}
  
\end{frame}

%%%%%%%%%%%%%%%%%%%%%%%%%%%%%%%%%%%%%%%%%%%%%%%%%%%%%%%%%%%%%%%%%%%%%%%%% 
%%%%%%%%%%%%%%%%%%%%%%%%%%%%%%%%%%%%%%%%%%%%%%%%%%%%%%%%%%%%%%%%%%%%%%%% 
\begin{frame}[fragile]
  \frametitle{C++ functors and Lambda}

  \textcolor{violet}{\bf What is a functor class ?}\\
  Functor $=$ Function object, can be called like a function.

  \begin{itemize}
    \begin{onlyenv}<1>
    \item a simple computation
      \begin{minted}[autogobble=true]{c++}
        void do_a_for_loop(std::vector<double>& data) {
          for (int i=0; i<data.size; ++i) {
            data[i] += 12;
          }
        }
      \end{minted}
    \end{onlyenv}
    \begin{onlyenv}<2>
    \item same with a function pointer
      \begin{minted}[autogobble=true]{c++}
        void doSomething(double &value) {
          value += 12;
        }
        
        // use a function pointeur
        void do_a_for_loop(std::vector<double>& data, void f(double&)) {
          
          for (int i=0; i<data.size; ++i) {
            f(data[i]);
          }
        }
    \end{minted}
  \end{onlyenv}
  \begin{onlyenv}<3>
    \item same with a \textcolor{blue}{\bf function object (functor)}
      {\small
      \begin{minted}[autogobble=true]{c++}
        class DoSomething {
          // a functor can have parameters, members, execution context, ...
          // can be copied, passed to function, to threads, ...
          DomeSometing(double param) : param(param) {}

          void operator() (double &value) {
            value += param;
          }
          private:
          double param;
        }
        
        // use a function pointeur
        void do_a_for_loop(std::vector<double>& data, DoSomething f) {
          for (int i=0; i<data.size; ++i) {
            f(data[i]);
          }
        }
    \end{minted}
  }
  \end{onlyenv}
  \begin{onlyenv}<4>
  \item same with a \textcolor{red}{\bf lambda} : lambda $=$ shorhand for a functor, context is {captured} from the surrounding code.
      {\small
      \begin{minted}[autogobble=true]{c++}
        // use a function pointeur
        template<class ALambda>
        void do_a_for_loop(std::vector<double>& data, ALambda f) {
          for (int i=0; i<data.size; ++i) {
            f(data[i]);
          }
        }

        double param = 12;
        auto domesometing = [=](double& value) {value += param; };
        
        // do some computation
        do_a_for_loop(data, dosomething);

    \end{minted}
  }
  \end{onlyenv}
\end{itemize}

\end{frame}


% saxpy
%%%%%%%%%%%%%%%%%%%%%%%%%%%%%%%%%%%%%%%%%%%%%%%%%%%%%%%%%%%%%%%%%%%%%%%%% 
%%%%%%%%%%%%%%%%%%%%%%%%%%%%%%%%%%%%%%%%%%%%%%%%%%%%%%%%%%%%%%%%%%%%%%%% 
\begin{frame}[fragile=singleslide]
  \frametitle{Hands-On 2 : SAXPY}

  \hypertarget{handson2}{}
  {\large \textcolor{red}{\bf Purpose:} The simplest computing kernel in Kokkos, importance of hwloc}

  % penser à refactorer le timer, pas du tout adapté aux mesures de perf GPU.
  
  \begin{itemize}
  \item There 5 differents versions
  \item \textbf{1. Serial : no Kokkos)}
  \item \textbf{2. OpenMP : no Kokkos)}
  \item 3. Kokkos-Lambda-CPU : Kokkos with lambda for threads dispatch
  \item \textbf{4. Kokkos-Lambda : Kokkos with lambda for threads dispatch and data buffer (Kokkos::View)}
  \item 5. Kokkos-Functor-CPU : Kokkos with functor for threads dispatch only
  \end{itemize}

  {\large \textcolor{red}{\bf Proposed activity (get the sources):}}
  {\small
    \begin{enumerate}
    \item First, make sure you cloned kokkos sources inside \texttt{\${HOME}/kokkos-tutorial}:\\
      \textcolor{magenta}{\texttt{mkdir -p \${HOME}/kokkos-tutorial; cd \${HOME}/kokkos-tutorial}}\\
      \textcolor{magenta}{\texttt{git clone  https://github.com/kokkos/kokkos.git}}
    %\item Download \texttt{kokkos-tutorials} sources (anywhere in \texttt{\${HOME}}):\\
     % \textcolor{magenta}{\texttt{git clone https://github.com/kokkos/kokkos-tutorials.git}}
    %\item \textcolor{magenta}{\texttt{cd kokkos-tutorials/Intro-Full/SNL2015/Exercises/01\_AXPY}}
    \item From the provided material \textcolor{magenta}{\texttt{cd patc\_kokkos/code/handson/2/saxpy}}
    \end{enumerate}
  }
  
\end{frame}

%%%%%%%%%%%%%%%%%%%%%%%%%%%%%%%%%%%%%%%%%%%%%%%%%%%%%%%%%%%%%%%%%%%%%%%% 
%%%%%%%%%%%%%%%%%%%%%%%%%%%%%%%%%%%%%%%%%%%%%%%%%%%%%%%%%%%%%%%%%%%%%%%% 
\begin{frame}[fragile=singleslide]
  \frametitle{Hands-On 2 : SAXPY}

  {\large \textcolor{red}{\bf Proposed activity:}}
  \begin{itemize}
  \item \textcolor{red}{\textbf{Saxpy serial (reference executable on Power8)}}
    \begin{itemize}
    \item \texttt{cd handson/2/saxpy/Serial}
    \item \texttt{make KOKKOS\_ARCH=Power8}
    \item Alternatively, we could have modified \texttt{Makefile} and changed \texttt{SNB} (Sandy Bridge) into \texttt{Power8}
    \end{itemize}
  \item \textcolor{orange}{\textbf{Saxpy regular OpenMP (on Power8)}}
    \begin{itemize}
    \item \texttt{cd handson/2/saxpy/OpenMP}
    \item Rebuild: \texttt{make KOKKOS\_ARCH=Power8}; and observe performance by change vector size
    \end{itemize}
  \end{itemize}

  \textbf{see also slides from SC2016, page 42(74).}
  
\end{frame}

%%%%%%%%%%%%%%%%%%%%%%%%%%%%%%%%%%%%%%%%%%%%%%%%%%%%%%%%%%%%%%%%%%%%%%%% 
%%%%%%%%%%%%%%%%%%%%%%%%%%%%%%%%%%%%%%%%%%%%%%%%%%%%%%%%%%%%%%%%%%%%%%%% 
\begin{frame}[fragile=singleslide]
  \frametitle{Hands-On 2 : SAXPY}

  \begin{itemize}
  \item \textcolor{violet}{\textbf{Saxpy Kokkos OpenMP (on Power8)}}~\footnote{Make sure to use a very large data array; Power8 has very large cache memory. If you don't, this example will not measure memory bandwith. Maximum bandwidth is 230 GB/s on a 2 socket P8. You should measure around 170 GB/s.}
    \begin{itemize}
    \item \texttt{cd handson/2/saxpy/Kokkos-Lambda}
    \item Add the following lines in \texttt{saxpy.cpp} right after Kokkos initialization
      {\scriptsize
      \begin{minted}[autogobble=true]{c++}
        std::ostringstream msg;
        if ( Kokkos::hwloc::available() ) {
          msg << "hwloc( NUMA[" << Kokkos::hwloc::get_available_numa_count()
          << "] x CORE["    << Kokkos::hwloc::get_available_cores_per_numa()
          << "] x HT["      << Kokkos::hwloc::get_available_threads_per_core()
          << "] )"
          << std::endl ;
        }
        
        Kokkos::print_configuration( msg );
        std::cout << msg.str();
      \end{minted}
    }
    \item \texttt{make KOKKOS\_ARCH=Power8}
    \item Make sure all available CPU cores were used ($1\times 160 \times 1$)
    \item Change the number of OpenMP threads created by kokkos, e.g. :\\
      \texttt{./saxpy.host  --threads=20}
    \item Add again \texttt{KOKKOS\_USE\_TPLS="hwloc"} on the command line\\
      Rebuild and rerun, you should see that application uses \textbf{all the available numa domains}, and a strongly increased bandwidth usage !
    \end{itemize}
  \end{itemize}

\end{frame}

%%%%%%%%%%%%%%%%%%%%%%%%%%%%%%%%%%%%%%%%%%%%%%%%%%%%%%%%%%%%%%%%%%%%%%%% 
%%%%%%%%%%%%%%%%%%%%%%%%%%%%%%%%%%%%%%%%%%%%%%%%%%%%%%%%%%%%%%%%%%%%%%%% 
\begin{frame}[fragile=singleslide]
  \frametitle{Hands-On 2 : SAXPY}

  \begin{itemize}
  \item \textcolor{darkgreen}{\textbf{Saxpy CUDA (on Power8 + Nvidia K80/P100)}}
    \begin{itemize}
    \item \texttt{cd handson/2/saxpy/Kokkos-Lambda}
    \item \texttt{module load at/10.0 cuda/9.2}
    \end{itemize}
  \item Rebuild for K80, run on ouessant (front node):\\
    \texttt{make KOKKOS\_DEVICES="Cuda,OpenMP" KOKKOS\_ARCH="Kepler37,Power8" KOKKOS\_USE\_TPLS="hwloc"}
  \item Rebuild for P100, run on compute node using \texttt{submit\_ouessant.sh} (should see a strong difference):\\
    \texttt{make KOKKOS\_DEVICES="Cuda,OpenMP" KOKKOS\_ARCH="Pascal60,Power8" KOKKOS\_USE\_TPLS="hwloc"}\\
    Please note that \textbf{maximun bandwith is 732 GB/s for Pascal P100}, you can retrieve this number by examining \texttt{deviceQuery} example in CUDA/SDK.
  \end{itemize}
\end{frame}



%\section{Examples}

% hands-on 3a : kokkos - cmake integration

\section*{Additionnal Kokkos material}
%%%%%%%%%%%%%%%%%%%%%%%%%%%%%%%%%%%%%%%%%%%%%%%%%%%%%%%%%%%%%%%%%%%%%%%% 
%%%%%%%%%%%%%%%%%%%%%%%%%%%%%%%%%%%%%%%%%%%%%%%%%%%%%%%%%%%%%%%%%%%%%%%% 
\begin{frame}[fragile=singleslide]
  \frametitle{Kokkos - cmake integration (1)}

  \begin{itemize}
  \item Why Cmake ?
    \begin{itemize}
    \item cmake is supported by kokkos
    \item easy to integrate and configure (versus e.g. old autotools, versus regular Makefile): need to handle the architecture flags combinatorics
    \end{itemize}
  \item User application top-level cmake can be as small as 7 lines
    {\small
      \begin{minted}{cmake}
        cmake_minimum_required(VERSION 3.1)
        project(myproject CXX)
        
        # C++11 is for Kokkos
        set(CMAKE_CXX_STANDARD 11)
        set(CMAKE_CXX_EXTENSIONS OFF)
        
        # first buid kokkos (
        add_subdirectory(external/kokkos)
        
        # pass Kokkos include directories to our target application
        include_directories(${Kokkos_INCLUDE_DIRS_RET})
        
        # build the user sources
        add_subdirectory(src)
      \end{minted}
      }
  \end{itemize}
  
\end{frame}

%%%%%%%%%%%%%%%%%%%%%%%%%%%%%%%%%%%%%%%%%%%%%%%%%%%%%%%%%%%%%%%%%%%%%%%% 
%%%%%%%%%%%%%%%%%%%%%%%%%%%%%%%%%%%%%%%%%%%%%%%%%%%%%%%%%%%%%%%%%%%%%%%% 
\begin{frame}[fragile=singleslide]
  \frametitle{Kokkos - cmake integration (2)}

  {\large List of important kokkos-related {\bf cmake variables}}
  \begin{itemize}
  \item \textcolor{magenta}{\texttt{KOKKOS\_ENABLE\_OPENMP}}, \textcolor{magenta}{\texttt{KOKKOS\_ENABLE\_CUDA}},... $\Rightarrow$ which execution space are enabled (multiple possible)
  \item \textcolor{magenta}{\texttt{KOKKOS\_ARCH}} (bold values are relevant for \texttt{ouessant}), will trigger relevant arch flags
    \begin{tabular}{ll}
      \# Intel:    & KNC,KNL,SNB,HSW,BDW,SKX\\
    \# NVIDIA:   & Kepler,Kepler30,Kepler32,Kepler35,\textcolor{red}{\bf Kepler37},Maxwell,\\
      & Maxwell50,Maxwell52,Maxwell53,\textcolor{red}{\bf Pascal60},Pascal61,\\
      & Volta70,Volta72\\
    \# ARM:      &ARMv80,ARMv81,ARMv8-ThunderX,ARMv8-TX2\\
    \# IBM:      &BGQ,Power7,\textcolor{red}{\bf Power8},Power9\\
    \# AMD-GPUS: &Kaveri,Carrizo,Fiji,Vega\\
    \# AMD-CPUS: &AMDAVX,Ryzen,Epyc\\
    \end{tabular}
    % # possible values:
    % # Intel:    KNC,KNL,SNB,HSW,BDW,SKX
    % # NVIDIA:   Kepler,Kepler30,Kepler32,Kepler35,Kepler37,Maxwell,
    %             Maxwell50,Maxwell52,Maxwell53,Pascal60,Pascal61,
    %             Volta70,Volta72
    % # ARM:      ARMv80,ARMv81,ARMv8-ThunderX,ARMv8-TX2
    % # IBM:      BGQ,Power7,Power8,Power9
    % # AMD-GPUS: Kaveri,Carrizo,Fiji,Vega
    % # AMD-CPUS: AMDAVX,Ryzen,Epyc
  \end{itemize}

\end{frame}

%%%%%%%%%%%%%%%%%%%%%%%%%%%%%%%%%%%%%%%%%%%%%%%%%%%%%%%%%%%%%%%%%%%%%%%% 
%%%%%%%%%%%%%%%%%%%%%%%%%%%%%%%%%%%%%%%%%%%%%%%%%%%%%%%%%%%%%%%%%%%%%%%% 
\begin{frame}[fragile=singleslide]
  \frametitle{Kokkos - cmake integration (3)}

  \begin{itemize}
  \item curse gui interface: \texttt{ccmake}
    \begin{center}
      \includegraphics[width=7cm]{images/ccmake_kokkos}
    \end{center}
  \item command line interface : \texttt{cmake}
    \texttt{mkdir build\_openmp; cd build\_openmp; ccmake -DKOKKOS\_ENABLE\_OEPNMP ..}
  \item How to build ? for OpenMP / CUDA ?
  \end{itemize}

\end{frame}
  
%%%%%%%%%%%%%%%%%%%%%%%%%%%%%%%%%%%%%%%%%%%%%%%%%%%%%%%%%%%%%%%%%%%%%%%%
%%%%%%%%%%%%%%%%%%%%%%%%%%%%%%%%%%%%%%%%%%%%%%%%%%%%%%%%%%%%%%%%%%%%%%%% 
\begin{frame}[fragile=singleslide]
  \frametitle{Hands-On 3a}

  Use the template project

\end{frame}


%%%%%%%%%%%%%%%%%%%%%%%%%%%%%%%%%%%%%%%%%%%%%%%%%%%%%%%%%%%%%%%%%%%%%%%% 
%%%%%%%%%%%%%%%%%%%%%%%%%%%%%%%%%%%%%%%%%%%%%%%%%%%%%%%%%%%%%%%%%%%%%%%% 
\begin{frame}[fragile=singleslide]
  \frametitle{Kokkos compute Kernels}

  \begin{itemize}
  \item How to specify a compute kernel in Kokkos ?
    \begin{enumerate}
    \item \textcolor{blue}{\textbf{Use Lambda functions.}}\\
      NB: a lambda in c++11 is an unnamed function object capable of capturing variables in scope.
      \begin{minted}{c++}
        Kokkos::parallel_for (100, KOKKOS_LAMBDA (const int i) {
          data(i) = 2*i;
        });
      \end{minted}
      Here we do 2 things in 1 step: define the computation body (lambda func) and launch computation.
    \item \textcolor{darkgreen}{\textbf{Use a C++ functor class.}}\\
      A functor is a class containing a function to execute in parallel.
      \begin{minted}{c++}
        class FunctorType {
          public:
          KOKKOS_INLINE_FUNCTION
          void operator() ( const int i ) const ;
        };
        ...
        FunctorType func;
        Kokkos::parallel_for (100, func);
      \end{minted}
    \end{enumerate}
  \end{itemize}

\end{frame}


%%%%%%%%%%%%%%%%%%%%%%%%%%%%%%%%%%%%%%%%%%%%%%%%%%%%%%%%%%%%%%%%%%%%%%%% 
%%%%%%%%%%%%%%%%%%%%%%%%%%%%%%%%%%%%%%%%%%%%%%%%%%%%%%%%%%%%%%%%%%%%%%%% 
\begin{frame}[fragile=singleslide]
  \frametitle{Kokkos compute Kernels}

  \textbf{Lambda or Functor: which one to use in Kokkos ? Both !}
  \begin{enumerate}
  \item \textcolor{blue}{\textbf{Use Lambda functions.}}\\
    \begin{itemize}
    \item easy way for small compute kernels
    \item For GPU, requires Cuda 7.5 (8.0 is current and latest CUDA version)
    \end{itemize}
  \item \textcolor{darkgreen}{\textbf{Use a C++ functor class.}}\\
    \begin{itemize}
    \item More flexible, allow to design more complex kernel
    \end{itemize}
  \end{enumerate}
\end{frame}


%\subsection{Use an installed version of Kokkos}
%%%%%%%%%%%%%%%%%%%%%%%%%%%%%%%%%%%%%%%%%%%%%%%%%%%%%%%%%%%%%%%%%%%%%%%%% 
%%%%%%%%%%%%%%%%%%%%%%%%%%%%%%%%%%%%%%%%%%%%%%%%%%%%%%%%%%%%%%%%%%%%%%%% 
\begin{frame}[fragile=singleslide]
  \frametitle{Using an installed Kokkos}

  \begin{itemize}
  \item As you will surely \textcolor{red}{\textbf{use multiple versions}} of Kokkos (OpenMP, Cuda, ...), with/without Lambda, UVM, different compilers, etc ... it will be very usefull to use some \textbf{modulefiles}.
  \item A \textcolor{blue}{\textbf{module environment}} is not a tool specific to a super-computer, it can be used on a \textbf{Desktop/Laptop} to configure an execution environment.\\
    e.g. \textcolor{blue}{\texttt{sudo apt-get install environment-modules}} (Debian/Ubuntu)
  \item \textcolor{blue}{What is a modulefiles ?}
    A simple way to set env variables to ease the use of a given software package.
  \item You will find some examples modulefiles for Kokkos in \texttt{/pwrwork/workshops/patc-201701/kokkos/modulefiles/kokkos} you can easily adapt to your own platform.
  \end{itemize}

\end{frame}

%%%%%%%%%%%%%%%%%%%%%%%%%%%%%%%%%%%%%%%%%%%%%%%%%%%%%%%%%%%%%%%%%%%%%%%% 
%%%%%%%%%%%%%%%%%%%%%%%%%%%%%%%%%%%%%%%%%%%%%%%%%%%%%%%%%%%%%%%%%%%%%%%% 
\begin{frame}[fragile=singleslide]
  \frametitle{Using an installed Kokkos (2)}

  \begin{itemize}
  \item A simple modulefiles for Kokkos should at minimum set variable \textcolor{blue}{\texttt{KOKKOS\_PATH}} pointing to the installed directory (the one which contains \textcolor{blue}{\texttt{Makefile.kokkos}}
  \item \textcolor{blue}{How to use your own modulefiles ?} Just use the following:
    \begin{minted}{bash}
      # Assuming you placed the module file in
      # /somewhere_on_your_machince/modulefiles
      module use /somewhere_on_your_machince/modulefiles
      
      # e.g. load Kokkos for GPU
      module load kokkos/cuda80_gnu485_dev_k80
    \end{minted}
  \end{itemize}

\end{frame}


%\subsection{Use Kokkos from Trilinos}
%%%%%%%%%%%%%%%%%%%%%%%%%%%%%%%%%%%%%%%%%%%%%%%%%%%%%%%%%%%%%%%%%%%%%%%%% 
%%%%%%%%%%%%%%%%%%%%%%%%%%%%%%%%%%%%%%%%%%%%%%%%%%%%%%%%%%%%%%%%%%%%%%%% 
\begin{frame}
  \frametitle{About Kokkos in Trilinos}

  \begin{itemize}
  \item \myhref{https://github.com/kokkos/kokkos}{kokkos} is originally a subpackage of \myhref{https://trilinos.org/}{trilinos} (application framework for solving problems requiring parallel large distributed linear algebra solvers).
  \item \textcolor{red}{Kokkos is the performance portable layer}, to allow running Trilinos as efficiently as possible on multiple architectures.
  \item \textbf{Kokkos can be build independently from Trilinos} and used in other applications
  \end{itemize}

  \begin{center}
    \includegraphics[width=6cm]{images/Kokkos-Multi-CoE_slide2}
  \end{center}
  
\end{frame}

%%%%%%%%%%%%%%%%%%%%%%%%%%%%%%%%%%%%%%%%%%%%%%%%%%%%%%%%%%%%%%%%%%%%%%%% 
%%%%%%%%%%%%%%%%%%%%%%%%%%%%%%%%%%%%%%%%%%%%%%%%%%%%%%%%%%%%%%%%%%%%%%%% 
\begin{frame}
  \frametitle{About Kokkos in Trilinos}

  \begin{itemize}
  \item \textcolor{red}{\textbf{Don't do the following on \texttt{ouessant}, your home is too small}}, just keep the spirit to try on your own machine
  \item \textbf{Build a minimal featured Trilinos with Kokkos for GPU activated} : \textcolor{red}{Tpetra + kokkos + Cuda}
    \begin{enumerate}
    \item \textbf{Example config plateform:} Ubuntu 16.04 + openmpi + cuda 8.0\\
      compiler is gcc 5.4.0
    \item \textbf{Get Trilinos sources:}\\
      \texttt{git clone https://github.com/trilinos/Trilinos.git;} \texttt{cd Trilinos; git checkout develop}
    \item \textbf{CMake configuration script:} Use the provided configuration file \texttt{configure\_tpetra\_kokkos\_cuda\_nvcc\_wrapper.sh} located in the provided archive (\texttt{doc/trilinos})\\
      this script needs slights changes (var \texttt{OMPI\_CXX} and install prefix)\\
      this script must be run in a build directory (not directly in trilinos sources).\\
      this config will build kokkos with unit tests and examples.
    \item \textbf{Build:} \texttt{make -j; make install}
    \item \textbf{Build a sample project.}
    \end{enumerate}
  \end{itemize}

\end{frame}

%%%%%%%%%%%%%%%%%%%%%%%%%%%%%%%%%%%%%%%%%%%%%%%%%%%%%%%%%%%%%%%%%%%%%%%% 
%%%%%%%%%%%%%%%%%%%%%%%%%%%%%%%%%%%%%%%%%%%%%%%%%%%%%%%%%%%%%%%%%%%%%%%% 
\begin{frame}
  \frametitle{Trilinos/Tpetra example project}

  \begin{itemize}
  \item Directory \texttt{doc/trilinos/tpetra\_example} contains a minimal example application for trilinos/tpetra. You just need to set env variable \texttt{TRILINOS\_PATH} to install directory.
  \end{itemize}
  
\end{frame}


\subsection{Custom monitoring / intrumenting / profiling}
%%%%%%%%%%%%%%%%%%%%%%%%%%%%%%%%%%%%%%%%%%%%%%%%%%%%%%%%%%%%%%%%%%%%%%%% 
%%%%%%%%%%%%%%%%%%%%%%%%%%%%%%%%%%%%%%%%%%%%%%%%%%%%%%%%%%%%%%%%%%%%%%%% 
\begin{frame}[fragile=singleslide]
  \frametitle{Kokkos profiling interface (1)}

  \begin{itemize}
  \item Kokkos provides by default a \textcolor{violet}{profiling interface} through a {\bf plugin mechanism}
  \item {\bf Usage: \textcolor{darkgreen}{profiling / monitoring / instrumenting}}
  \item From an application point of view, there is nothing to do, just provide a plugin (shared library), e.g.
    \begin{minted}{bash}
      # define path to the plugin
      export KOKKOS_PROFILE_LIBRARY=/somewhere/kp_kernel_logger.so
      # run as usal Kokkos application
    \end{minted}
  \item Examples of Kokkos profile plugins can be found at\\
    \myurl{https://github.com/kokkos/kokkos-tools}
  \end{itemize}

\end{frame}

%%%%%%%%%%%%%%%%%%%%%%%%%%%%%%%%%%%%%%%%%%%%%%%%%%%%%%%%%%%%%%%%%%%%%%%% 
%%%%%%%%%%%%%%%%%%%%%%%%%%%%%%%%%%%%%%%%%%%%%%%%%%%%%%%%%%%%%%%%%%%%%%%% 
\begin{frame}[fragile=singleslide]
  \frametitle{Kokkos profiling interface (2)}

  \begin{itemize}
  \item A Kokkos profile plugin must provide implementation for callback routines
    \begin{itemize}
    \item \texttt{kokkosp\_init\_library}
    \item \texttt{kokkosp\_finalize\_library}
    \end{itemize}
  \item A Kokkos profile interface can provide implementation for callback routines specific to a type a parallel construct, e.g. \texttt{Kokkos::parallel\_for}
    \begin{itemize}
    \item \texttt{kokkosp\_begin\_parallel\_for}
    \item \texttt{kokkosp\_end\_parallel\_for}
    \end{itemize}
    which are called every time application enters / exits this construct.
    \item see file \texttt{core/src/impl/Kokkos\_Profiling\_Interface.cpp} for a detailed list of possible callbacks.
  \end{itemize}
  
\end{frame}

% additionnal stuff
%%%%%%%%%%%%%%%%%%%%%%%%%%%%%%%%%%%%%%%%%%%%%%%% 
%%%%%%%%%%%%%%%%%%%%%%%%%%%%%%%%%%%%%%%%%%%%%%%% 
\begin{frame}
  \frametitle{Evaluating Peak Memory bandwidth}

  \textcolor{violet}{\Large How to determine the {\bf peak} hardware memory bandwidth of your compute platform ?}

  \begin{itemize}
  \item \textcolor{red}{\bf Multicore CPU} (e.g. Intel Skylake):
    \begin{itemize}
    \item Memory type ? e.g. DDR4-2666
    \item Number of channels ? e.g. 6
    \item Max $BW =$ \# NbOfChannel $\times$ Frequency(GHz) $\times$ BusWidth/8 (Bytes) $\times$ \# NbOfSockets
    \item e.g. on \myhref{http://www-hpc.cea.fr/fr/complexe/tgcc-Irene.htm}{TGCC/IRENE}, BW = 6 $\times$ 2.6 $\times$ 64/8 $\times$ 2 = 256 GBytes per node
    \end{itemize}
  \item \textcolor{orange}{\bf Manycore CPU} (e.g. Intel KNL):
    \begin{itemize}
    \item depends on \myhref{https://colfaxresearch.com/knl-mcdram/}{HBM configuration} (CACHE, FLAT, HYBRID)
    \item e.g. KNL on TGCC/IRENE configured in CACHE mode, BW $\geqslant 400$ GBytes/s
    \end{itemize}
  \item \textcolor{blue}{\bf NVIDIA GPU} (e.g. Pascal P100):
    \begin{itemize}
    \item Use sample application {\tt deviceQuery} to retrieve hardware spec.
    \item \# Memory Clock rate:  715 Mhz
    \item \# Memory Bus Width:   4096-bit
    \item $BW =$ 732.1 Gbytes/s
    \end{itemize}
  \item \textcolor{RoyalPurple}{\bf NVIDIA GPU} (e.g. Pascal V100):
    \begin{itemize}
    \item $BW =$ 898.0 Gbytes/s
    \end{itemize}
  \end{itemize}

% ref on KNL : http://sites.utexas.edu/jdm4372/tag/memory-bandwidth/
  
\end{frame}

%%%%%%%%%%%%%%%%%%%%%%%%%%%%%%%%%%%%%%%%%%%%%%%% 
%%%%%%%%%%%%%%%%%%%%%%%%%%%%%%%%%%%%%%%%%%%%%%%% 
\begin{frame}
  \frametitle{Evaluating Peak Memory bandwidth}

  \textcolor{violet}{\Large What about {\bf achievable} memory bandwidth ?}

  \begin{itemize}
  \item Use the stream benchmark. e.g. \myurl{https://github.com/UoB-HPC/BabelStream}
    \begin{itemize}
    \item Copy : $C[i] = A[i]$
    \item Trias : $A[i] = B[i] + scalar * C[i]$
    \end{itemize}
  \item On \textcolor{red}{\bf TGCC/Irene/Skylake} (2 sockets per node), one can measure:
    \begin{itemize}
    \item copy: $190$ GBytes/s (74 \% of peak)
    \item triad: $160$ GBytes/s (63 \% of peak)
    \end{itemize}
  \item On \textcolor{orange}{\bf TGCC/Irene/KNL} (1 socket), one can measure:
    \begin{itemize}
    \item copy: $260$ GBytes/s (65 \% of peak)
    \item triad: $330$ GBytes/s (80 \% of peak)
    \end{itemize}
  \item On \textcolor{blue}{\bf NVIDIA P100}:
    \begin{itemize}
    \item copy: $530$ GBytes/s (72 \% of peak)
    \item triad: $550$ GBytes/s (75\% of peak)
    \end{itemize}    
  \item On \textcolor{RoyalPurple}{\bf NVIDIA V100}:
    \begin{itemize}
    \item copy: $650$ GBytes/s (72 \% of peak)
    \item triad: $860$ GBytes/s (95\% of peak)
    \end{itemize}    
  \end{itemize}

\end{frame}

%%%%%%%%%%%%%%%%%%%%%%%%%%%%%%%%%%%%%%%%%%%%%%%%%%%%%%%%%%%%%%%%%%%%%%%%
%%%%%%%%%%%%%%%%%%%%%%%%%%%%%%%%%%%%%%%%%%%%%%%%%%%%%%%%%%%%%%%%%%%%%%%%
\begin{frame}[fragile=singleslide]
  \frametitle{Kokkos for Cuda users}

  From a pure software engineering point of view, how does {\bf Kokkos} manage to turn \textcolor{red}{\bf a pur C++ functor} into a \textcolor{darkgreen}{\bf CUDA kernel} ?

  \begin{enumerate}
    % explain from core/src/Kokkos_Parallel.hpp
    % explain from core/src/Cuda/Kokkos_Cuda_Parallel.hpp
  \item entry point of parallel computation is through \texttt{parallel\_for} (function call, templated by execution policy, functor, ...)
    \begin{minted}[autogobble=true]{c++}
      // parallel_for is defined in
      // core/src/Kokkos_Parallel.hpp : line 200
      template< class FunctorType >
      inline
      void parallel_for( const size_t       work_count
                       , const FunctorType& functor
                       , const std::string& str = ""
                       )
      {
        // ...
        Impl::ParallelFor< FunctorType , policy >
        closure( functor , policy(0,work_count) );
        // ...
      }
    \end{minted}
  \end{enumerate}

\end{frame}

%%%%%%%%%%%%%%%%%%%%%%%%%%%%%%%%%%%%%%%%%%%%%%%%%%%%%%%%%%%%%%%%%%%%%%%%
%%%%%%%%%%%%%%%%%%%%%%%%%%%%%%%%%%%%%%%%%%%%%%%%%%%%%%%%%%%%%%%%%%%%%%%%
\begin{frame}[fragile=singleslide]
  \frametitle{Kokkos for Cuda users}

  %From a pure software engineering point of view, how does {\bf Kokkos} manage to turn \textcolor{red}{\bf a pur C++ functor} into a \textcolor{darkgreen}{\bf CUDA kernel} ?

  \begin{enumerate}
    \setcounter{enumi}{1}
  \item \texttt{closure} is an instance of the {\bf driver} class \texttt{Kokkos::Impl::ParallelFor}; the precise object type created is off course Kokkos-backend dependent
    % Kokkos_Parallel.hpp : line 221
    % Impl::ParallelFor< FunctorType , policy > closure( functor , policy(0,work_count) );
  \item If CUDA backend is activated, the instantiated class \texttt{Kokkos::Impl::ParallelFor} is defined in \texttt{Cuda/Kokkos\_Cuda\_Parallel.hpp}; there are multiple specialization for the different execution policies (Range, multi-dimensional range, team policy, ...); e.g. for range
    {\scriptsize
      \begin{minted}[autogobble=true]{c++}
        template< class FunctorType , class ... Traits >
        class ParallelFor< FunctorType
                         , Kokkos::RangePolicy< Traits ... >
                         , Kokkos::Cuda
                         >
        {
          // this is where for a given iteration id, the functor is called
          // kind of generic cuda kernel work definition
          inline   __device__  void operator()(void) const { ... };

          // this is where the actual CUDA kernel run time config
          // is setup : block and grid dimension
          // then create a CudaParallelLaunch object
          inline void execute() const { ... };
        }

      \end{minted}
  }
  \end{enumerate}

\end{frame}

%%%%%%%%%%%%%%%%%%%%%%%%%%%%%%%%%%%%%%%%%%%%%%%%%%%%%%%%%%%%%%%%%%%%%%%%
%%%%%%%%%%%%%%%%%%%%%%%%%%%%%%%%%%%%%%%%%%%%%%%%%%%%%%%%%%%%%%%%%%%%%%%%
\begin{frame}[fragile=singleslide]
  \frametitle{Kokkos for Cuda users}

  %From a pure software engineering point of view, how does {\bf Kokkos} manage to turn \textcolor{red}{\bf a pur C++ functor} into a \textcolor{darkgreen}{\bf CUDA kernel} ?

  \begin{enumerate}
    \setcounter{enumi}{3}
  \item when \texttt{closure.execute()} is called, an object \texttt{CudaParallellaunch} is created
  \item struct \texttt{CudaParallellaunch} contains only a constructor, which only purpose is to actually launch the CUDA kernel (using the $<<<$ ... $>>>$ syntax)
  \item Copy closure (driver instance) to GPU memory (either constant, local or global) using Cuda API (e.g \texttt{cudaMemcpyToSymbolAsync} to copy constant memory space)
  \item finally the actual generated cuda kernel, using one of the static functions defined (e.g. \texttt{cuda\_parallel\_launch\_constant\_memory})
    % see Kokkos_Cuda_KernelLaunch.hpp
  \end{enumerate}

\end{frame}


\end{document}

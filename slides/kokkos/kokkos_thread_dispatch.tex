%%%%%%%%%%%%%%%%%%%%%%%%%%%%%%%%%%%%%%%%%%%%%%%%%%%%%%%%%%%%%%%%%%%%%%%% 
%%%%%%%%%%%%%%%%%%%%%%%%%%%%%%%%%%%%%%%%%%%%%%%%%%%%%%%%%%%%%%%%%%%%%%%% 
\begin{frame}[fragile=singleslide]
  \frametitle{Kokkos compute Kernels}

  \begin{itemize}
  \item How to specify a compute kernel in Kokkos ?
    \begin{enumerate}
    \item \textcolor{blue}{\textbf{Use Lambda functions.}}\\
      NB: a lambda in c++11 is an unnamed function object capable of capturing variables in scope.
      \begin{minted}{c++}
        Kokkos::parallel_for (100, KOKKOS_LAMBDA (const int i) {
          data(i) = 2*i;
        });
      \end{minted}
      Here we do 2 things in 1 step: define the computation body (lambda func) and launch computation.
    \item \textcolor{darkgreen}{\textbf{Use a C++ functor class.}}\\
      A functor is a class containing a function to execute in parallel.
      \begin{minted}{c++}
        class FunctorType {
          public:
          KOKKOS_INLINE_FUNCTION
          void operator() ( const int i ) const ;
        };
        ...
        FunctorType func;
        Kokkos::parallel_for (100, func);
      \end{minted}
    \end{enumerate}
  \end{itemize}

\end{frame}


%%%%%%%%%%%%%%%%%%%%%%%%%%%%%%%%%%%%%%%%%%%%%%%%%%%%%%%%%%%%%%%%%%%%%%%% 
%%%%%%%%%%%%%%%%%%%%%%%%%%%%%%%%%%%%%%%%%%%%%%%%%%%%%%%%%%%%%%%%%%%%%%%% 
\begin{frame}[fragile=singleslide]
  \frametitle{Kokkos compute Kernels}

  \textbf{Lambda or Functor: which one to use in Kokkos ? Both !}
  \begin{enumerate}
  \item \textcolor{blue}{\textbf{Use Lambda functions.}}\\
    \begin{itemize}
    \item easy way for small compute kernels
    \item For GPU, requires Cuda 7.5 (8.0 is current and latest CUDA version)
    \end{itemize}
  \item \textcolor{darkgreen}{\textbf{Use a C++ functor class.}}\\
    \begin{itemize}
    \item More flexible, allow to design more complex kernel
    \end{itemize}
  \end{enumerate}
\end{frame}
